\section{Volvo Cars architecture framework}\label{sec:VCGAF}

The starting point for defining an architecture framework is to start from the identification of established stakeholders within the domain of the framework. Stakeholders may be individuals, teams,
organizations or classes (of individuals, teams or organizations), while concerns may be fine-grained or very broad in scope~\cite{Emery-Hilliard:2009}. 
Then the identified stakeholders motivate the set of concerns on which the architecture framework will focus.
This will help the consumer of the architecture framework and of the views and connected modeling tools to understand why
they are modeling and when they are done.

The identified architecture-related concerns determine the choice of viewpoints and view to be included. 
It is important to note that almost each viewpoint detained in the following is already handled within the current architecture of Volvo Cars. However, we are investigating the definition of proper viewpoints that will create the architecture framework. %of the
%Viewpoints are the principal content of an architecture framework. In fact the viewpoint defines the conventions and establishes the basis for interpreting views. Moreover, ``{\em each viewpoint establishes the notations, models, techniques and methods to be used in architecture descriptions resulting from applying the framework.}"~\cite{Emery-Hilliard:2009}.
Therefore, in addition to the viewpoints summarized in Section~\ref{sec:AAF} we foresee the following viewpoints:
	\begin{itemize}
	\item \emph{Continuous integration and deployment} (detailed in Section~\ref{sec:CIDviewpoint}) - OEMs are increasingly interested to reduce the development time, to increase flexibility, to have early feedback on decisions made, and to add new functionalities incrementally even after production. %However, another trend is the rapid development of more advanced active safety systems that require \del{a} special handling.\eric{does the last point here not relate more to connected cars and safety?}
	\item \emph{Connected cars and safety} (detailed in Section~\ref{sec:SoSviewpoint}) - Future scenarios of collaborating autonomous vehicles, like platooning, will require to extend the vehicle safety architecture across the classical boundaries of single vehicles and will ask for an open and adaptive architecture able to support runtime assessment of safety. 
	\item \emph{Security and privacy of connected cars} - Connected cars open new important challenges from the point of view of security and privacy.
	\item \emph{Ecosystem and transparency} - related to the value net viewpoint of AAF~\cite{TUM-I0915,Broy}. %activities of, and the dependencies between the stakeholders of the end-toend value creation process which happens within a specific value net. 
	The ecosystem around the OEM can be seen as a virtual %A value net is defined as a virtual
organization consisting of the OEM, its suppliers and other partners %and other constituents 
involved in the process of creating customer value.
	\item \emph{Autonomous cars} - autonomous cars require special architecture solutions, e.g. inspired to autonomous and self-adaptive systems~\cite{Salehie2009}.
	\item \emph{Modes management} - a mode viewpoint is needed to design the different modes of a vehicle as well as the transitions from one mode to another.
	\item Special viewpoints and views might be conceived to enable dissemination and communication of the architecture to developers and other stakeholders.
	\end{itemize}

%In order to define the viewpoints we use a template similar to the one suggested in~\cite{Yania}:
%\begin{itemize}
%\item Definition: Definition of the viewpoint is presented.
%\item Stakeholders: Although the stakeholders are not explicitly identified for the viewpoints
%in the AAF and ADF, we list the stakeholders.
%\item Concerns: Stakeholder concerns are defined.
%\item Views: The views governed by the viewpoints are presented.
%\item Model kinds: The model kinds used in the viewpoint are presented.
%\end{itemize}



The natural consequence of the use of multiple viewpoints and views in architecture
descriptions is the need to express correspondences and consistency rules between those views.
The mechanisms introduced in~\cite{42010} is called model correspondences and it allows the definition of relations between
two or more architecture models. Since architecture views are composed of architecture models~\cite{42010}, model correspondences can be used to
relate views to express consistency, traceability, refinement or other dependencies~\cite{Emery-Hilliard:2009}.
These mechanisms allow an architect to impose constraints between types of models and then demonstrate that them
are satisfied by the architecture. 



%
%
%
%Architecture descriptions take many forms and
%serve many purposes throughout the life cycle of development,
%operation and maintenance activities. The use of {\em multiple views}
%- diverse representations for distinct audiences and uses - has
%been a major tenet of architecture description since the earliest
%work in software architecture. 
%The use of multiple views has become standard practice
%in industry~\cite{4+1,ICSE2010,42010}. A survey recently conducted
%on the industrial needs from architectural languages~\cite{whatindustrywants} (see Section~\ref{sect:survey})
%revealed that 85\% of the 48 interviewed practitioners use
%multiple views when architecting a software system, with
%a total of nine different views and a predominant use of
%structural, behavioral, and physical views reported. 


\section{Continuous integration and deployment viewpoint}\label{sec:CIDviewpoint}

\subsection{Overview}
Agile approaches and practices such as continuous integration and deployment promise to help reducing development time, to increase flexibility, and to generally shorten the feedback cycle time. However, 
the  complex supplier network,
and typical setup with a large number of ECUs,
pose specific challenges to %\chg{agile development methods, and specifically 
%with respect 
%to continuous integration and deployment of software.}{
these practices. %}.

First,  dependencies between ECUs raise multiple concerns,
regarding organization, versioning and testing:
(i)  organization -
%the question is related to who should be 
identifying the recipient
of a given software change; (ii)
 versioning -
the question is related to the compatibility of the software version of specific ECUs; and
% software
%that are compatible.
(iii)  testing -  %This then carries over to the testing effort,
compatible combinations need to be validated. 
Second, support for continuous deployment has to face with safety concerns.
%relates to the connectivity of ECUs. 
%the extent of this, in turn, raises not least security concerns.
Should, for instance, the software of an ECU responsible for a safety critical function
be modifiable at runtime?

Dependencies between ECUs are a property of the architecture.
As mentioned, the emergent architecture may differ from the intended architecture,
and continuous integration and deployment of software may entail architectural changes.
This highlights both the need for collaboration % touches on the concern of the need for collaboration
between parts of the organization working on different architectural levels, and the need of a proper support
for agile and flexible architecting.
%Furthermore, the architectural framework should support agile architecting.

Addressing these concerns suggests
two architectural views and viewpoints:
(i) one covering architecture as an enabler
of continuous integration and deployment,
facilitating variant handling and coordination of updates, and
%
(ii) another considering continuous integration and deployment
on the architecture level,
facilitating reasoning about modifications to the architecture itself.

\subsection{Concerns}
\patrizio{A listing of the architecture-related concerns framed by this viewpoint. This is
crucial information for the Architect, because it helps her decide whether this
viewpoint will be useful to apply to a given system of interest, and to
communicate with its stakeholders.}

\subsection{Anti-concerns}
\patrizio{Optional. It can be useful to document the kinds of issues a viewpoint is not
appropriate for. Articulating anti‐concerns may be a good antidote for certain
overly used notations.}

\subsection{Typical stakeholders}
\patrizio{Optional. The typical audiences for views prepared using this viewpoint. Who
are the usual stakeholders for this kind of view?}

\subsection{Model languages}
\patrizio{For each type of model used, describe the language or modeling techniques to
be used. Each model language is a key modeling resource that the viewpoint
makes available. Model languages provide the vocabularies for constructing the
view. ISO/IEC 42010 does not specify how a modeling language is documented.
It could be by reference to an existing modeling language (e.g., EAST-ADL or UML)
or technique (e.g., M/M/4 queues); by providing a metamodel for the language
to define the language's core constructs; via a template that users fill in; or by
some combination of these methods.}

\subsection{Model correspondence rules}
\patrizio{The viewpoint may specify model correspondence rules. Each one may be
documented here.}

\subsection{Operations on views}
\patrizio{Operations define the methods which may be applied to views and their
models. Operations can be divided into categories: Creation methods are the
means by which views are prepared using the viewpoint. These could be in the
form of process guidance (how to start, what to do next); or work product
guidance (templates for views of this type); heuristics, styles, patterns, or other
idioms. Interpretive methods provide the means by which views are to be
understood by readers and system stakeholders. Analysis methods are used to
check, reason about, transform, predict, apply and evaluate architectural
results from this view. Implementation methods capture how to realize or
construct systems using information from this view.}

\subsection{Examples}
\patrizio{Optional. This section provides examples for the reader.}

\subsection{Sources}
\patrizio{What are the sources for this viewpoint, if any? This may include author,
history, literature references, prior art, etc.}

\section{Connected cars and Safety viewpoint}\label{sec:SoSviewpoint}

\subsection{Overview}
Connected cars will benefit from Intelligent Transport Systems (ITS), Smart Cities and IoT to provide new application scenarios like smart traffic control,  smart platooning coordination, collective collision avoidance, etc. 
Vehicles will combine data collected through its sensors %from the inside vehicle 
with external data coming from the environment, e.g. other vehicles, road, cloud, etc. %In such scenario, different applications will be possible: smart traffic control, better platooning coordination and enhanced safety in general. 
%While solving congestion is undoubtedly beneficial, safety remains the top priority for OEMs.

%In connected vehicles, safety aspects become more complex. 
Connected vehicles will face new challenges and opportunities related to safety issues.
%On the one hand, c
Current %While there are clear 
regulations for safety aspects, like the ISO 26262 standard, do not account for scenarios in which the %provide a clear regulation for  
% regarding isolated car system (seat belts, airbags etc..) like in the safety standard ISO 26262, there is a lack of 
%safety requirements for 
%systems of connected cars. The 
vehicle is part of a more complex system; the challenge is on how to manage new hazards that coming from the environment could jeopardize safety. 
%The core ECUs of the car (in charge of braking, steering, engine control, etc.) are part of a closed system in which all the safety requirements can be certified at design time. Exposing these crucial components to the outside environment would invalidates the assumptions that have been made so far. In fact, manufactures have preferred to limit the interfaces of the communication infrastructure due to the enormous hazard potential. In a connected world,  this is no longer possible. Sharing and controlling sensitive information could be crucial in dangerous situations. 
%On the other hand, c
At the same time, connected cars open new opportunities for safety, called ``connected safety" within Volvo Cars\footnote{\url{https://goo.gl/mIWWS3}}: %({\small \url{https://goo.gl/mIWWS3}}):
e.g., this new technology will allow a connected car to be aware of a slippery road, of cyclists on the road, etc., so to initiate all the actions needed to avoid accidents and collisions. 
%
%after it has been informed by another car and the information has been propagated via the Volvo Cloud. In such scenario, the informed car could automatically slow down reducing the risk for potential accidents. 

These scenarios are posing new requirements to the architecture. %To deal with similar scenarios, changes need to be made in the architecture. 
We foreseen two different viewpoints and views: (i) %one viewpoint and view showing the architecture 
from the point of view of a single connected car, which has to offer the functionalities needed to realize the scenario, and (ii) %another viewpoint and view representing the 
from the point of view of the system of systems  (i.e. cars connected with other entities of their environment), spanning from the agreements between the different systems affected, e.g. different OEMs, cloud providers, road infrastructures, etc., to the definition of the scenario that the system of systems has to realize, like the slippery road mentioned above. 

% agreements between the different systems affected. In order to achieve the proper behavior of such system different stakeholders must be involved. From tier 1 and tier 2 suppliers to the cloud provider, they all have to agree on common interfaces and safety guarantees that must be respected.

%The core ECUs of the car (in charge of braking, steering, engine control, etc.) are part of a closed system in which all the safety requirements can be certified at design time. Exposing these crucial components to the outside environment would invalidates the assumptions that have been made so far. In fact, manufactures have preferred to limit the interfaces of the communication infrastructure due to the enormous hazard potential. In a connected world,  this is no longer possible. Sharing and controlling sensitive information could be crucial in dangerous situations. 

%Until lately, safety has been dedicated in surviving crashes, with the emerging connected vehicle it will be about avoiding them. For example, the new technologies developed by Volvo allow a connected car to be aware of a slippery road after it has been informed by another car and the information has been propagated via the Volvo Cloud. In such scenario, the informed car could automatically slow down reducing the risk for potential accidents.

%To accomplish similar scenarios, radical changes need to be made in the architecture descriptions and agreements between the different systems affected. In order to achieve the proper behavior of such system different stakeholders must be involved. From tier 1 and tier 2 suppliers to the cloud provider, they all have to agree on common interfaces and safety guarantees that must be respected.

%<<<<<<< HEAD
In general, the main issue with safety guarantees in connected cars is that a full analysis of the system is not possible at design time. When moving from a single vehicle to a cooperative system, a new safety analysis is required to handle uncertainties at runtime. Some approaches have been proposed to deal with certification at runtime, e.g.~\cite{runtime1, runtime3}, but a clear framework that can be used to define the connected safety requirements is still missing.
%=======
%The main issue with safety guarantees in connected cars, is that a full analysis of the system is not possible at design time. When moving from a single vehicle to a cooperative system, a new safety analysis is required to handle the uncertainties at runtime. Therefore, in order to benefit from the collaborative system, is necessary to move some of the safety analysis from design time to runtime. Some approaches have been proposed to deals with certification of the safety guarantees at runtime \cite{runtime1}, \cite{runtime2}, \cite{runtime3}, but there is still not a clear framework utilized to define the connected safety requirements.
%>>>>>>> 4ffb434a25eb0042a3685e7d03d264e7c0eb62b1

\subsection{Concerns}
\patrizio{A listing of the architecture-related concerns framed by this viewpoint. This is
crucial information for the Architect, because it helps her decide whether this
viewpoint will be useful to apply to a given system of interest, and to
communicate with its stakeholders.}

\subsection{Anti-concerns}
\patrizio{Optional. It can be useful to document the kinds of issues a viewpoint is not
appropriate for. Articulating anti‐concerns may be a good antidote for certain
overly used notations.}

\subsection{Typical stakeholders}
\patrizio{Optional. The typical audiences for views prepared using this viewpoint. Who
are the usual stakeholders for this kind of view?}

\subsection{Model languages}
\patrizio{For each type of model used, describe the language or modeling techniques to
be used. Each model language is a key modeling resource that the viewpoint
makes available. Model languages provide the vocabularies for constructing the
view. ISO/IEC 42010 does not specify how a modeling language is documented.
It could be by reference to an existing modeling language (e.g., EAST-ADL or UML)
or technique (e.g., M/M/4 queues); by providing a metamodel for the language
to define the language's core constructs; via a template that users fill in; or by
some combination of these methods.}

\subsection{Model correspondence rules}
\patrizio{The viewpoint may specify model correspondence rules. Each one may be
documented here.}

\subsection{Operations on views}
\patrizio{Operations define the methods which may be applied to views and their
models. Operations can be divided into categories: Creation methods are the
means by which views are prepared using the viewpoint. These could be in the
form of process guidance (how to start, what to do next); or work product
guidance (templates for views of this type); heuristics, styles, patterns, or other
idioms. Interpretive methods provide the means by which views are to be
understood by readers and system stakeholders. Analysis methods are used to
check, reason about, transform, predict, apply and evaluate architectural
results from this view. Implementation methods capture how to realize or
construct systems using information from this view.}

\subsection{Examples}
\patrizio{Optional. This section provides examples for the reader.}

\subsection{Sources}
\patrizio{What are the sources for this viewpoint, if any? This may include author,
history, literature references, prior art, etc.}

\section{Ecosystem and transparency viewpoint}

\subsection{Overview}

\subsection{Concerns}
\patrizio{A listing of the architecture-related concerns framed by this viewpoint. This is
crucial information for the Architect, because it helps her decide whether this
viewpoint will be useful to apply to a given system of interest, and to
communicate with its stakeholders.}

\subsection{Anti-concerns}
\patrizio{Optional. It can be useful to document the kinds of issues a viewpoint is not
appropriate for. Articulating anti‐concerns may be a good antidote for certain
overly used notations.}

\subsection{Typical stakeholders}
\patrizio{Optional. The typical audiences for views prepared using this viewpoint. Who
are the usual stakeholders for this kind of view?}

\subsection{Model languages}
\patrizio{For each type of model used, describe the language or modeling techniques to
be used. Each model language is a key modeling resource that the viewpoint
makes available. Model languages provide the vocabularies for constructing the
view. ISO/IEC 42010 does not specify how a modeling language is documented.
It could be by reference to an existing modeling language (e.g., EAST-ADL or UML)
or technique (e.g., M/M/4 queues); by providing a metamodel for the language
to define the language's core constructs; via a template that users fill in; or by
some combination of these methods.}

\subsection{Model correspondence rules}
\patrizio{The viewpoint may specify model correspondence rules. Each one may be
documented here.}

\subsection{Operations on views}
\patrizio{Operations define the methods which may be applied to views and their
models. Operations can be divided into categories: Creation methods are the
means by which views are prepared using the viewpoint. These could be in the
form of process guidance (how to start, what to do next); or work product
guidance (templates for views of this type); heuristics, styles, patterns, or other
idioms. Interpretive methods provide the means by which views are to be
understood by readers and system stakeholders. Analysis methods are used to
check, reason about, transform, predict, apply and evaluate architectural
results from this view. Implementation methods capture how to realize or
construct systems using information from this view.}

\subsection{Examples}
\patrizio{Optional. This section provides examples for the reader.}

\subsection{Sources}
\patrizio{What are the sources for this viewpoint, if any? This may include author,
history, literature references, prior art, etc.}