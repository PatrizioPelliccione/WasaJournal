% !TEX root = main.tex
%%% Version 2.1b %%%
\section{Ecosystem and transparency viewpoint}\label{sec:ET_VP}
\renewcommand{\Fillin}[1]{{Ecosystem and Transparency}}
%%%%%%%%%%
\subsection{\Fillin{Viewpoint Name}}\label{vp:eco}
%%%%%%%%%%

\todo[inline,size=\small]{Provide the name for the viewpoint.

If there are any synonyms or other common names by which this viewpoint is
known or used, record them here.}

The viewpoint ``\Fillin{Viewpoint Name}'' focuses on the value-chain of \ugh{logical components of the  architecture}\eric{is this a good way of describing what is delivered by the ecosystem? It should be some abstraction of hw, software, logical components, no?} and aims at answering the following questions:
\begin{enumerate}
\item How can an electrical architecture enable a network of organizations (including OEM, Tier-1 supplier, Tier-2 supplier) to work together to (continuously) provide value?
\item How are architectural decisions affected by the need for transparent and continuous collaboration within the value-chain?
\end{enumerate}
\eric{Really struggling with those questions...}

%%%%%%%%%%
\subsection{Overview} 
%%%%%%%%%%

\todo[inline,size=\small]{Provide an abstract or brief overview of the viewpoint. 

Describe the viewpoint's key features.}

The \Fillin{Viewpoint Name} viewpoint takes into account that architectural assets in automotive engineering are provided based on a complex network of cooperating organizations. 
A logical component in the architecture will be implemented by a variaty of physical components, including hardware (ECUs) and software on different levels of abstraction. 

In a concrete scenario, a logical component could be based on several AUTOSAR ECUs.
The hardware in this scenario would be provided by different Tier-1 suppliers, while the AUTOSAR layer would be provided by one or more certified AUTOSAR suppliers.
These AUTOSAR suppliers could be characterised as Tier-2 suppliers, since they are contracted by the Tier-1 suppliers who are supposed to deliver an ECU with basic software, but different setups are possible. 
On top of these ECUs, application software would be developed either by the OEM or by separate software suppliers in order to provide the services defined by a logical component. 

The example scenario above shows the importance of the \Fillin{Viewpoint Name} viewpoint: if an electrical architecture should take into account strategic business goals such as improved flexibility, reduced time-to-market, or development efficiency, the \Fillin{Viewpoint Name} viewpoint will offer important information that must be considered.
This viewpoint ties into business decisions and often long-running sourcing contracts, but also into new business models with respect to after market software updates.
Thus, we expect that lifecycles of hardware and software components are important architectural aspects as well as managing different development cycles of suppliers and components.


%%%%%%%%%%
\subsection{Concerns and stakeholders} 
%%%%%%%%%%

\todo[inline, size=\small]{Architects looking for an architecture viewpoint suitable for their
purposes often use the identified concerns and typical stakeholders to
guide them in their search.  Therefore it is important (and required
by the Standard) to document the concerns and stakeholders for which a
viewpoint is intended.}

%%%%%%%%%%
\subsubsection{Concerns}\label{vp:concerns}
%%%%%%%%%%
In this section we focus on the concerns that are essential to enable efficient work in the ecosystem of automotive electrical systems. 
We express the concerns in the form of questions as suggested by the ISO/IEC/IEEE 42010 standard.
\eric{Wonder if it is good or bad that all three viewpoints have very similar text here...}

\begin{itemize}
\item \emph{Which types of value-chains are implied by a given electrical architecture and what is their purpose?} A given electrical architecture will define the interplay of different types of logical components with the goal to support user visible features. 
While some of these features will be well-understood and stable, others will be highly innovative and their development will be subject to volatile requirements. 
Value-chains for implementing those logical components will differ based on the type of feature that they support. 

\item \emph{How to map supplier development capabilities to demands created by a specific electrical architecture?} Different capabilities are demanded for development across the value-chain, depending of the type of feature. 
An electrical architecture might be able to minimize or limit interaction on critical innovative features in the ecosystem to only actors that can support a highly iterative development. 

\item \emph{What is the suitability of the architecture for managing different kinds of value-chains?}

\item \emph{How can we establish the required level of transparency in a value-chain?} The more volatile the requirements of a feature are, the more information exchange is needed between ecosystem actors. 
If we were to implement continuous delivery throughout the value-chain, all involved partners should have access to critical information so that they can take responsibility. 
In a less volatile environment the (legal and organizational) effort of creating high transparency will not pay off. 

\item \emph{How can we manage transparency (e.g. of architectural decisions) in the face of changing suppliers?} In order to maintain a powerful automotive engineering ecosystem we need to be able to change actor relationships. This will also affect the level of transparency (including what information to share and at what frequency).
\end{itemize}

\todo[inline, size=\footnotesize]{\must{Provide a listing of architecture-relevant concerns to be framed by
this architecture viewpoint per \std{7a}.}

Describe each concern.

Concerns name ``areas of interest'' in a system.

\note{Following ISO/IEC/IEEE 42010, \textbf{system} is a shorthand for
  any number of things including man-made systems, software products
  and services, and software-intensive systems such as ``individual
  applications, systems in the traditional sense, subsystems, systems
  of systems, product lines, product families, whole enterprises, and
  other aggregations of interest''.}

Concerns may be very general (e.g., \textit{Reliability}) or quite
specific (\textit{e.g., How does the system handle network latency?}).
  
Concerns identified in this section are critical information for an
architect because they help her decide when this viewpoint will be
useful.

When used in an architecture description, the viewpoint becomes a
``contract'' between the architect and stakeholders that these
concerns will be addressed in the view resulting from this viewpoint.

It can be helpful to express concerns \emph{in the form of questions}
that views resulting from that viewpoint will be able to answer. E.g.,

~~--~\textit{How does the system manage faults?}

~~--~\textit{What services does the system provide?}


\note{``In the form of a question'' is inspired by the television quiz
  show, \textit{Jeopardy!}}
 
\std{5.3} contains a candidate list of concerns that must be considered
when producing an architecture description. These can be considered
here for their relevance to the viewpoint being specified:
%\begin{itemize}

~~--~ What are the purpose(s) of the system-of-interest?

~~--~What is the suitability of the architecture for achieving the
  system-of-interest's purpose(s)?

~~--~How feasible is it to construct and deploy the
  system-of-interest?

~~--~What are the potential risks and impacts of the
  system-of-interest to its stakeholders throughout its life cycle?

~~--~How is the system-of-interest to be maintained and evolved?

See also: \std{4.2.3}.}

%%%%%%%%%%
\subsubsection{Typical stakeholders} 
%%%%%%%%%%
For the ecosystem and transparency viewpoint, we need to consider stakeholders within the different ecosystem actors.
With respect to the OEM (the keystone actor in the automotive engineering ecosystem), potentially every stakeholder described in Section~\ref{sec:VCGAF} and summarized in Table~\ref{tab:stakeholders} will be part of the audience of this viewpoint. 

Other actors (such as Tier-1 and Tier-2 suppliers) will have very similar stakeholders that need to be considered. 
Those need to be taken into account, even when constructing this viewpoint strictly from the perspective of the OEM. 
For example for the concerns above, it is crucial to consider both managers and technical experts at suppliers.

\todo[inline, size=\footnotesize]{
\must{Provide a listing of the typical stakeholders of a system who
  are in the potential audience for views of this kind, per \std{7b}.}

Typical stakeholders would include those likely to read such views
and/or those who need to use the results of this view for another
task.

Stakeholders to consider include:
%\begin{itemize}

~~--~users of a system; 

~~--~operators of a system; 

~~--~acquirers of a system;

~~--~owners of a system; 

~~--~suppliers of a system; 

~~--~developers of a system; 

~~--~builders of a system; 

~~--~maintainers of a system.
%\end{itemize}
}
%%%%%%%%%%
%\subsubsection{``Anti-concerns'' \Optional} 
%%%%%%%%%%
%\todo[inline, size=\footnotesize]{
%It may be helpful to architects and stakeholders to
%document the kinds of issues for which this viewpoint is \emph{not
%  appropriate or not particularly useful}.

%Identifying the ``anti-concerns'' of a given notation or approach may
%be a good antidote for certain overly used models and notations.
%}
% \tbd{Examples!}



%%%%%%%%%%
\subsection{Model kinds+}\label{mk:list}
%%%%%%%%%%
\todo[inline, size=\footnotesize]{

\must{Identify each model kind used in the viewpoint per \std{7c}.}

In the Standard, each architecture view consists of multiple
architecture models. Each model is governed by a \textit{model kind}
which establishes the notations, conventions and rules for models of
that type.  See: \std{4.2.5, 5.5 and 5.6}.

Repeat the next section \eric{commented out} for each model kind listed here the viewpoint
being specified.

}
%%%%%%%%%%
%\subsection{\Fillin{Model Kind Name}}\label{vp:mk}
%%%%%%%%%%
%\todo[inline, size=\footnotesize]{

%\must{Identify the model kind.}

%}
%%%%%%%%%%
%\subsubsection{\Fillin{Model Kind Name} conventions} 
%%%%%%%%%%
%\todo[inline, size=\footnotesize]{

%\must{Describe the conventions for models of this kind.}

%Conventions include languages, notations, modeling techniques,
%analytical methods and other operations. These are key modeling
%resources that the model kind makes available to architects and
%determine the vocabularies for constructing models of the kind and
%therefore, how those models are interpreted and used.

%It can be useful to separate these conventions into a \emph{language
%  part}: in terms of a metamodel or specification of notation to be
%used and a \emph{process part}: to describe modeling techniques used
%to create the models and methods which can be used on the models that
%result.  These include operations on models of the model kind.

%The remainder of this section focuses on the language part. The next
%section focuses on the process part.

%The Standard does not prescribe \emph{how} modeling conventions are to
%be documented.  The conventions could be defined:
%\begin{description}

%\textbf{I)} by reference to an existing notation or language (such as
%  SADT, UML or an architecture description language such as ArchiMate
%  or SysML) or to an existing technique (such as $M/M/4$ queues);
  
%\textbf{II)} by presenting a metamodel defining its core constructs;

%\textbf{III)} via a template for users to fill in;

%\textbf{IV)} by some combination of these methods or in some other
 % manner.
%\end{description}

%Further guidance on methods I) through III) is provided below.
 
%Sometimes conventions are applicable across more than one model kind
%-- it is not necessary to provide a separate set of conventions, a
%metamodel, notations, or operations for each, when a single
%specification is adequate.
%}


%%%%%%%%%%
%\subsubsection*{I) Model kind languages or notations \Optional}
%%%%%%%%%%
%\todo[inline, size=\footnotesize]{

%Identify or define the notation used in models of the kind.

%Identify an existing notation or model language or define one that can
%be used for models of this model kind. Describe its syntax, semantics,
%tool support, as needed.
%}
%\eric{seems like we will have to be a bit soft on how to do this. One could use goal models to define win-win situations. One could use process models to describe development cycles. What are good models to describe component lifecycles? Probably a template would be good at this point of time.}

Since the we are proposing a new viewpoint for Ecosystem and Transperency concerns in this paper, there is no established way of modelling it so far. 
Models would need to provide the following information:
\begin{description}
\item[Analysis of Value-Chain:] In order to document the different value-chains, on could either create a map of a (part of an) ecosystem as proposed by Janssen et al. \cite{Jansen2012b} or single out a specific value chain as we for example did in our previous work on the AUTOSAR ecosystem \cite{Soltani2015a} based on Boucharas et al.'s notation \cite{BJB2009}.
\item[Analysis of Actors and Goals:] In addition to the critical value chains, one should maintain information about the actors involved in those chains and their goals.
Yu and Franch have proposed to use goal models for this purpose, which is a promising approach to align goals of different actors based on architectural decisions and to achieve win-win situations. These are important when targeting strong collaborations and partnerships \cite{FSY2015}\eric{might need to find a better reference}.
\item[Reasoning about anonymous actors:] To some extend, the concrete partners and suppliers are not known when creating the electrical architecture. 
In such situations, we propose a persona based approach, i.e. defining archetypical actors with important characteristics as well as their expectation towards the ecosystem \cite{Knauss2014c,Hammouda2015}.
\item[Defining Transparency Requirements:] Hussaini et al. have proposed a framework for defining transparency requirements which could be a good starting point for these aspects \cite{HSP+2016}\eric{I believe they have a full paper at ASE or so, but cannot google}. 
For practical use in this viewpoint, this framework would need to be extended with dynamic aspects, e.g. for withdrawing information when a partnership ends. 
Obviously, this includes also ownership and digital rights management issues, for which we hope to borrow concepts from other works \cite{Muller,Averbakh2014}.
\end{description}
\eric{obviously need to add some details and references here...}

%%%%%%%%%%
%\subsubsection*{II) Model kind metamodel \Optional} 
%%%%%%%%%%
%\todo[inline, size=\footnotesize]{

%A metamodel presents the AD elements that constitute the
%vocabulary of a model kind, and their rules of combination. There are
%different ways of representing metamodels (such as UML class diagrams, OWL,
%eCore). The metamodel should present:
%\begin{description}
%\item[entities] What are the major sorts of conceptual elements that  are present in models of this kind?
% \item[attributes] What properties do entities possess in models of this kind?
% \item[relationships] What relations are defined among entities in models of this kind?
% \item[constraints] What constraints are there on entities, attributes  and/or relationships and their combinations in models of this kind?
%\end{description}

%\textbf{entities} What are the major sorts of conceptual elements that
%  are present in models of this kind?
  
%\textbf{attributes} What properties do entities possess in models of
%  this kind?
  
%\textbf{relationships} What relations are defined among entities in
%  models of this kind?
  
%\textbf{constraints} What constraints are there on entities, attributes
%  and/or relationships and their combinations in models of this kind?


%\note{Metamodel constraints should not be confused with architecture
%  constraints that apply to the subject being modeled, not the
 % notations used.}

%In the terms of the Standard, entities, attributes, relationships are
%\textit{AD elements} per \std{3.4, 4.2.5 and 5.7}.

%In the \textit{Views-and-Beyond} approach~\cite{DSA:2010}, each
%viewtype (which is similar to a viewpoint) is specified by a set of
%elements, properties, and relations (which correspond to entities,
%attributes and relationships here, respectively).

%When a viewpoint specifies multiple model kinds it can be useful to
%specify a single viewpoint metamodel unifying the definition of the
%model kinds and the expression of correspondence rules.  When defining
%an architecture framework, it may be helpful to use a single metamodel
%to express multiple, related viewpoints and model kinds.
%}
% \tbd{EXAMPLE -- In \cite{Hilliard:1999} and earlier work, we said that
%   all views are built from primitives called components, connections
%   and constraints which basically gives views a graph structure with
%   components as nodes and two types of edges (connections and
%   constraints). There are two issues with this: (\textit{1})
%   components and \textit{connectors} have taken on a specialized
%   meaning from the work by CMU and others \cite{Shaw-Garlan:1996};
%   (\textit{2}) this ur-ontology may be over-commiting for some views.}

%\eric{seems like we can anticipate a few entities and their relationshipts that will be important in this viewpoint, but much is left for future work.}

%%%%%%%%%%
%\subsubsection*{III) Model kind templates \Optional}
%%%%%%%%%%
%\todo[inline, size=\footnotesize]{

%Provide a template or form specifying the format and/or content of
%models of this model kind.
%}
%% \tbd{EXAMPLE} 


%%%%%%%%%%
%\subsubsection{\Fillin{Model Kind Name} operations \Optional} 
%%%%%%%%%%
%\todo[inline, size=\footnotesize]{

%Specify operations defined on models of this kind.

%See~\S\ref{Opns} for further guidance.

%}

%
%%%%%%%%%%
%\subsubsection{\Fillin{Model Kind Name} correspondence rules}
%%%%%%%%%%
%\todo[inline, size=\footnotesize]{

%\must{Document any correspondence rules associated with the model
%  kind.}

%See~\S\ref{CRs} for further guidance.
%}

%\eric{no clue what to write here.}

%%%%%%%%%%
\subsection{Operations on views}\label{Opns}
%%%%%%%%%%
\eric{Cannot see how we could write something completely different from what is in Section 8.5. Made a suggestion for addition there. Perhaps copy that text here and reference it in Section 7.5 and 8.5?}


\eric{Suggestions: ``identify win-win situations'', identify ``development lifecycle compatibility'', Define transparency needs and collaboration patterns}

\todo[inline, size=\footnotesize]{

Operations define the methods to be applied to views and their models.
Types of operations include:

%\begin{description}

%\item[construction methods] are the means by which views are  constructed under this viewpoint. 
% These operations could be in the  form of process guidance (how to start, what to do next); or work product guidance (templates for views of this type). 
% Construction techniques may also be heuristic: identifying styles, patterns, or other idioms to apply in the synthesis of the view.

% \item[interpretation methods] which guide readers to understanding and interpreting architecture views and their models.

% \item[analysis methods] are used to check, reason about, transform,  predict, and evaluate architectural results from this view, including operations which refer to model correspondence rules.

% \item[implementation methods] are the means by which to design and build systems using this view.

% \end{description}


\textbf{construction methods} are the means by which views are  constructed under this viewpoint. 
 These operations could be in the  form of process guidance (how to start, what to do next); or work product guidance (templates for views of this type). 
 Construction techniques may also be heuristic: identifying styles, patterns, or other idioms to apply in the synthesis of the view.

\textbf{interpretation methods} which guide readers to understanding and interpreting architecture views and their models.

\textbf{analysis methods} are used to check, reason about, transform,  predict, and evaluate architectural results from this view, including operations which refer to model correspondence rules.

\textbf{implementation methods} are the means by which to design and build systems using this view.


Another approach to categorizing operations is from Finkelstein et
al. \cite{Finkelstein+1992}. The \emph{work plan} for a viewpoint
defines 4 kinds of actions (on the view representations):
\textit{assembly actions} which contains the actions available to the
developer to build a specification; \textit{check actions} which
contains the actions available to the developer to check the
consistency of the specification; \textit{viewpoint actions} which
create new viewpoints as development proceeds; \textit{guide actions}
which provide the developer with guidance on what to do and when.
}


%%%%%%%%%%
\subsection{Correspondence rules}\label{CRs}
%%%%%%%%%%
\todo[inline, size=\footnotesize]{

\must{Document any correspondence rules defined by this viewpoint or
  its model kinds.}

Usually, these rules will be across models or across views since,
constraints within a model kind will have been specified as part of
the conventions of that model kind.

See: \std{4.2.6 and 5.7}
}
\eric{What ever could that be? Obviously, Continuous Integration and Deployment, since if we want to do that, the syncing of development cycles becomes important. Also Systems of Systems. But in what way? Perhaps that is rather something for discussion!}
%%\tbd{examples or specs}



%%%%%%%%%%
%\subsection{Examples \Optional} 
%%%%%%%%%%
%\todo[inline, size=\footnotesize]{

%Provide helpful examples of use of the viewpoint for the reader
%(architects and other stakeholders).
%}
%\eric{Future work}

%%%%%%%%%%
%\subsection{Notes \Optional} 
%%%%%%%%%%
%\todo[inline, size=\footnotesize]{

%Provide any additional information that users of the viewpoint may
%need or find helpful.

%}
%\eric{future work}

%%%%%%%%%%
%\subsection{Sources} 
%%%%%%%%%%
%\todo[inline, size=\footnotesize]{

%\must{Identify sources for this architecture viewpoint, if any,
%  including author, history, bibliographic references, prior art, per
%  \std{7e}.}
%}
%\eric{future work?!?}
