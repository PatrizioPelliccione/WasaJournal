% !TEX root = main.tex
%%% Version 2.1b %%%

\section{Continuous Integration and Deployment viewpoint}\label{sec:CID_VP}
%%%%%%%%%%
\renewcommand{\Fillin}[1]{{Continuous Integration and Deployment}}
\subsection{\Fillin{Viewpoint Name}}\label{vp:template}
%%%%%%%%%%

\del{\must{Provide the name for the viewpoint.}}

\del{If there are any synonyms or other common names by which this viewpoint is
known or used, record them here.}

The viewpoint ``Continuous Integration and Deployment'' adds a development perspective to the electrical architecture and aims to give an answer to the following questions:
\begin{enumerate}
\item How do continuous integration and deployment practices in automotive engineering impact electrical architecture?
\item How do architectural decisions in the electrical architecture impact continuous integration and deployment practices?
\end{enumerate}
\eric{I think it is crucial that we align on those questions. First draft above, please comment. I am not sure I hit a good level of ganularity.}

%%%%%%%%%%
\subsection{Overview} 
%%%%%%%%%%

Agile approaches and practices such as continuous integration and deployment promise to help reducing development time, to increase flexibility, and to generally shorten the feedback cycle time\ins{which in turn can lead to a better management of complex system knowledge}. 
However, the  \ugh{complex supplier network}\eric{should we mention ecosystem viewpoint here to reduce overlap?},
and typical setup with a large number of ECUs,
pose specific challenges to %\chg{agile development methods, and specifically 
%with respect 
%to continuous integration and deployment of software.}{
these practices. %}.

First,  dependencies between ECUs raise multiple concerns,
regarding organization, versioning and testing:
(i)  organization -
%the question is related to who should be 
identifying the recipient
of a given software change; (ii)
 versioning -
the question is related to the compatibility of the software version of specific ECUs; and
% software
%that are compatible.
(iii)  testing -  %This then carries over to the testing effort,
compatible combinations need to be validated. 
Second, support for continuous deployment has to face \del{with} safety concerns.
%relates to the connectivity of ECUs. 
%the extent of this, in turn, raises not least security concerns.
Should, for instance, the software \chg{of an}{delivered by an} ECU responsible\eric{does an ECU responsible actually deliver software? Seems weird, as this seems to imply responsibility for a specifc ECU, while the software should be hw independent. Should it be a ``Software Responsible'' instead?} for a safety critical function
be modifiable at runtime?

Dependencies between ECUs are a property of the architecture.
As mentioned, the emergent architecture may differ from the intended architecture,
and continuous integration and deployment of software may entail architectural changes.
This highlights both the need for collaboration % touches on the concern of the need for collaboration
between parts of the organization working on different architectural levels, and the need of a proper support
for agile and flexible architecting.
%Furthermore, the architectural framework should support agile architecting.

Addressing these concerns suggests \ugh{two architectural views and viewpoints}\eric{Should we then not split it into two viewpoints? Otherwise needs to be rephrased. I think we should have one viewpoint answering both questions.}:
(i) one covering architecture as an enabler
of continuous integration and deployment,
facilitating variant handling and coordination of updates, and
%
(ii) another considering continuous integration and deployment
on the architecture level,
facilitating reasoning about modifications to the architecture itself.\eric{In a way, continuous integration and deployment might also enable certain architecural decisions that otherwise might not be possible...}


%%%%%%%%%%
\subsection{Concerns and stakeholders} 
%%%%%%%%%%
\todo[inline, size=\footnotesize]{
Architects looking for an architecture viewpoint suitable for their
purposes often use the identified concerns and typical stakeholders to
guide them in their search.  Therefore it is important (and required
by the Standard) to document the concerns and stakeholders for which a
viewpoint is intended.
}
%%%%%%%%%%
\subsubsection{Concerns}\label{vp:concerns}
%%%%%%%%%%
In this section we focus on the concerns that are essential to enable continuous integration and deployment when engineering software for cars. 
We express the concerns in the form of questions as suggested by the ISO/IEC/IEEE 42010 standard.
\eric{Suggesting some edits and a change in order here to improve the flow. Please check if it is still correct!}
\begin{itemize}
\item \emph{How can we avoid building the wrong architecture?} \chg{It is no use in making a good architecture if it is for the wrong purpose.}{Even technically sound architecture is worthless if it does not fit the desired purpose. 
Naturally, an electrical architecture needs to exist long before the product under consideration is developed and can reach the market, since it is required to guide organization of development activities. 
Even though a lot of information is needed early in the product lifecycle, a crucial amount of information will only become available during its development and early market phase.}
\item \emph{How can we reduce the number of architectural assumptions?} The more of the architecture \chg{one make}{is defined} up-front,  the more \ins{it is based on }assumptions \chg{are needed to be added}{instead of facts}. \del{It is hard to guarantee that these assumptions are true, moreover, if there are too many assumption one might loose the overviews as well. }\ins{This can lead to late changes, duplicate work, or in the worst case to problems that only surface when the product is in the market.} 
\item \emph{How can \chg{we}{an electrical architecture} respond quicker to changes in the market?} \ins{Many of the assumptions will concern the situation at the market once the product is released. 
Markets are however prone to constant change and such change might invalidate architectural decisions during the development of software. }
\item \emph{How can we deal with changing interfaces?} \ins{Reacting to change in the market will lead to late changes of the architecture, which in turn will affect continuous integration and deployment: It is not possible to integrate a component into the system after the interface changes unless the platform and other dependent components have already been adjusted to the new interface. 
Deployment after an interface change might therefore involve a large number of updated components, which will not be possible in a continuous fashion. 
With other words, after an interface change it will be hard to guarantee that the main branch of all components is in a deployable state. }
\end{itemize}

%\must{Provide a listing of architecture-relevant concerns to be framed by
%this architecture viewpoint per \std{7a}.}

%Describe each concern.

%Concerns name ``areas of interest'' in a system.

%\note{Following ISO/IEC/IEEE 42010, \textbf{system} is a shorthand for
%  any number of things including man-made systems, software products
%  and services, and software-intensive systems such as ``individual
%  applications, systems in the traditional sense, subsystems, systems
%  of systems, product lines, product families, whole enterprises, and
%  other aggregations of interest''.}

%Concerns may be very general (e.g., \textit{Reliability}) or quite
%specific (\textit{e.g., How does the system handle network latency?}).
  
%Concerns identified in this section are critical information for an
%architect because they help her decide when this viewpoint will be
%useful.

%When used in an architecture description, the viewpoint becomes a
%``contract'' between the architect and stakeholders that these
%concerns will be addressed in the view resulting from this viewpoint.

%It can be helpful to express concerns \emph{in the form of questions}
%that views resulting from that viewpoint will be able to answer. E.g.,
%\begin{itemize}
%\item \textit{How does the system manage faults?}
%\item \textit{What services does the system provide?}
%\end{itemize}

%\note{``In the form of a question'' is inspired by the television quiz
%  show, \textit{Jeopardy!}}
 
%\std{5.3} contains a candidate list of concerns that must be considered
%when producing an architecture description. These can be considered
%here for their relevance to the viewpoint being specified:
%\begin{itemize}
%\item What are the purpose(s) of the system-of-interest?
%\item What is the suitability of the architecture for achieving the
%  system-of-interest's purpose(s)?
%\item How feasible is it to construct and deploy the
%  system-of-interest?
%\item What are the potential risks and impacts of the
%  system-of-interest to its stakeholders throughout its life cycle?
%\item How is the system-of-interest to be maintained and evolved?
%\end{itemize}

%See also: \std{4.2.3}.

%%%%%%%%%%
\subsubsection{Typical stakeholders} 
%%%%%%%%%%

For the continuous integration and deployment viewpoint, we need to consider all stakeholders described in Section~\ref{sec:VCGAF} and summarized in Table~\ref{tab:stakeholders}.
In addition, we need to consider suppliers, as continuous integration and deployment will depend on their deliveries (see also the ecosystem and transparency viewpoint in the next section).

\todo[inline, size=\footnotesize]{
\must{Provide a listing of the typical stakeholders of a system who
  are in the potential audience for views of this kind, per \std{7b}.}

Typical stakeholders would include those likely to read such views
and/or those who need to use the results of this view for another
task.

Stakeholders to consider include:
%\begin{itemize}

~~--~users of a system; 

~~--~operators of a system; 

~~--~acquirers of a system;

~~--~owners of a system; 

~~--~suppliers of a system; 

~~--~developers of a system; 

~~--~builders of a system; 

~~--~maintainers of a system.
%\end{itemize}
}

%%%%%%%%%%
\subsubsection{``Anti-concerns'' \Optional}
%%%%%%%%%%

\del{It may be helpful to architects and stakeholders to
document the kinds of issues for which this viewpoint is \emph{not
  appropriate or not particularly useful}.}

\del{Identifying the ``anti-concerns'' of a given notation or approach may
be a good antidote for certain overly used models and notations.}
\eric{I think we can remove this optional section here, no?}
% \tbd{Examples!}



%%%%%%%%%%
\subsection{Model kinds for Continuous Integration and Deployment Viewpoint}\label{mk:list}
%%%%%%%%%%
\todo[inline, size=\footnotesize]{
\must{Identify each model kind used in the viewpoint per \std{7c}.}

In the Standard, each architecture view consists of multiple
architecture models. Each model is governed by a \textit{model kind}
which establishes the notations, conventions and rules for models of
that type.  See: \std{4.2.5, 5.5 and 5.6}.

Repeat the next section for each model kind listed here the viewpoint
being specified.}

Various notations have been proposed to model continuous integration and deployment pipelines \ref{example1,example2,example3}\eric{Take something from the CID SLR}.
Among those, several could be applicable to this viewpoint, e.g. \ref{Stahl}.
In addition, one would need to relate those to architectural approaches as well as to the concerns defined above. 
We are currently working on a qualitative assessment method that allows to identify disruptors and bottlenecks for continuous integration and deployment\eric{I am thinking about the KACI model here (not published yet). What else?}.

\eric{Several subsections commented out in accordance to Section 8 (I believe).}
%%%%%%%%%%
% \subsection{\Fillin{Model Kind Name}}\label{vp:mk}
%%%%%%%%%%

%\must{Identify the model kind.}


%%%%%%%%%%
%\subsubsection{\Fillin{Model Kind Name} conventions} 
%%%%%%%%%%

%\must{Describe the conventions for models of this kind.}

%Conventions include languages, notations, modeling techniques,
%analytical methods and other operations. These are key modeling
%resources that the model kind makes available to architects and
%determine the vocabularies for constructing models of the kind and
%therefore, how those models are interpreted and used.

%It can be useful to separate these conventions into a \emph{language
%  part}: in terms of a metamodel or specification of notation to be
%used and a \emph{process part}: to describe modeling techniques used
%to create the models and methods which can be used on the models that
%result.  These include operations on models of the model kind.

%The remainder of this section focuses on the language part. The next
%section focuses on the process part.

%The Standard does not prescribe \emph{how} modeling conventions are to
%be documented.  The conventions could be defined:
%\begin{description}
%\item[I)] by reference to an existing notation or language (such as
%  SADT, UML or an architecture description language such as ArchiMate
%  or SysML) or to an existing technique (such as $M/M/4$ queues);
%\item[II)] by presenting a metamodel defining its core constructs;
%\item[III)] via a template for users to fill in;
%\item[IV)] by some combination of these methods or in some other
%  manner.
%\end{description}

%Further guidance on methods I) through III) is provided below.
 
%Sometimes conventions are applicable across more than one model kind
%-- it is not necessary to provide a separate set of conventions, a
%metamodel, notations, or operations for each, when a single
%specification is adequate.


%%%%%%%%%%
%\subsubsection*{I) Model kind languages or notations \Optional}
%%%%%%%%%%

%Identify or define the notation used in models of the kind.

%Identify an existing notation or model language or define one that can
%be used for models of this model kind. Describe its syntax, semantics,
%tool support, as needed.


%%%%%%%%%%
%\subsubsection*{II) Model kind metamodel \Optional} 
%%%%%%%%%%

%A metamodel presents the AD elements that constitute the
%vocabulary of a model kind, and their rules of combination. There are
%different ways of representing metamodels (such as UML class diagrams, OWL,
%eCore). The metamodel should present:
%\begin{description}
%\item[entities] What are the major sorts of conceptual elements that
 % are present in models of this kind?
%\item[attributes] What properties do entities possess in models of
%  this kind?
%\item[relationships] What relations are defined among entities in
%  models of this kind?
%\item[constraints] What constraints are there on entities, attributes
%  and/or relationships and their combinations in models of this kind?
%\end{description}

%\note{Metamodel constraints should not be confused with architecture
%  constraints that apply to the subject being modeled, not the
%  notations used.}

%In the terms of the Standard, entities, attributes, relationships are
%\textit{AD elements} per \std{3.4, 4.2.5 and 5.7}.

%In the \textit{Views-and-Beyond} approach~\cite{DSA:2010}, \eric{reference missing?} each
%viewtype (which is similar to a viewpoint) is specified by a set of
%elements, properties, and relations (which correspond to entities,
%attributes and relationships here, respectively).

%When a viewpoint specifies multiple model kinds it can be useful to
%specify a single viewpoint metamodel unifying the definition of the
%model kinds and the expression of correspondence rules.  When defining
%an architecture framework, it may be helpful to use a single metamodel
%to express multiple, related viewpoints and model kinds.

% \tbd{EXAMPLE -- In \cite{Hilliard:1999} and earlier work, we said that
%   all views are built from primitives called components, connections
%   and constraints which basically gives views a graph structure with
%   components as nodes and two types of edges (connections and
%   constraints). There are two issues with this: (\textit{1})
%   components and \textit{connectors} have taken on a specialized
%   meaning from the work by CMU and others \cite{Shaw-Garlan:1996};
%   (\textit{2}) this ur-ontology may be over-commiting for some views.}


%%%%%%%%%%
%\subsubsection*{III) Model kind templates \Optional}
%%%%%%%%%%

%Provide a template or form specifying the format and/or content of
%models of this model kind.

%% \tbd{EXAMPLE} 


%%%%%%%%%%
%\subsubsection{\Fillin{Model Kind Name} operations \Optional} 
%%%%%%%%%%

%Specify operations defined on models of this kind.

%See~\S\ref{Opns} for further guidance.


%%%%%%%%%%
%\subsubsection{\Fillin{Model Kind Name} correspondence rules}
%%%%%%%%%%

%\must{Document any correspondence rules associated with the model
%  kind.}

%See~\S\ref{CRs} for further guidance.


%%%%%%%%%%
\subsection{Operations on views}\label{Opns}
%%%%%%%%%%

\eric{Cannot see how we could write something completely different from what is in Section 8.5. Made a suggestion for addition there. Perhaps copy that text here and reference it in Section 7.5 and 8.5?}

\todo[inline, size=\footnotesize]{

Operations define the methods to be applied to views and their models.
Types of operations include:

%\begin{description}

%\item[construction methods] are the means by which views are  constructed under this viewpoint. 
% These operations could be in the  form of process guidance (how to start, what to do next); or work product guidance (templates for views of this type). 
% Construction techniques may also be heuristic: identifying styles, patterns, or other idioms to apply in the synthesis of the view.

% \item[interpretation methods] which guide readers to understanding and interpreting architecture views and their models.

% \item[analysis methods] are used to check, reason about, transform,  predict, and evaluate architectural results from this view, including operations which refer to model correspondence rules.

% \item[implementation methods] are the means by which to design and build systems using this view.

% \end{description}


\textbf{construction methods} are the means by which views are  constructed under this viewpoint. 
 These operations could be in the  form of process guidance (how to start, what to do next); or work product guidance (templates for views of this type). 
 Construction techniques may also be heuristic: identifying styles, patterns, or other idioms to apply in the synthesis of the view.

\textbf{interpretation methods} which guide readers to understanding and interpreting architecture views and their models.

\textbf{analysis methods} are used to check, reason about, transform,  predict, and evaluate architectural results from this view, including operations which refer to model correspondence rules.

\textbf{implementation methods} are the means by which to design and build systems using this view.


Another approach to categorizing operations is from Finkelstein et
al. \cite{Finkelstein+1992}. The \emph{work plan} for a viewpoint
defines 4 kinds of actions (on the view representations):
\textit{assembly actions} which contains the actions available to the
developer to build a specification; \textit{check actions} which
contains the actions available to the developer to check the
consistency of the specification; \textit{viewpoint actions} which
create new viewpoints as development proceeds; \textit{guide actions}
which provide the developer with guidance on what to do and when.
}

%%%%%%%%%%
%\subsection{Correspondence rules}\label{CRs}
%%%%%%%%%%

%\must{Document any correspondence rules defined by this viewpoint or
%  its model kinds.}

%Usually, these rules will be across models or across views since,
%constraints within a model kind will have been specified as part of
%the conventions of that model kind.

%See: \std{4.2.6 and 5.7}

%%\tbd{examples or specs}

%%%%%%%%%%
%\subsection{Examples \Optional} 
%%%%%%%%%%

%Provide helpful examples of use of the viewpoint for the reader
%(architects and other stakeholders).


%%%%%%%%%%
%\subsection{Notes \Optional} 
%%%%%%%%%%

%Provide any additional information that users of the viewpoint may
%need or find helpful.


%%%%%%%%%%
%\subsection{Sources} 
%%%%%%%%%%

%\must{Identify sources for this architecture viewpoint, if any,
%  including author, history, bibliographic references, prior art, per
%  \std{7e}.}

