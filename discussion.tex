% !TEX root = main.tex
\section{Discussion and Concluding remarks}\label{sec:conclusion}

In this paper we describe the current evaluation of Volvo Cars towards the definition of an architecture framework.
In this stage we are identifying potential viewpoints of the vehicles of the near future based on our shared experience in the automotive domain as well as a series of workshops for identification and validation for important concepts with a wide range of domain experts. 
Specifically, we present challenging scenarios about future demands related to architecture and proposed three new viewpoints (Continuous Integration and Deployment, Ecosystem and Transparency, System of System).
For each of these viewpoints, we started to collect actual needs and consumers of their respective views. 
Based on these, future work needs to identify the most suitable modeling languages to be used to describe the architecture. 
For these modeling languages, it is important to find the right tradeoff between a language's
(i) ability to support the level of formality and precision required by disciplined development processes, or
(ii) simplicity and it being intuitive enough to communicate the right message to stakeholders and to promote collaboration~\cite{whatindustrywants,IEEESoftwarePatrizio}.

For what concerns the realization of the architecture framework we will investigate MEGAF~\cite{MEGAF2010,MEGAF2012}, 
%
%MEGAF has been conceived with the aim of overcoming limitations of existing architecture frameworks. Most of the architecture framework in use are
%closed (i.e. they cannot be easily extended or adapted to new needs),
%%and the construction of existing ones requires a complete rework since reuse is not supported at all.
%and tool support for architecture frameworks exhibits this limitation
%as well: automated support, where it exists, follows the predefined
%viewpoints and models; support for developing extensions of
%architecture frameworks in terms of new viewpoints is non-existent.
%
%MEGAF is 
which is an infrastructure that enables software architects to realize their
own architecture frameworks specialized for a particular
application domain or community of stakeholders and compliant to the ISO/IEC/IEEE 42010 standard~\cite{42010}. 
%It offers
%an effective way of managing, storing, retrieving, and combining
%existing viewpoints, by properly selecting and reusing
%models previously defined and resident in MEGAF. 
%Architecture
%frameworks that can be realized by using MEGAF
%conform to ISO/IEC/IEEE 42010~\cite{42010}. 
%Once an architecture
%framework has been defined within MEGAF, it can be used
%to create architecture descriptions of systems-of-interest.
%%Architecture descriptions are easily made to adhere to the
%%architecture framework used to realize them. For instance,
%%the realized architecture models adhere to the conventions of
%%their model kinds as defined by the architecture framework
%%and its viewpoints.
%Moreover, the MEGAF infrastructure enables the architect
%to express and enforce relations both between various
%elements within an architecture description and across architecture
%descriptions. %Further details about MEGAF\footnote{\url{http://megaf.di.univaq.it/}} might be found at~\cite{MEGAF2010,MEGAF2012}. 

\section*{Acknowledgement}

We would like to thank Rich Hilliard for his precious comments and support. We would like to thanks also the consortium of the NGEA project for the valuable input that allowed the definition of the architecture framework.