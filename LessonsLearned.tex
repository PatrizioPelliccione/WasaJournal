% !TEX root = main.tex
\section{Software Development Challenges within Volvo Cars}\label{sec:lessonsLearned}

In a previous paper we studied within Volvo Cars,
from the architecture point of view,
the challenges that OEMs are facing in the last years~\cite{WICSA2015}. 
To better understand some of the
organizational issues with having different parts of the organization responsible for different parts or layers in the architecture, we decided to conduct nine in depth interviews with
focus on the roles of architects and how organizational factors affect them.
These interviews have been carried out at Volvo Cars  and Volvo Group Truck Technology
(VGTT), extending the knowledge about these specific companies, hence providing a detailed understanding of two independent but nevertheless similar automotive companies. %, information which is compared with the third, non-automotive, company. 
%The goal of this work was to build on previous results published in the
%literature {\bf Fix proper citations. [7], [6]}, to investigate the dual and
%complementary roles of software architects within the automotive domain.
In~\cite{WICSA2015} we 
%We 
followed a lightweight grounded theory-type approach since we started from a general research question and we involved few people in order to collect more information. Based on the analysis of the first data and on the first emerging ideas from the data, we then refined the research questions, carefully planned the study and accordingly selected the people to be interviewed. 
In addition to initial workshops in which we collected initial ideas, we used interviews as data generation methods. Specifically, we used semi-structured interviews: we defined a list of themes to be covered and questions we aimed at asking. However, in some interviews we changed the order of questions according to received answers and to the flow of the ``conversation". 
Each interview was around one hour long, we collected field notes and recorded audio. Each interview begins with introduction, clarification about the purpose of the study, asking permission to record and giving assurances of confidentiality of the information. \ins{More information about the research method might be found in~\cite{WICSA2015}}. 

%Moreover, in these years of strong collaboration with Volvo Cars and thanks also to the industrial co-authors (the last two authors of this paper) we had the possibility to further improve the knowledge in the domain and to collect a set of insights, as described in the following subsections.

%In order to mitigate the threat to validity in this study we followed the guidelines for conducting and reporting case study research in software engineering~\cite{Runeson2009,Wohlin2000}.
%
%For what concerns {\em construct validity}, we performed semi-structured interviews by following a questionnaire that has been defined by exploiting knowledge collected through two workshops, as described above. The workshops enabled also to identify and set a proper language to be used during the interviews.
%In addition to that, interviews have been performed by an industrial PhD working at VCG and then expert of the domain. This permits us to be confident that interviewed architects correctly interpreted the questions, and, in turn, we correctly interpreted their answers. 
%
%For what concerns {\em external validity}, having two authors from VCG could be seen as a threat to validity due to the
%risk of the study becoming heavily focused and influenced from the situation at
%one company. 
%%However, having two authors from academia and being aware of
%%the risk of such a bias, permitted to actively counter it. Moreover, the first author is hired by VCG, but, for the main purpose of making his PhD studies. This puts him in the position of being a perfect observer, without having strong influences and bias from the company.
%We performed the study in two different companies in order to reduce this bias. Having interviewees from different levels in the organization and verify their
%claims against each other also work to counter threats to validity. %It is also interesting to mention here that in our study we found confirmations of results of previous studies as discussed in Section~\ref{sec:discussion}. This gives us some flavour of generalizability of the findings of the study.
%%towards only one company
%%is also why a second company was brought into the study making it possible to
%%validate findings between the companies. 
%
%Finally, for what concerns {\em reliability}, as mentioned before we performed the interviews by following a well defined questionnaire. This allowed us to collect comparable answers through the different interviews. However, this enables also another researcher to conduct the same study. Moreover, the finding we obtained clearly emerged from the performed interviews: in other words there was no room for interpretation, finding we discussed in the paper are evident in the collected data.

\subsection{Gap between Prescriptive and Descriptive Architecture}\label{sec:gap}

We identified that there is not always an obvious connection between  
architecture (or top-level design) and  design; it seems also that this connection vanishes over time, once development requires changes on the system design. % and once time passes and , especially during later development. 
The architecture is communicated as large documents,
%, or models, 
which are supposed to be read by stakeholders. However, this not always corresponds to the reality. Volvo Cars works also in cross functional teams and other type of communications are also used to communicate the architecture. More analysis is needed to understand the root of the problem. 
Moreover, maintenance of the
architecture, while the design evolves, is demanding and not
always performed in all parts. 
This shows a discrepancy between the planned architecture defined according to a V-Model process, and the architecture that is actually emerging from the system development. 

We identified that the architecture has a temporal aspect, that means: at any given point in time the system
has only one architecture, however, the architecture will change over time.
We call as {\em prescriptive architecture} the architecture that captures
the design decisions made prior to the system's construction\footnote{A discussion in depth about prescriptive and descriptive models might be found in~\cite{Models2016}.}. This architecture is the as-conceived or as-intended architecture.
We can call as {\em descriptive architecture} the architecture that describes how the system has been built. This architecture describes the as-implemented or as-realized architecture. We observed that there is often a discrepancy between the as-intended and the as-implemented architecture. This causes what is called {\em architecture degradation}. The degradation might show up in two different ways: (i) {\em architectural drift} when the descriptive architecture includes changes that are not included in, encompassed by, or implied by the
prescriptive architecture, but which do not violate any of the prescriptive architecture's
design decisions; (ii) {\em architectural erosion} is the introduction of
architectural design decisions into a system's descriptive architecture that violate its prescriptive
architecture.
 
When looking at the causes of architecture degradation, we can summarize the various reasons in the following three items: 

\begin{enumerate} 
\item During the development some important directives of the as-intended architecture have been violated due to time constraints, mistakes, misunderstanding, etc. 
\item Some of the architectural choices of the as-intended architecture were based on assumptions (might be implicit) that then were identified as imprecise or wrong, 
\item Some of the architectural choices of the as-intended architecture were made under uncertainty and during development these choices were judged as suboptimal.
\end{enumerate}


\subsection{Organization and Process Challenges}\label{sec:organizationAndProcessChallenges}
The first item of the causes of architecture degradation discussed in the section above (Section~\ref{sec:gap}) might be solved by improving communication as mentioned before or by finding ways to guarantee the preservation of the  important directives of the as-intended architecture that should not be violated for any reason.

On the organizational side, we found the need of improving the communication between
different groups, for instance by making teams more cross-functional. Today
there are several levels between architects and designers/developers and at some of
these levels the connection is not very tight. 

On one side this seems to be unavoidable since the company is 
big and there is the need of structuring the organization in sub-organizations (departments, groups, or teams). 
However, there is the risk that sub-organizations will grow up independently and decoupled from each other; this might 
cause silos thinking.  
For instance, this might lead designers/developers to think
that the architects are sitting in their cloud above, without having
any connection to the reality. Architects might also feel
a frustration because they are not aware of everything that is happening; as a
consequence, a big part of their work is to just keep up with what is happening
in the construction groups. Espousing the terminology
in~\cite{whatindustrywants,IEEESoftwarePatrizio}, architects should be both ``Introvert" architects (conceptually related to the internal focus
of~\cite{Kruchten2008}), i.e., focused on more constructive work and definition of the architecture,
and 
``Extrovert'' architects (conceptually related to the external focus
of~\cite{Kruchten2008}), i.e. devoted to communicating the architectural
decisions and knowledge to the other stakeholders. 

The different organizations have different competencies, attitude, and
characteristics. However, new problems emerge. In fact, we can confirm that we
found many architecture antipatterns. We found the {\em Goldplating}
antipattern~\cite{Kruchten2008} since architects seem to be not
really engaged with developers. They are doing a good technical job, however,
their output is not really aligned with the needs of the developers and in the
end they are often ignored. Another antipattern that we found is the {\em Ivory
tower}~\cite{Kruchten2008}: the architecture team looks isolated
sitting on a separate floor from the development groups and do not engage with the
developers and the other stakeholders on a daily basis. This creates tensions in
the organization. 

It emerges the need to explore both organizational and technical
possibilities for tighter cooperation between architecture levels, and to
measure effects such improvements would have. On the technical side, one partial solution might be % idea is
to define specific viewpoints and to automatically generate corresponding architecture views from the design. 
%a framework able to automatically generate high-level views from the
%design. 
%The challenge here is to support multiple views, e
Each of the  views should focus on 
%devoted to 
showing only what is relevant for respective stakeholders. Moreover,
both architects and designers/developers need ways to perform early validation
of their solution and to sketch and try different visions of how the future
systems should look like; this will permit to understand the effect of design decisions on
the architecture. As mentioned before this can be only a partial solution since it cannot 
%However, this solution cannot 
solve the architecting problem since this solution only focus to visualize through specific viewpoints and views something that already exist. Other solutions are needed to solve the support just-in-time %agile 
architecting as mentioned above, as well as to enable stakeholders different from the architects, such as developers, to improve the architecture when actually needed.


%
%
%On the organizational side, we found a need to improve the communication between
%different groups, by finding a good balance between verbal and document based communication. 
%%.%for instance by making teams more cross-functional. 
%Espousing the terminology
%in~\cite{IEEESoftwarePatrizio}, system architects should be also
%``Extrovert'' architects (conceptually related to the external focus
%of~\cite{Kruchten2008}), i.e. devoted to communicating the architectural
%decisions and knowledge to the other stakeholders. 
%
%Often, the %se working architectures are developed in parallel by 
%different groups 
%%within the organizations, often having 
%have very different views on the meaning
%of architecture and how it shall be done. One must also note that, in general these groups are formed by domain experts, responsible for developing the mechatronic functions, not the optimized communication matrix or system architecture. In general we can identify the following types of conflict:
%(i) %what is 
%selecting the optimal solution for the %system (or function) 
%architecture; %?
%(ii) %which 
%selecting the platform %should 
%to be defined to facilitate development in the long run; %?
%(iii) %what should be 
%identifying the priority of the construction groups in order to deliver in time. %?
%%\jonn{The last one is unclear} \patrizio{To reduce the lenght of the introduction this part from ``One must also note that..." might be moved to another section.}
%
%%There is also not always an obvious connection between the high-level
%%architecture and the working architectures, especially during later development. 
%%The working architectures are
%%directly used by the construction groups. Since these architectures are forced to be consistent with the implemented system, they will inevitably diverge from the original design when the project evolves.
%%%They will inevitably diverge from the original design when the project evolves, as it is forced to be consistent with the implemented system. 
%%Contrariwise, the high-level architecture 
%%%only exists as 
%%is only communicated as large documents
%%%, or models, 
%%that are supposed to be read by stakeholders. Maintenance of the high-level
%%architecture, while the working architectures evolve, is demanding and not
%%always performed in all parts; consequently, it is difficult to determine the
%%quality of the high-level architecture description at any time.
%
%%In this sense we can say that the high-level
%%architecture has a descriptive nature, i.e., it aims at describing network design decisions and the main structures in the system architecture. The working architecture, on the other hand, is prescriptive, i.e., it defines interfaces, sets of rules, and recommendations, to follow during the implementation of the specific functionalities. 
%
%It emerges the need to explore both organizational and technical
%possibilities for tighter cooperation between architecture and design, and to
%measure effects such improvements would have. 
%%On the technical side, one partial solution might be % idea is
%%to define a framework able to automatically generate high-level views from the
%%low-level architecture. The challenge here is to support multiple views, each
%%devoted to showing only what is relevant for respective stakeholders. Moreover,
%%both high-level and low-level architects need ways to perform early validation
%%of their solution and to sketch and try different visions of how the future
%%systems should look like, to understand the effect of design decisions affect
%%the architecture. 
%%However, this solution cannot solve the architecting problem since this solution only focus to create a ``different view" for something that already exist. Other s





%Our findings are that 
%%for large and complex systems two different types of architectures, with
%%different abstraction levels, are used. A high-level architecture guiding and
%%diving the work between the construction or development groups is designed. Each of these groups creates 
%%a detailed communication matrix or working architecture\footnote{In the remaining of the paper we simply refer to this type of architecture as working architecture.} 
%%%Moreover, a detailed
%%%architecture is required 
%%to define strict interfaces and how the system implementation should be exactly realized. 
%%%The working architecture is monolithic since it is ``blamed" on the bandwidth, i.e. optimization of
%%%the communication.
%%%\jonn{it is quite central that this working arch is monolithic. Imagine a more
%%%internet-like architecture, it would have cased very different challenges. The
%%%monolithic architecture today is "blamed" on the bandwidth, i.e. optimization of
%%%the communication. Worth mentioning here??}
%often, %the %se working architectures are developed in parallel by 
%various groups within the same company 
%%within the organizations, often having 
%have different opinions on the meaning
%of architecture and how it shall be done. Example of topics in which we identified discrepancies are: 
%%. One must also note that, in general these groups are formed by domain experts, responsible for developing the mechatronic functions, not the optimized communication matrix or system architecture. In general we can identify the following topics of conflict:
%(i) %what is 
%selecting the optimal solution for the %system (or function) 
%architecture; %?
%(ii) %which 
%selecting the platform %should 
%to be defined to facilitate development in the long run; %?
%(iii) %what should be 
%identifying the priority of the construction groups in order to deliver in time. %?
%%\jonn{The last one is unclear} \patrizio{To reduce the lenght of the introduction this part from ``One must also note that..." might be moved to another section.}

%We identified that there is not always an obvious connection between  
%architecture (or top-level design) and  design, especially during later development. 
%The architecture is only communicated as large documents
%%, or models, 
%that are supposed to be read by stakeholders. However, this not always corresponds to the reality. Maintenance of the
%architecture, while the design evolves, is demanding and not
%always performed in all parts. 
%This shows a discrepancy between the planned architecture defined according to a V-Model process, and the architecture that is actually emerging from the system development. 

%\subsubsection{Towards agile architecting}\label{sec:agileArch}
%
%The role of the high-level architecture, if asking the architects themselves, is
%to serve as a set of guidelines and to identify the boundaries for the detailed
%design. In reality, this architecture is often, if not ignored, at least not in
%the mind of the engineers doing the low-level architecture on a daily basis. It
%emerged that low-level architects seldom, if ever, read the documentation
%produced by the high-level architects. Now, this does not mean that the work of
%the high-level architects is without value. However, having the high-level
%architects in a group of their own, separated from the other groups creates
%tensions between the different groups and a power struggle. 
%
%This shows a discrepancy between the planned architecture defined according to a V-Model process, and the architecture that is actually emerging from the system development. From the study it stems out in fact that the team responsible for the architecture, called high-level architecture above, tends to get isolated from the rest of the development organization, with few communications. This creates tensions within the organizations, as well as suboptimal design of the communication matrix and limited usage of the high-level architecture in the development teams. This clearly shows that in order to adapt to the current pace of software development and rapidly growing software systems new ways of working are required, both on technical and on an organizational level.
%
%What emerges here has been observed also in other domains. Specifically, this recalls the tension between waterfall and agile approaches. On one side waterfall approaches consider architecting as a phase of development that is somehow instructing the other phases. On the other side, agile development processes consider that ``The best architectures, requirements, and designs emerge from self-organizing teams.", as stated in the \#11 in the agile manifesto\footnote{Agile Alliance, Manifesto for Agile Software Development, June 2001 \url{http://agilemanifesto.org/}}.
%
%Building on that, Philippe Kruchten discusses the concept of ``agile architecture" that evokes two different interpretations (by quoting the text from the blog)\footnote{\url{http://philippe.kruchten.com/2013/12/11/agile-architecture/}}:
%
%\begin{itemize}
%\item a system or software architecture that is versatile, easy to evolve, to modify, flexible in a way, while still resilient to changes
%\item an agile way to define an architecture, using an iterative lifecycle, allowing the architectural design to tactically evolve gradually, as the problem and the constraints are better understood
%\end{itemize}
%
%These two interpretations are related but different; in fact as discussed by Kruchten, we may have a non-agile development process leading to a flexible and adaptable architecture, and on the other side, an agile process may lead to a rather rigid and inflexible architecture. The best would be an agile process, leading to a flexible architecture. 

%\subsubsection{Identifying the architects and their role}\label{sec:architects}
%
%In the same blog\footnote{\url{http://philippe.kruchten.com/2014/10/08/three-tures-architecture-infrastructure-and-team-structure/}} Kruchten points out also a conjecture that comes out at the XP 2014 workshop: architects typically work on three distinct but interdependent structures, which are:
%
%\begin{itemize}
%\item The architecture of the system under design, development, or refinement;
%\item The structure of the organization, including also partners, subcontractors, and others;
%\item The production infrastructure used to develop and deploy the system.
%\end{itemize}
%
%When these structures are not kept aligned over time, different kinds of ``debts" may show up: {\bf technical debt} when the architecture is lagging, {\bf social debt} when the structure of the organization is missing. 
%
%The alignment between the architecture and the production infrastructure is getting increasing interest with the concept of DevOps~\cite{Bass2015}, which puts the focus on combining the development organization with the operations organization, and on having the tools in place to ensure continuous delivery or deployment, even in the context of very large mission-critical systems, such as, Netflix, Amazon, and Facebook. Our paradigm is focusing on the evolution within the organization in contrast to the toolchain from the development organization to the customers, which is the main focus of DevOps.
%
%
%%\subsubsection{Organizational aspects}\label{sec:communication}
%
%On the organizational side, we found a need to improve the communication between
%different groups, for instance by making teams more cross-functional. 
%%Today
%%there are several levels between architects and implementers and at some of
%%these steps the connection is not very tight. This leads implementers to think
%%that the high-level architects are sitting in their cloud above, without having
%%any connection to the reality. On the other hand the high-level architects feel
%%a frustration because they are not aware of everything that is happening; as a
%%consequence, a big part of their work is to just keep up with what is happening
%%in the construction groups. 
%Espousing the terminology
%in~\cite{IEEESoftwarePatrizio}, system architects should be also
%``Extrovert'' architects (conceptually related to the external focus
%of~\cite{Kruchten2008}), i.e. devoted to communicating the architectural
%decisions and knowledge to the other stakeholders. 

%The different organizations have different competencies, attitude,
%characteristics. %However, new problems emerge. In fact, we can confirm that we
%%found many 
%We idneitifed some known architecture antipatterns, like %. We found 
%the {\em Goldplating}
%antipattern~\cite{Kruchten2008} since system architects seem to be not
%really engaged with developers. They are doing a good technical job, however,
%their output is not really aligned with the needs of the developers and in the
%end they are often ignored. Another antipattern that we found is the {\em Ivory
%tower}~\cite{Kruchten2008}: the %high-level 
%architecture team looks isolated
%sitting on a separate floor from the development groups and do not engage with the
%developers and the other stakeholders on a daily basis. This creates tensions in
%the organization. 
%
%It emerges the need to explore both organizational and technical
%possibilities for tighter cooperation between architecture levels, and to
%measure effects such improvements would have. 
%%On the technical side, one partial solution might be % idea is
%%to define a framework able to automatically generate high-level views from the
%%low-level architecture. The challenge here is to support multiple views, each
%%devoted to showing only what is relevant for respective stakeholders. Moreover,
%%both high-level and low-level architects need ways to perform early validation
%%of their solution and to sketch and try different visions of how the future
%%systems should look like, to understand the effect of design decisions affect
%%the architecture. 
%%However, this solution cannot solve the architecting problem since this solution only focus to create a ``different view" for something that already exist. Other s
%Solutions are needed to support agile architecting~\cite{shahrokni2016organic} %\footnote{\url{http://philippe.kruchten.com/2013/12/11/agile-architecture/}}  
%as well as to enable stakeholders different from the architects, such as developers, to improve the architecture, such as fixing wrong assumptions or making decision deliberately postponed.



\subsection{Towards ``Just-in-Time Architecture"}

The second and third items of the causes of architecture degradation discussed in Section~\ref{sec:gap} 
call for a ``{\em just-in-time architecture}" or agile architecting~\cite{shahrokni2016organic}  as well as to enable stakeholders different from the architects, such as developers, to improve the architecture, e.g. by fixing wrong assumptions or making decision deliberately postponed.

%
%Our analysis till now shows a discrepancy between the planned architecture defined according to a V-Model process,
%and the architecture that is actually emerging from the system development. According to what discussed at the
%above, this is a clear example of architecture degradation that can take the form of architecture
%erosion and/or architecture drift. From the study it stems out in fact that the team responsible for the architecture tends
%to get isolated from the rest of the development organization, with few communications. This creates tensions within
%the organizations, as well as suboptimal design of the communication matrix and limited usage of the high-level
%architecture in the development teams. This clearly shows that in order to adapt to the current pace of software
%development and rapidly growing software systems new ways of working are required, both on technical and on an
%organizational level.
%What emerges here has been observed also in other domains. Specifically, this recalls the tension between waterfall
%and agile approaches. On one side waterfall approaches consider architecting as a phase of development that is
%somehow instructing the other phases. On the other side, agile development processes consider that ``The best architectures, requirements, and designs emerge from self-organizing teams.'', as stated in the \#11 in the agile manifesto\footnote{Agile Alliance, Manifesto for Agile Software Development, June 2001 \url{http://agilemanifesto.org/}}.
One hypothesis made from some practitioners is that when developing large and complex systems ``a clear and well-defined architecture facilitates and enables
agility". This hypothesis implies that some upfront specification is needed when building complex products like cars.
However this hypothesis is not completely true when the product to be realized is not clearly defined and companies
want to go fast to the market (as for example observed by Waterman et al. \cite{WNA2015}. In these situations, modifiability, support for evolution, etc. are not really main aspects
to be considered. The hypothesis seems to be true when the product is well-defined. Another aspect to be considered
is that agile calls for refactoring, however refactoring often is not performed since the priority is given to what should
be realized.
%Further investigation is needed, however the conclusion we can draw at this point is that there is the need of a
%``just-in-time architecture" that enables even stakeholders that are different from the architects, such as developers, to
%improve the architecture, such as fixing wrong assumptions or making decisions deliberately postponed by the
%architects.




\subsection{Towards a Software Ecosystem Perspective}\label{sec:subcontractors}

Another interesting finding is that the architecture is not clearly considering the highly complex supplier-network that characterizes automotive engineering. 

%a high-level architect did not feel that
%there was a difference between using in-house developers and subcontractors.
%This might be due to the fact that the architects are more distant from the
%product. In the case of the people working with working architecture, this is
%not the case as we also found in our work~\cite{burden_comparing_2014}. They
%find it very frustrating to wait for part of the system to be integrated.  
%\patrizio{Eric, please add a description of the findings} Subcontractors and Architecture - Mozhan RE \cite{Soltani2015,Soltani2015a}
%	\begin{itemize}
%	\item How the architecture can better support the work with suppliers? \eric{Sure, but we had an RE Workshop paper, so it is clear that it will be from the requirements viewpoint}
%	\item How much information should be shared with suppliers (transparency)? \eric{I believe this is more like an outlook and needs to be researched in 2.4 and 2-2.4}
%	\end{itemize}
%
%\eric{Putting some content here, but of course it is not really about architecture.}
%Automotive engineering is characterised by a highly complex supplier-network.
%As a first step, we have investigated the impact of this complexity from the perspective of the requirements viewpoint based on a qualitative case study with an AUTOSAR Tier-2 supplier, a Tier-1 supplier and an OEM \cite{Soltani2015,Soltani2015a}\footnote{A video presentation of this work can be found at \href{https://oerich.wordpress.com/2015/08/14/how-does-the-autosar-ecosystem-impact-requirement-engineering/}{https://oerich.wordpress.com/2015/08/14/how-does-the-autosar-ecosystem-impact-requirement-engineering/}}.
%Through seven semi-structured interviews, we found that a clear, 
In a previous study~\cite{Soltani2015,Soltani2015a} 
we found that a holistic strategy for aligning work across the value-chain is currently missing. 
Specifically, mixing commodity and differentiating components lead to sub-optimal situations, resulting in duplicated work (an observation in line with \cite{OB2015}).  
We argue that automotive architecture needs to assume a holistic perspective with respect to the whole value-chain and optimize the architecture for facilitating beneficial subcontractor interaction. %\eric{perhaps we need to cite Helena and Jan here, no?}
\begin{itemize}
\item \textit{Commodity Components} require clearly defined technical and organizational interfaces. 
The goal is to develop them as efficiently as possible, thus reducing coordination overhead. 
Ideally, of-the-shelf commodity components can be integrated with minimal adjustment. 
\item \textit{Differentiating Components} should be developed as independent from the commodity components as possible, probably in-house. 
\item \textit{Innovative Components} naturally require coordination and iterative work between a number of partners. Proper communication means should be established in order to effectively stimulate and develop  innovative behaviours.  
%To effectively develop innovative behaviour, could communication channels need to be established. 
\end{itemize}

%This calls for a proper management of the automotive ecosystem%~\cite{knauss2014towards}. %
%, which is characterized by relying heavily on complex supplier networks, 
%and strong dependence on hardware and software development~\cite{knauss2014towards}.

In traditional software engineering, a software product is often the result of an effort of a single independent software vendor, investing into creating a monolithic product \cite{jansen2013defining, wnuk2014evaluating}.
Modern software engineering strongly relies on components and infrastructure from third-party vendors or open source suppliers \cite{jansen2013defining, wnuk2014evaluating}.
The emergence of Software Ecosystems (SECOs) is a recent development within the software industry \cite{hanssen2012longitudinal, wnuk2014evaluating}.
It implies a shift from closed-organizations to open structures where external actors become involved in the development to create competitive value \cite{jansen2013defining, hanssen2012longitudinal}.

Based on the ecosystem classification model \cite{jansen2013defining}, we  understand the automotive value chain as an ecosystem of cross-organizational collaborations among automotive suppliers~\cite{knauss2014towards}. 
In this ecosystem, the Original Equipment Manufacturer (OEM) is in the role of the ecosystem coordinator.
The ecosystem is privately owned, and participation to it is based on a list of screened actors. 

The automotive ecosystem is characterized by relying heavily on complex supplier networks, and strong dependence on hardware and software development \cite {knauss2014towards}.
Due to the increasing number of networked components, a level of complexity has been reached which is difficult to handle using traditional development processes \cite{fennel2006achievements}. 
The automotive industry addresses this problem through a paradigm shift from a hardware-, component-driven to a requirement- and function-driven development process, and a stringent standardization of infrastructure elements. %\eric{Patrizio, does this clash with your earlier reasoning? Please feel free to adjust.}
%One central standardization initiative is the AUTomotive Open System ARchitecture (AUTOSAR) \cite{fennel2006achievements}. 

The principal aim of the AUTomotive Open System ARchitecture (AUTOSAR) standard is to master the growing complexity of automotive electronic architectures \cite{furst2009autosar}.
We refer to the \emph{AUTOSAR ecosystem} as a subset of the \emph{automotive ecosystem}, where different actors participate in value creation (i.e. development of software components) by exchanging products and services based on a technical platform defined by the AUTOSAR standard. %\eric{perhaps too much focus on AUTOSAR? Then could delete starting from ``One central standardization initiative...''}

Several challenging areas, including requirements engineering, are reported in the automotive domain \cite{broy2006challenges}.
OEMs and suppliers need to communicate requirements based on the requirements documents, which are imprecise and incomplete nowadays~\cite{broy2006challenges}.
In this regard, the work in~\cite{fricker2010requirements} reports that the requirements value chain is little understood beyond software projects. 
It is unclear which requirements communication, collaboration, and decision-making principles lead to efficient, value-creating and sustainable alignment of interests between interdependent stakeholders across software projects and products~\cite{fricker2010requirements}. 
This is an important point since the way the interests and expectations of stakeholders of SECOs are communicated is critical for successfully influencing stakeholders in conceiving future solutions that meet their needs~\cite{fricker2009specification}. 
%however is important, because the way the interests and expectations of stakeholders of SECOs are communicated is critical for whether they are heard, hence whether the stakeholders are successful in influencing future solutions to meet their needs \cite{fricker2009specification}.
It is important to highlight that in the automotive domain, communication, collaboration, and decision-making are cross-organizational challenges related to the requirements viewpoint and requirements engineering.


%\begin{itemize}
%\item \textit{Commodity Components} require clearly defined technical and organizational interfaces. 
%The goal is to develop them as efficiently as possible, thus reducing coordination overhead. 
%Ideally, of-the-shelf commodity components can be integrated with minimal adjustment. 
%\item \textit{Differentiating Components} should be developed as independent from the commodity components as possible, probably in-house. 
%\item \textit{Innovative Components} naturally require coordination and iterative work between a number of partners. 
%To effectively develop innovative behaviour, could communication channels need to be established. 
%\end{itemize}
