%%
%% Copyright 2007, 2008, 2009 Elsevier Ltd
%%
%% This file is part of the 'Elsarticle Bundle'.
%% ---------------------------------------------
%%
%% It may be distributed under the conditions of the LaTeX Project Public
%% License, either version 1.2 of this license or (at your option) any
%% later version.  The latest version of this license is in
%%    http://www.latex-project.org/lppl.txt
%% and version 1.2 or later is part of all distributions of LaTeX
%% version 1999/12/01 or later.
%%
%% The list of all files belonging to the 'Elsarticle Bundle' is
%% given in the file `manifest.txt'.
%%

%% Template article for Elsevier's document class `elsarticle'
%% with numbered style bibliographic references
%% SP 2008/03/01
%%
%%
%%
%% $Id: elsarticle-template-num.tex 4 2009-10-24 08:22:58Z rishi $
%%
%%
\documentclass[preprint,12pt,3p]{elsarticle}

%% Use the option review to obtain double line spacing
%% \documentclass[preprint,review,12pt]{elsarticle}

%% Use the options 1p,twocolumn; 3p; 3p,twocolumn; 5p; or 5p,twocolumn
%% for a journal layout:
%% \documentclass[final,1p,times]{elsarticle}
%\documentclass[final,1p,times,twocolumn]{elsarticle}
%% \documentclass[final,3p,times]{elsarticle}
%\documentclass[final,3p,times,twocolumn]{elsarticle}
%% \documentclass[final,5p,times]{elsarticle}

%\documentclass[final,5p,times,twocolumn]{elsarticle}

%% if you use PostScript figures in your article
%% use the graphics package for simple commands
%% \usepackage{graphics}
%% or use the graphicx package for more complicated commands
%% \usepackage{graphicx}
%% or use the epsfig package if you prefer to use the old commands
%% \usepackage{epsfig}

%% The amssymb package provides various useful mathematical symbols
\usepackage{amssymb}

\usepackage{amsmath}%
\usepackage{amsfonts}%
\usepackage{amssymb}%
\usepackage{graphicx}
\usepackage{multirow}
\usepackage[disable]{todonotes}
\usepackage{url}
\usepackage{graphics}
\usepackage{balance}
\usepackage{hyperref}
\usepackage[utf8]{inputenc}

%% The amsthm package provides extended theorem environments
%% \usepackage{amsthm}

%% The lineno packages adds line numbers. Start line numbering with
%% \begin{linenumbers}, end it with \end{linenumbers}. Or switch it on
%% for the whole article with \linenumbers after \end{frontmatter}.
%% \usepackage{lineno}

%% natbib.sty is loaded by default. However, natbib options can be
%% provided with \biboptions{...} command. Following options are
%% valid:

%%   round  -  round parentheses are used (default)
%%   square -  square brackets are used   [option]
%%   curly  -  curly braces are used      {option}
%%   angle  -  angle brackets are used    <option>
%%   semicolon  -  multiple citations separated by semi-colon
%%   colon  - same as semicolon, an earlier confusion
%%   comma  -  separated by comma
%%   numbers-  selects numerical citations
%%   super  -  numerical citations as superscripts
%%   sort   -  sorts multiple citations according to order in ref. list
%%   sort&compress   -  like sort, but also compresses numerical citations
%%   compress - compresses without sorting
%%
%% \biboptions{comma,round}

% \biboptions{}

% Macros for proof-reading
\usepackage[normalem]{ulem} % for \sout
\usepackage{xcolor}
\newcommand{\ra}{$\rightarrow$}
\newcommand{\ugh}[1]{\textcolor{red}{\uwave{#1}}} % please rephrase
\newcommand{\ins}[1]{\textcolor{blue}{\uline{#1}}} % please insert
\newcommand{\del}[1]{\textcolor{red}{\sout{#1}}} % please delete
\newcommand{\chg}[2]{\textcolor{red}{\sout{#1}}{\ra}\textcolor{blue}{\uline{#2}}} % please change

% Put edit comments in a really ugly standout display
\usepackage{ifthen}
\usepackage{amssymb}
\newboolean{showcomments}
\setboolean{showcomments}{true} % toggle to show or hide comments
\ifthenelse{\boolean{showcomments}}
  {\newcommand{\nb}[2]{
    \fcolorbox{gray}{yellow}{\bfseries\sffamily\scriptsize#1}
    {\sf\small$\blacktriangleright$\textit{#2}$\blacktriangleleft$}
   }
   \newcommand{\version}{\emph{\scriptsize$-$working$-$}}
  }
  {\newcommand{\nb}[2]{}
   \newcommand{\version}{}
  }


\newcommand\patrizio[1]{\nb{Patrizio}{#1}}
\newcommand\eric[1]{\nb{Eric}{#1}}
\newcommand\rogardt[1]{\nb{Rogardt}{#1}}
\newcommand\magnus[1]{\nb{Magnus}{#1}}
\newcommand\pier[1]{\nb{Pier}{#1}}

% \newcommand\ulf[1]{}
% \newcommand\patrizio[1]{}
% \newcommand\rogardt[1]{}
% \newcommand\eric[1]{}



\newcommand\theme[1]{\emph{\textbf{#1}}}
\newcommand{\nextitem}{\par\hspace*{\labelsep}\textbullet\hspace*{\labelsep}}
% Nice tables
\usepackage{booktabs}
\usepackage{array}
\newcolumntype{v}[1]{>{\raggedright \hspace {0pt}}p{#1}}
\usepackage{colortbl}



\journal{Embedded Software Design (Journal of System Architecture)}


\newcommand{\Fillin}[1]{{$<$#1$>$}}
\newcommand{\HRule}{\rule{\linewidth}{0.5mm}}
\newcommand{\Optional}{{\textsf(optional)}}
\newcommand{\angles}[1]{$\langle$#1$\rangle$}
\newcommand{\must}[1]{{$\star$ #1}}
\newcommand{\note}[1]{\small\textsc{note: }\textit{#1}}
\newcommand{\should}[1]{{$\Box$ #1}}
\newcommand{\std}[1]{{ISO/IEC/IEEE~42010,~#1}}
\newcommand{\tbd}[1]{\noindent{\textbf{TBD: }{\textsf{#1}}}}
\newcommand{\working}[1]{\noindent{\textsf{#1}}}


\begin{document}

\begin{frontmatter}

\title{Automotive Architecture Framework: the experience of Volvo Cars\tnoteref{label0}}
\tnotetext[label0]{This work was partially supported by the NGEA and NGEA step 2 Vinnova projects led by Volvo Cars and by the Wallenberg Autonomous Systems Program (WASP)}




\author[label1]{Patrizio Pelliccione\corref{cor1}}
\ead{patrizio.pelliccione@gu.se}
\ead[url]{www.patriziopelliccione.com}
\author[label1]{Eric Knauss}
\ead{eric.knauss@gu.se}
\author[label1,label3]{Rogardt Heldal}
\ead{heldal@chalmers.se}
\author[label1]{Magnus \AA gren}
\ead{magnus.agren@chalmers.se}
\author[label1]{Piergiuseppe Mallozzi}
\ead{mallozzi@chalmers.se}
\author[label2]{Anders Alminger}
\ead{anders.alminger@volvocars.com}
\author[label2]{Daniel Borgentun}
\ead{daniel.borgentun@volvocars.com}

\address[label1]{Chalmers University of Technology $|$ University of Gothenburg, Department of Computer Science and Engineering, Sweden}
\address[label2]{Volvo Cars, Sweden}
\address[label3]{Bergen University College, Norway}

\cortext[cor1]{Patrizio is the corresponding author.}



\begin{abstract}
%\patrizio{taken from WASA paper. It needs to be actualized}
The automotive domain is living an extremely challenging historical moment	shocked by many emerging
business and technological needs. Electrification, autonomous driving, and connected cars are some
of the driving needs in this changing world. 
%During the past
%twenty years vehicles have become more and more robot like, interpreting and
%exploiting input from various sensors to make decisions and finally commit
%actions that were previously made by humans. 
Increasingly, vehicles are becoming software-intensive complex systems and most of the innovation within the automotive industry is based on electronics and software.
% Such features will require continuous evolution and updates to ensure safety, security, and suitability for  supporting drivers in an ever changing world.
Modern vehicles can have over 100 Electronic Control Units (ECUs), which are
small computers, together executing gigabytes of software. ECUs are
connected to each other through several networks within the car, and the car is increasingly connected with the outside world. 
%\chg{This evolution of the automotive industry %, illustrated by the exponential
%increase of embedded software, 
%creates new challenges regarding software
%architecture development and maintenance.}{
%This need for addressing ever increasing complexity as well as for offering flexibility, support of continuous evolution, and very late changes in user visible features introduces new challenges for developing and maintaining a suitable electronic architecture. %} 
These novelties ask for a change on how the software is engineered and produced and for a disruptive renovation of the electrical and software architecture of the car.   
In this paper we describe the current investigation of Volvo Cars to create an 
architecture framework able to cope with the complexity and needs of present and future vehicles.  
Specifically, we present scenarios that describe demands for the architectural framework and introduce three new viewpoints that need to be taken into account for future architectural decisions: Continuous Integration and Deployment, Ecosystem and Transparency, and car as a constituent of a System of Systems.
Our results are based on a series of focus groups with experts in automotive engineering and architecture from different companies and universities.
\end{abstract}

\begin{keyword}
%% keywords here, in the form: keyword \sep keyword
Architecture Framework \sep Software architecture \sep Automotive domain \sep Systems of Systems \sep Continuous Integration and Deployment \sep Automotive ecosystem
%% MSC codes here, in the form: \MSC code \sep code
%% or \MSC[2008] code \sep code (2000 is the default)
\end{keyword}

\end{frontmatter}

%%
%% Start line numbering here if you want
%%
% \linenumbers

%% main text
\section{Introduction}
\label{sec:intro}
%\eric{just checking if github/sharelatex work as expected before working on plane.}
%\patrizio{Taken from the WASA paper.}
Today's automotive industry is concerned with much more than the traditional view of assembled
mechanical parts controlled by a human for transportation. During the past
twenty years vehicles have become more and more robot like, interpreting and
exploiting input from various sensors to make (semi-)autonomous decisions and finally commit
actions that were previously made by humans.  
%Modern vehicles can have over 100 Electronic Control Units (ECUs), which are
%small computers, together executing gigabytes of software. ECUs are
%connected to each other through several networks within the car and, in some cases also to the outside world. 

Accordingly, the role of software is continuously changing. Initially, it was introduced in cars to optimize the
control of the engines. Since then, the growth of software within the car has
been exponential for each year and today not a single function is performed
without the involvement of software. 
80\% to 90\% of the innovation within the automotive industry is based on electronics~\cite{Peter2014}, 
%, 
as mentioned for instance by Peter van Staa - Vice-President Engineering of Robert Bosch GmbH at the European Technology Congress in June 2014~\cite{Peter2014}. 
A big part of electronics is software.

Considerable parts of the
software is safety-critical, with human life at stake if the system is not performing as
expected. Thus, the focus is gradually switching over from human control of the
mechanics to software and electronics supporting decision-making and even taking
over the control. 
The development is similar to the more popularized history of
air plane development and the invention of auto pilots and ``fly by wire", with
some striking differences. The higher complexity of the environment where the
vehicle moves has slowed down the development of automated vehicle control. The
most advanced cars have more or at least comparable amounts of software than fighter
airplanes\footnote{As said by Alfred Katzenbach, the director of information technology management at Daimler:\url{http://spectrum.ieee.org/transportation/systems/this-car-runs-on-code}}. 
Vehicles are also produced in higher volumes than airplanes, which puts harder requirements on the technology cost.

This evolution of the automotive industry %, illustrated by the exponential
%increase of embedded software, 
creates new challenges regarding the electrical and software
architecture development and maintenance. 
The architecture of a modern car has to cope with a large amount of concerns, including safety, security, variability management, networking, costs, weight, etc.
Also, the increasing amount of people
involved in the software development projects imposes additional challenges to the
architecture teams, as the development and design literally cannot be
controlled, or even understood, in detail by a single group any more. The
development is inevitably parallelized; % \eric{BE or AE?}; 
this obviously also holds for the large
amounts of externally developed software, which is integrated as black box
functionality. The important integration work is done in an iterative manner by
developers and test teams, focusing on vital systems first and then gradually
establishing various functionalities. Architects might be not involved in integration, 
however, the architecture for sure influences the integration.

In this paper we report a current initiative of the Volvo Cars to renovate the
electrical architecture. 
The work is part of a Vinnova FFI Swedish project, called Next Generation Electrical Architecture
 (NGEA)\footnote{http://www.vinnova.se/sv/Resultat/Projekt/Effekta/2009-02186/Next-generation-electrical-architecture},
which mainly focuses on the following areas: (i) continuous integration and deployment; (ii)
cars as constituents of a system of systems; (iii) reducing the number of ECUs through an architecture that allows the identification of key functions to be implemented in domain nodes; and (iv) strategies to improve the automotive ecosystem so to enable rapid communication with suppliers and flexible development. The reason for the NGEA project to choose these topics were that they are strategically important for automotive industry and it is important to find a way of handling them. In this paper we largely extend beyond a short paper~\cite{WasaPaper}: the paper in~\cite{WasaPaper} only sketches preliminary results.
 
%\begin{itemize}
%\item Continuous Integration and Deployment ;
%\item Cars as constituents of a system of systems;
%\item Reducing the number of ECUs through an architecture that allows the identification of key functions to be implemented in domain nodes; % so to make simpler more standardized and interchangeable other nodes; 
%\item Strategies to improve the automotive ecosystem so to enable rapid communication with suppliers and flexible development
%%for enabling the key functions developed locally and thus ensure rapid communication with the partners and flexible development 
%\end{itemize}

Within the project we are investigating the possibility to create a Volvo Cars Architecture Framework. %tailored to the needs of Volvo Cars. In this paper we refer to the conceptual foundations of
We believe that an architecture framework~\cite{42010}, together with its multiple viewpoints, is the instrument to manage the increasing complexity of modern vehicles.
It aims at ensuring that descriptions of vehicle architectures can be compared and related across different vehicle programs,
development units, and organizations, thus increasing flexibility and innovation, while reducing development time and risks. \ins{However, the definition and description of an architecture require a cost in terms of human and financial resources. Moreover, sometimes it is not evident that investing effort and money on properly defining and describing the architecture will result in saving money later in the development. For instance, as described later in the paper, we identified discrepancies between the as-intended and the as-implemented architecture. This motivates the shift towards ``just-in-time" architecting and on continuous integration and deployment.} 

We build on existing architecture frameworks in the automotive domain, i.e.,~\cite{Broy,Yania} and we base our work on the conceptual foundations provided by the ISO/IEC/IEEE~42010:2011 standard~\cite{42010}.   
%, \textit{Systems and software engineering ---
%  Architecture description}~\cite{42010} to investigate the essential
%elements of architecture description and support the architect's
%architecture description-related activities.
%ISO/IEC/IEEE~42010 is the joint ISO and IEEE revision of IEEE~Std
%1471, first published in 2000~\cite{1471}. 
%The standard is method-neutral: it
%is intended for use by architects employing various architecting methods. 
The standard 
 addresses architecture description, i.e. the practices of recording software, system
and enterprise architectures so that architectures can be understood,
documented, analysed, and realized. 
%Architecture descriptions are created by
%architects and used by architects and other stakeholders throughout
%all stages of a system's life cycle, from development through
%operation and maintenance. 
%In this paper we report about the identification of the main viewpoints we identified so far and in which we are focusing our effort.

The paper is organized as follows. Section~\ref{sec:AAF} introduces the concept of architecture framework and explain a template to document architecture frameworks. Section~\ref{sec:automotiveAF} gives an overview of 
 the state of the art on architecture framework in the automotive domain. Section~\ref{sec:lessonsLearned} analyses the state of practice in the context of Volvo Cars and reports some of the lessons learned. 
 Section~\ref{sec:VCGAF} introduces the architecture framework we are defining within Volvo Cars in the context of the NGEA project, including an overview of stakeholders as well as critical scenarios that define demands and expectations towards the architecture framework. %\eric{should we mention scenarios?}
 Sections~\ref{sec:CID_VP}, \ref{sec:ET_VP}, and~\ref{sec:SoSVP} detail three new and challenging viewpoints of the framework, namely the {\em Continuous Integration and Deployment} viewpoint (Section~\ref{sec:CID_VP}), the {\em Ecosystem and Transparency} viewpoint (Section~\ref{sec:ET_VP}), and the {\em System of Systems: vehicle point of view} viewpoint (Section~\ref{sec:SoSVP}).
 %
Finally the paper concludes in Section~\ref{sec:conclusion} with final remarks and a discussion about future work. 

\section{Architecture Framework}\label{sec:AAF}

%\subsection{Architecture Framework}\label{sec:archFram}

%\subsubsection{Overview}

An architecture framework is a
coordinated set of viewpoints, conventions, principles and practices
for architecture description within a specific domain of application
or community of stakeholders~\cite{42010}.  More specifically, it is determined by: (i) a set of architecture-related concerns,
(i) a set of stakeholders holding those concerns, (ii) a set of architecture viewpoints which frame (i.e.,
cover) those concerns, and (iv) a set of model correspondence rules to impose constraints between
types of models and then demonstrate that constraints
are satisfied by the architecture. 
Then, an architecture framework establishes a common practice for creating, interpreting, analyzing and using architecture descriptions within a particular domain of application or stakeholder community, developing
architecture modelling tools and architecting methods, and establishing processes to facilitate communication, commitments and interoperation across multiple projects and/or organizations~\cite{42010}.


Uses of architecture frameworks include, but are not limited to~\cite{42010}: 

\begin{itemize}
\item creating architecture descriptions; 
\item developing
architecture modelling tools and architecting methods; 
\item establishing processes to facilitate communication, commitments and interoperation across multiple projects and/or organizations.
\end{itemize}

An architecture framework is a prefabricated knowledge structure, identified by {\em architecture viewpoints}, that
architects use to organize an architecture description into {\em architecture views}.  The terms architecture view and architecture viewpoint are central to the standard~\cite{42010}:
``{\em A viewpoint is a way of looking at systems; a view is the result of applying a viewpoint to a particular
system-of-interest}". 
An {\em architecture viewpoint} encapsulates notations, conventions, methods and techniques to
be used according to specific 
model kinds framing particular
concerns and for a particular audience of system stakeholders. 
The
concerns determine what the model kinds must be able to express: e.g.,
functionality, security, reliability, cost, deployment, etc.  
A model
kind determines the notations, conventions, methods and techniques to
be used. 
Viewpoints, defining the contents of each architecture view,
are built up from one or more model kinds and correspondence rules
linking them together to maintain consistency.
Viewpoints, like patterns and styles, are a form of reusable
architectural knowledge for solving certain kinds of architecture
description problems derived from best practices.  
Viewpoints
originated in the 1970s (in Ross' Structured Analysis) and refined in~\cite{Finkelstein+92}. Architecting methods often define one or
more viewpoints, e.g.~\cite{4+1,RozWooBook,ClementsBachmannEtAl03, Eeles-Cripps:2010}.


Many existing practices express architectures through collections of
models, and models are further organized into cohesive groups, called {\em views}. A view can be defined as a ``{\em work product expressing the architecture of a system from the perspective of specific system concerns}"~\cite{42010}.
As noted in the standard, the cohesion of a
group of models is determined by specific concerns, which are addressed by that group of models. Viewpoints refer to the conventions
for expressing an architecture with respect to a set of concerns.

For further discussion of the
content model and architecture frameworks mechanism,
see~\cite{Emery-Hilliard:2009}. 
Recent
frameworks include the ISO Reference Model for Open Distributed
Processing, GERAM (Generalized Enterprise Reference Architecture
and Methodology)~\cite{ISO15704}, DODAF (US Department of Defense Architecture
Framework)~\cite{DODAF}, TOGAF~\cite{TOGAF}, and MODAF~\cite{MODAF}. 
For an extensive survey of
frameworks, see~\cite{AFS}. 
    


%\subsection{Automotive domain}\label{sec:automotiveAFs}



%Automotive embedded systems are typically categorized into different domains, such as vehicle-centric functional domains
%(including powertrain control, chassis control, and active/passive safety systems) and
%passenger-centric functional domains (covering multimedia/telematics, body/comfort, and
%human machine interface (HMI))~\cite{Navet2008}. 
%Each functional domain needs to consider different stakeholders and
%system concerns~\cite{Yania}. %As an example, we can say that body domain supports the functioning of the airbag, wiper, and lighting
%%and other functions for the vehicle users, while the powertrain control enables the longitudinal propulsion
%%of the vehicle~\cite{Yania}). 
%However, at noted in~\cite{Yania} all the integrated functionalities must
%not jeopardize the key vehicle requirements of safety and efficiency.
%
%These considerations motivate the need of considering different viewpoints and views from the perspective of specific system concerns, which are relevant to one or more stakeholders. At the same time it is of key importance to identify and devise suitable connections among the various views and viewpoints.
%In other words, these considerations testify the need of an automotive architecture framework.

In this paper we use the template called \textit{Architecture Viewpoint Template} for specifying architecture viewpoints in
  accordance with ISO/IEC/IEEE~42010:2011, \textit{Systems and
    software engineering---Architecture description}\footnote{The template is called \textit{Architecture Viewpoint Template}, which is released under copyright \copyright\
2012--2014 by \href{http://www.iso-architecture.org/42010/templates/}%
{Rich Hilliard}. 
The template is licensed under a
Creative Commons Attribution 3.0 Unported License. The terms of use
are here: 
\url{http://creativecommons.org/licenses/by/3.0/}.}   


%\input{42010-template-kit/viewpoint-template.tex}

%According to the description above and to the standard~\cite{42010}, in this paper we will describe architecture viewpoints according to the following template:
%
In the following we report the template sections and an excerpt of the guidelines; please refer to the links above for a complete description of the template: 

\begin{itemize}
\item {\em Viewpoint Name}: name of the viewpoint. If there are any synonyms or other common names by which this viewpoint is
known or used, record them here.
\item {\em Overview}: Provide an abstract or brief overview of the viewpoint. 
Describe the viewpoint's key features.
\item {\em Concerns and stakeholders}: Architects looking for an architecture viewpoint suitable for their
purposes often use the identified concerns and typical stakeholders to
guide them in their search.  Therefore it is important (and required
by the Standard) to document the concerns and stakeholders for which a
viewpoint is intended.
%a listing of the architecture-related concerns framed by this viewpoint. This is
%crucial information for the Architect, because it helps her decide whether this
%viewpoint will be useful to apply to a given system of interest, and to
%communicate with its stakeholders.
\begin{itemize}
\item {\bf Concerns}: Describe each concern.
Concerns name ``areas of interest'' in a system.
Concerns may be very general (e.g., \textit{Reliability}) or quite
specific (\textit{e.g., How does the system handle network latency?}).
Concerns identified in this section are critical information for an
architect because they help her decide when this viewpoint will be
useful.
When used in an architecture description, the viewpoint becomes a
``contract'' between the architect and stakeholders that these
concerns will be addressed in the view resulting from this viewpoint.
It can be helpful to express concerns \emph{in the form of questions}
that views resulting from that viewpoint will be able to answer. E.g., {\em How does the system manage faults?} or {\em What services does the system provide?}
\item {\bf Typical stakeholders}: Provide a listing of the typical stakeholders of a system who
  are in the potential audience for views of this kind.
Typical stakeholders would include those likely to read such views
and/or those who need to use the results of this view for another
task.
\item {\bf Anti-concerns [Optional]} It may be helpful to architects and stakeholders to
document the kinds of issues for which this viewpoint is \emph{not
  appropriate or not particularly useful}.
Identifying the ``anti-concerns'' of a given notation or approach may
be a good antidote for certain overly used models and notations.
\end{itemize}
%\item {\em Anti-concerns [Optional]}: It can be useful to document the kinds of issues a viewpoint is not
%appropriate for. Articulating anti?concerns may be a good antidote for certain
%overly used notations.
%\item {\em Typical stakeholders}: The typical audience for views prepared using this viewpoint. This item should answer the question: Who
%are the usual stakeholders for this kind of view?
%\item {\bf Model Kind+} 
\item {\em Model Kind name\footnote{In the Standard, each architecture view consists of multiple
architecture models. Each model is governed by a \textit{model kind}
which establishes the notations, conventions and rules for models of
that type.  
Repeat the next section for each model kind listed here the viewpoint
being specified.}}:  Identify the model kind
%For each type of model used, describe the language or modeling techniques to
%be used. Each model language is a key modeling resource that the viewpoint
%makes available. Model languages provide the vocabularies for constructing the
%view. ISO/IEC 42010 does not specify how a modeling language is documented.
%It could be by reference to an existing modeling language (e.g., EAST-ADL or UML)
%or technique (e.g., M/M/4 queues); by providing a metamodel for the language
%to define the language's core constructs; via a template that users fill in; or by
%some combination of these methods.
\begin{itemize}
\item {\em Conventions}: Describe the conventions for models of this kind. Conventions include languages, notations, modeling techniques,
analytical methods and other operations. These are key modeling
resources that the model kind makes available to architects and
determine the vocabularies for constructing models of the kind and
therefore, how those models are interpreted and used.
The Standard does not prescribe \emph{how} modeling conventions are to
be documented.  The conventions could be defined:
\begin{description}
\item[I)] by reference to an existing notation or language (such as
  SADT, UML or an architecture description language such as ArchiMate
  or SysML) or to an existing technique (such as $M/M/4$ queues);
\item[II)] by presenting a metamodel defining its core constructs;
\item[III)] via a template for users to fill in;
\item[IV)] by some combination of these methods or in some other
  manner.
\end{description}
\item {\em Operations}: Specify operations defined on models of this kind.
\item {\em Correspondence rules}: Document any correspondence rules associated with the model
  kind.
\end{itemize}
\item {\em Operations on views}: Operations define the methods to be applied to views and their models.
Types of operations include:
\begin{description}
\item[construction methods] are the means by which views are
  constructed under this viewpoint. These operations could be in the
  form of process guidance (how to start, what to do next); or work
  product guidance (templates for views of this type). Construction
  techniques may also be heuristic: identifying styles, patterns, or
  other idioms to apply in the synthesis of the view.
\item[interpretation methods] which guide readers to understanding
  and interpreting architecture views and their models.
\item[analysis methods] are used to check, reason about, transform,
  predict, and evaluate architectural results from this view,
  including operations which refer to model correspondence rules.
\item[implementation methods] are the means by which to design and
  build systems using this view.
\end{description}
\item {\em Correspondence rules}: 
Document any correspondence rules defined by this viewpoint or
  its model kinds.
Usually, these rules will be across models or across views since,
constraints within a model kind will have been specified as part of
the conventions of that model kind.
%The viewpoint may specify model correspondence rules. Each one may be
%documented here.
%\item {\em Operations on views}: Operations define the methods which may be applied to views and their
%models. Operations can be divided into categories: Creation methods are the
%means by which views are prepared using the viewpoint. These could be in the
%form of process guidance (how to start, what to do next); or work product
%guidance (templates for views of this type); heuristics, styles, patterns, or other
%idioms. Interpretive methods provide the means by which views are to be
%understood by readers and system stakeholders. Analysis methods are used to
%check, reason about, transform, predict, apply and evaluate architectural
%results from this view. Implementation methods capture how to realize or
%construct systems using information from this view.
\item {\em Examples [Optional]}: 
Provide helpful examples of use of the viewpoint for the reader
(architects and other stakeholders).
\item {\em Notes [Optional]} Provide any additional information that users of the viewpoint may
need or find helpful.
%Optional. This section provides examples for the reader.
\item {\em Sources}: Identify sources for this architecture viewpoint, if any,
  including author, history, bibliographic references, prior art, etc.
%What are the sources for this viewpoint, if any? This may include author,
%history, literature references, prior art, etc.
\end{itemize}


\section{State of the art in Automotive Architecture Frameworks}\label{sec:automotiveAF}
Architecture frameworks have not been standardized in the automotive domain and automotive industry. 
However, some attempts exist and different types of architecture viewpoints and views have been introduced recently as part
of automotive architecture frameworks. 
A first attempt towards a standardized architectural foundation and automotive-specific
architecture framework is the Automotive
Architecture Framework (AAF) proposed in~\cite{Broy}. 
The purpose of AAF is to describe the entire vehicle system
across all functional and engineering domains and drive the thought process within the
automotive industry.
The AAF conforms to the ISO 42010 international standard~\cite{42010}, and therefore it is defined in terms of a set of viewpoints and views. 
%
%Figure~\ref{fig:aaf} shown the levels of architectures and architecture frameworks, spanning from the Meta Architecture Framework, which is the most generic layer that is fully independent of any type of system to develop, to the Product Architecture, which defines the architecture for a concrete product. 
%
  %
%\begin{figure}
%\begin{center}
%%  \vspace{-.3cm}
%  \includegraphics[width=.7\columnwidth]{figures/AAF.pdf}  
%  \caption{Automotive Architecture Framework (AAF) - Figure taken from~\cite{Broy}}
%%  \vspace{-.5cm}
%  \label{fig:aaf}
%    \end{center}
%\end{figure}
%
%Referring to~\cite{TUM-I0915}:
%
%The AAF is defined according to several levels; more details might be found at~\cite{TUM-I0915}. 
% of details as described below~\cite{TUM-I0915}:
%
%\begin{itemize}
%\item {\em Meta Architecture Framework} introduces terms like component, interface, view, viewpoint and concern. 
%\item {\em Common Architecture Framework} defines concepts, which are necessary for any kind of system. 
%%It is partitioned into three main parts, which are representing the total system at different layers of abstraction,
%%namely the Functional Architecture (functionality that the system offers
%% to its outside world - black-box and hardware agnostic), the Logical Architecture (decomposition of a system into a number of Logical Components - white-box and hardware agnostic) and the Technical Architecture (lower level of abstraction - runtime model, allocation, and hardware topology). 
%% % - see Figure~\ref{fig:abstractionDetails}.
%%\begin{itemize}
%%\item The Functional Architecture consists of a function hierarchy that contains the description of the functionality that the system offers
%% to its outside world. The description is black-box and is hardware agnostic. 
%%\item The Logical (Component) Architecture defines the decomposition of a system into a number of Logical Components, which interact
%%and cooperate to offer the functionality that is described in the Functional Architecture.
%%The description is then white-box, it can describe the functionality of a subsystem, and is hardware agnostic.
%%\item The Technical Architecture describes how the system that is specified by means of Logical Components can be integrated
%%into a given hardware platform. The system is presented from the  realization perspective. The technical architecture consists of three parts: The Runtime Model, the Allocation, and the Hardware Topology.
%%\end{itemize}
%\item {\em Domain-specific Architecture Framework for the Automotive Domain} defines the foundational framework providing a
%common denominator in terms of terminology, structure, methods, architecture models, guidance and rules. 
%%, ``{\em for
%%developing, representing, understanding, and comparing domain-specific product architectures across a
%%(virtual) development organization (i.e. Automotive value net)}"~\cite{TUM-I0915}.
%\item {\em Company X Architecture Framework} customizes the domain-specific Architecture
%Framework for a concrete company like Volvo cars, BMW or FIAT. We elaborate on the VCG architecture framework in Section~\ref{sec:VCGAF}.
%\item {\em Vehicle line architectures} defines the architecture for a specific product line.
%\item {\em Vehicle architectures} define the architecture for a specific product.
%\end{itemize}

 
%\begin{figure}
% \begin{center}
%%  \vspace{-.3cm}
%\includegraphics[width=6cm,angle=270]{figures/abstractionDetails.pdf}
%  \caption{Level of Abstraction and Level of Detail - Figure taken from~\cite{TUM-I0915}}
%%  \vspace{-.5cm}
%  \label{fig:abstractionDetails}
%      \end{center}
%\end{figure}


%Focusing on the Domain-specific Architecture Framework for the Automotive Domain, t
The AAF distinguishes between two sets of architecture viewpoints and views: (i) mandatory and general viewpoints and (ii) optional viewpoints. The following viewpoints are presented according to the viewpoint catalog in~\cite{Rozanski2005}.
The mandatory viewpoints and their respective views include: (i)
%
%\begin{itemize} 
%\item 
Functional viewpoint - functional decomposition, functional
architecture;
%\item 
(ii) Logical viewpoint - logical decomposition, logical architecture; this viewpoint is only mentioned in~\cite{Broy} but not detailed in~\cite{TUM-I0915};
%\item 
(iii) Technical viewpoint - physical decomposition, technical architecture;
%\item 
(iv) Information viewpoint - perspective of information or data objects used to
define and manage a vehicle;
%\item 
(v) Driver/vehicle operations viewpoint - vehicle environment;
%\item 
(vi) Value net viewpoint - OEM stakeholders and those of its suppliers and engineering partners. 
%\end{itemize}
The optional viewpoints suggested by the AAF are:
(i) Safety, (ii) Security, (iii) Quality and RAS - Reliability, Availability, Serviceability, (iv) Energy, possibly including performance, (v) Cost, (vi) NVH - noise, vibration, harshness, and (vii) Weight.

\subsection{Architectural Design Framework}

The Architectural Design Framework (ADF)~\cite{AFRenault} is developed by Renault %to support
%the construction of an architecture framework for the automotive industry. The ADF
and 
includes operational, functional, constructional, and requirements viewpoints. Although
the AAF and ADF are related they have some differences. 

\begin{itemize}
\item {\em Operational Viewpoint} - this is the more abstract viewpoint. %The actors, the system scope, the system environment and
%high-level interactions have to be identified; t
The system is observed from a black box and user perspective~\cite{AFRenault}.
\item {\em Functional  Viewpoint} - system functions %are obtained from 
identified in the views associated to the operational viewpoint %and by properly grouping or refining Activities (actions) and allocating them on 
are allocated to SysML Blocks. 
\item {\em Constructional Viewpoint} - this viewpoint %is more detailed and 
describes a further allocation (or grouping) of system functions into physical components.
\item {\em Requirements Viewpoint} - This viewpoint is orthogonal to other viewpoints. Each requirement view has a relationship with views of
other viewpoints. 
\end{itemize}

\subsection{Architecture Framework For Automotive Systems}
The Architecture Framework for Automotive Systems (AFAS) is proposed in~\cite{Yania} through an analysis of AAF, ADF and of Architecture Description Languages~\cite{ADL_Neno00,whatindustrywants} tailored to automotive domain, like EAST-ADL~\cite{EAST-ADL} %, TADL\footnote{TADL: Timing Augmented Description Language version
%2. \url{http://www.timmo-2-use.org/timmo/index.htm}.}, AADL~\cite{aadl}, 
and AML~\cite{AML}. 
It contains an elaboration and unification of the viewpoints proposed in AAF and ADF and then proposes additional viewpoints, e.g.:

\begin{itemize}
\item {\em Feature viewpoint} - to be used to support the product line engineering. %; it is inspired to~\cite{EAST-ADL}, the only automotive ADL supporting product lines in the architecture description.
\item {\em Implementation viewpoint} - 
it describes the software architecture of the Electrical/Electronic (E/E) system in the vehicle~\cite{EAST-ADL}. %, supported by the system architecture
%and software architecture of AUTOSAR. 
\end{itemize}



% !TEX root = main.tex
\section{Software Development Challenges within Volvo Cars}\label{sec:lessonsLearned}

In a previous paper we studied within Volvo Cars,
from the architecture point of view,
the challenges that OEMs are facing in the last years~\cite{WICSA2015}. 
To better understand some of the
organizational issues with having different parts of the organization responsible for different parts or layers in the architecture, we decided to conduct nine in depth interviews with
focus on the roles of architects and how organizational factors affect them.
These interviews have been carried out at Volvo Cars  and Volvo Group Truck Technology
(VGTT), extending the knowledge about these specific companies, hence providing a detailed understanding of two independent but nevertheless similar automotive companies. %, information which is compared with the third, non-automotive, company. 
%The goal of this work was to build on previous results published in the
%literature {\bf Fix proper citations. [7], [6]}, to investigate the dual and
%complementary roles of software architects within the automotive domain.
In~\cite{WICSA2015} we 
%We 
followed a lightweight grounded theory-type approach since we started from a general research question and we involved few people in order to collect more information. Based on the analysis of the first data and on the first emerging ideas from the data, we then refined the research questions, carefully planned the study and accordingly selected the people to be interviewed. 
In addition to initial workshops in which we collected initial ideas, we used interviews as data generation methods. Specifically, we used semi-structured interviews: we defined a list of themes to be covered and questions we aimed at asking. However, in some interviews we changed the order of questions according to received answers and to the flow of the ``conversation". 
Each interview was around one hour long, we collected field notes and recorded audio. Each interview begins with introduction, clarification about the purpose of the study, asking permission to record and giving assurances of confidentiality of the information.  

Moreover, in these years of strong collaboration with Volvo Cars and thanks also to the industrial co-authors (the last two authors of this paper) we had the possibility to further improve the knowledge of to collect a set of insights, as described in the following subsections.

%In order to mitigate the threat to validity in this study we followed the guidelines for conducting and reporting case study research in software engineering~\cite{Runeson2009,Wohlin2000}.
%
%For what concerns {\em construct validity}, we performed semi-structured interviews by following a questionnaire that has been defined by exploiting knowledge collected through two workshops, as described above. The workshops enabled also to identify and set a proper language to be used during the interviews.
%In addition to that, interviews have been performed by an industrial PhD working at VCG and then expert of the domain. This permits us to be confident that interviewed architects correctly interpreted the questions, and, in turn, we correctly interpreted their answers. 
%
%For what concerns {\em external validity}, having two authors from VCG could be seen as a threat to validity due to the
%risk of the study becoming heavily focused and influenced from the situation at
%one company. 
%%However, having two authors from academia and being aware of
%%the risk of such a bias, permitted to actively counter it. Moreover, the first author is hired by VCG, but, for the main purpose of making his PhD studies. This puts him in the position of being a perfect observer, without having strong influences and bias from the company.
%We performed the study in two different companies in order to reduce this bias. Having interviewees from different levels in the organization and verify their
%claims against each other also work to counter threats to validity. %It is also interesting to mention here that in our study we found confirmations of results of previous studies as discussed in Section~\ref{sec:discussion}. This gives us some flavour of generalizability of the findings of the study.
%%towards only one company
%%is also why a second company was brought into the study making it possible to
%%validate findings between the companies. 
%
%Finally, for what concerns {\em reliability}, as mentioned before we performed the interviews by following a well defined questionnaire. This allowed us to collect comparable answers through the different interviews. However, this enables also another researcher to conduct the same study. Moreover, the finding we obtained clearly emerged from the performed interviews: in other words there was no room for interpretation, finding we discussed in the paper are evident in the collected data.

\subsection{Gap between Prescriptive and Descriptive Architecture}\label{sec:gap}

We identified that there is not always an obvious connection between  
architecture (or top-level design) and  design; it seems also that this connection vanishes during time, once development requires changes on the system design. % and once time passes and , especially during later development. 
The architecture is communicated as large documents,
%, or models, 
which are supposed to be read by stakeholders. However, this not always corresponds to the reality. Volvo Cars works also in cross functional teams and other type of communications are also used to communicate the architecture. More analysis is needed to understand the root of the problem. 
Moreover, maintenance of the
architecture, while the design evolves, is demanding and not
always performed in all parts. 
This shows a discrepancy between the planned architecture defined according to a V-Model process, and the architecture that is actually emerging from the system development. 

We identified that the architecture has a temporal aspect, that means: at any given point in time the system
has only one architecture, however, the architecture will change over time.
We call as {\em prescriptive architecture} the architecture that captures
the design decisions made prior to the system's construction\footnote{A discussion in depth about prescriptive and descriptive models might be found in~\cite{Models2016}.}. This architecture is the as-conceived or as-intended architecture.
We can call as {\em descriptive architecture} the architecture that describes how the system has been built. This architecture describes the as-implemented or as-realized architecture. We observed that there is often a discrepancy between the as-intended and the as-implemented architecture. This causes what is called {\em architecture degradation}. The degradation might show up in two different ways: (i) {\em architectural drift} when the descriptive architecture includes changes that are not included in, encompassed by, or implied by the
prescriptive architecture, but which do not violate any of the prescriptive architecture's
design decisions; (ii) {\em architectural erosion} is the introduction of
architectural design decisions into a system's descriptive architecture that violate its prescriptive
architecture.
 
When looking at the causes of architecture degradation, we can summarize the various reasons in the following three items: 

\begin{enumerate} 
\item During the development some important directives of the as-intended architecture have been violated due to time constraints, mistakes, misunderstanding, etc. 
\item Some of the architectural choices of the as-intended architecture were based on assumptions (might be implicit) that then were identified as imprecise or wrong, 
\item Some of the architectural choices of the as-intended architecture were made under uncertainty and during development these choices were judged as suboptimal.
\end{enumerate}


\subsection{Organization and Process Challenges}\label{sec:organizationAndProcessChallenges}
The first item of the causes of architecture degradation discussed in the section above (Section~\ref{sec:gap}) might be solved by improving communication as mentioned before or by finding ways to guarantee the preservation of the  important directives of the as-intended architecture that should not be violated for any reason.

On the organizational side, we found the need of improving the communication between
different groups, for instance by making teams more cross-functional. Today
there are several levels between architects and designers/developers and at some of
these levels the connection is not very tight. 

On one side this seems to be unavoidable since the company is 
big and there is the need of structuring the organization in sub-organizations (departments, groups, or teams). 
However, the risk is that sub-organizations will grow up independently and decoupled from other sub-organizations and this might 
stimulate a silos thinking.  
For instance, this might lead designers/developers to think
that the architects are sitting in their cloud above, without having
any connection to the reality. Architects might also feel
a frustration because they are not aware of everything that is happening; as a
consequence, a big part of their work is to just keep up with what is happening
in the construction groups. Espousing the terminology
in~\cite{IEEESoftwarePatrizio,whatindustrywants}, architects should be both ``Introvert" architects (conceptually related to the internal focus
of~\cite{Kruchten2008}), i.e., focused on more constructive work and definition of the architecture,
and 
``Extrovert'' architects (conceptually related to the external focus
of~\cite{Kruchten2008}), i.e. devoted to communicating the architectural
decisions and knowledge to the other stakeholders. 

The different organizations have different competencies, attitude, and
characteristics. However, new problems emerge. In fact, we can confirm that we
found many architecture antipatterns. We found the {\em Goldplating}
antipattern~\cite{Kruchten2008} since architects seem to be not
really engaged with developers. They are doing a good technical job, however,
their output is not really aligned with the needs of the developers and in the
end they are often ignored. Another antipattern that we found is the {\em Ivory
tower}~\cite{Kruchten2008}: the architecture team looks isolated
sitting on a separate floor from the development groups and do not engage with the
developers and the other stakeholders on a daily basis. This creates tensions in
the organization. 

It emerges the need to explore both organizational and technical
possibilities for tighter cooperation between architecture levels, and to
measure effects such improvements would have. On the technical side, one partial solution might be % idea is
to define specific viewpoints and to automatically generate corresponding architecture views from the design. 
%a framework able to automatically generate high-level views from the
%design. 
%The challenge here is to support multiple views, e
Each of the  views should focus on 
%devoted to 
showing only what is relevant for respective stakeholders. Moreover,
both architects and designers/developers need ways to perform early validation
of their solution and to sketch and try different visions of how the future
systems should look like; this will permit to understand the effect of design decisions on
the architecture. As mentioned before this can be only a partial solution since it cannot 
%However, this solution cannot 
solve the architecting problem since this solution only focus to visualize through specific viewpoints and views something that already exist. Other solutions are needed to solve the support just-in-time %agile 
architecting as mentioned above, as well as to enable stakeholders different from the architects, such as developers, to improve the architecture when actually needed.


%
%
%On the organizational side, we found a need to improve the communication between
%different groups, by finding a good balance between verbal and document based communication. 
%%.%for instance by making teams more cross-functional. 
%Espousing the terminology
%in~\cite{IEEESoftwarePatrizio}, system architects should be also
%``Extrovert'' architects (conceptually related to the external focus
%of~\cite{Kruchten2008}), i.e. devoted to communicating the architectural
%decisions and knowledge to the other stakeholders. 
%
%Often, the %se working architectures are developed in parallel by 
%different groups 
%%within the organizations, often having 
%have very different views on the meaning
%of architecture and how it shall be done. One must also note that, in general these groups are formed by domain experts, responsible for developing the mechatronic functions, not the optimized communication matrix or system architecture. In general we can identify the following types of conflict:
%(i) %what is 
%selecting the optimal solution for the %system (or function) 
%architecture; %?
%(ii) %which 
%selecting the platform %should 
%to be defined to facilitate development in the long run; %?
%(iii) %what should be 
%identifying the priority of the construction groups in order to deliver in time. %?
%%\jonn{The last one is unclear} \patrizio{To reduce the lenght of the introduction this part from ``One must also note that..." might be moved to another section.}
%
%%There is also not always an obvious connection between the high-level
%%architecture and the working architectures, especially during later development. 
%%The working architectures are
%%directly used by the construction groups. Since these architectures are forced to be consistent with the implemented system, they will inevitably diverge from the original design when the project evolves.
%%%They will inevitably diverge from the original design when the project evolves, as it is forced to be consistent with the implemented system. 
%%Contrariwise, the high-level architecture 
%%%only exists as 
%%is only communicated as large documents
%%%, or models, 
%%that are supposed to be read by stakeholders. Maintenance of the high-level
%%architecture, while the working architectures evolve, is demanding and not
%%always performed in all parts; consequently, it is difficult to determine the
%%quality of the high-level architecture description at any time.
%
%%In this sense we can say that the high-level
%%architecture has a descriptive nature, i.e., it aims at describing network design decisions and the main structures in the system architecture. The working architecture, on the other hand, is prescriptive, i.e., it defines interfaces, sets of rules, and recommendations, to follow during the implementation of the specific functionalities. 
%
%It emerges the need to explore both organizational and technical
%possibilities for tighter cooperation between architecture and design, and to
%measure effects such improvements would have. 
%%On the technical side, one partial solution might be % idea is
%%to define a framework able to automatically generate high-level views from the
%%low-level architecture. The challenge here is to support multiple views, each
%%devoted to showing only what is relevant for respective stakeholders. Moreover,
%%both high-level and low-level architects need ways to perform early validation
%%of their solution and to sketch and try different visions of how the future
%%systems should look like, to understand the effect of design decisions affect
%%the architecture. 
%%However, this solution cannot solve the architecting problem since this solution only focus to create a ``different view" for something that already exist. Other s





%Our findings are that 
%%for large and complex systems two different types of architectures, with
%%different abstraction levels, are used. A high-level architecture guiding and
%%diving the work between the construction or development groups is designed. Each of these groups creates 
%%a detailed communication matrix or working architecture\footnote{In the remaining of the paper we simply refer to this type of architecture as working architecture.} 
%%%Moreover, a detailed
%%%architecture is required 
%%to define strict interfaces and how the system implementation should be exactly realized. 
%%%The working architecture is monolithic since it is ``blamed" on the bandwidth, i.e. optimization of
%%%the communication.
%%%\jonn{it is quite central that this working arch is monolithic. Imagine a more
%%%internet-like architecture, it would have cased very different challenges. The
%%%monolithic architecture today is "blamed" on the bandwidth, i.e. optimization of
%%%the communication. Worth mentioning here??}
%often, %the %se working architectures are developed in parallel by 
%various groups within the same company 
%%within the organizations, often having 
%have different opinions on the meaning
%of architecture and how it shall be done. Example of topics in which we identified discrepancies are: 
%%. One must also note that, in general these groups are formed by domain experts, responsible for developing the mechatronic functions, not the optimized communication matrix or system architecture. In general we can identify the following topics of conflict:
%(i) %what is 
%selecting the optimal solution for the %system (or function) 
%architecture; %?
%(ii) %which 
%selecting the platform %should 
%to be defined to facilitate development in the long run; %?
%(iii) %what should be 
%identifying the priority of the construction groups in order to deliver in time. %?
%%\jonn{The last one is unclear} \patrizio{To reduce the lenght of the introduction this part from ``One must also note that..." might be moved to another section.}

%We identified that there is not always an obvious connection between  
%architecture (or top-level design) and  design, especially during later development. 
%The architecture is only communicated as large documents
%%, or models, 
%that are supposed to be read by stakeholders. However, this not always corresponds to the reality. Maintenance of the
%architecture, while the design evolves, is demanding and not
%always performed in all parts. 
%This shows a discrepancy between the planned architecture defined according to a V-Model process, and the architecture that is actually emerging from the system development. 

%\subsubsection{Towards agile architecting}\label{sec:agileArch}
%
%The role of the high-level architecture, if asking the architects themselves, is
%to serve as a set of guidelines and to identify the boundaries for the detailed
%design. In reality, this architecture is often, if not ignored, at least not in
%the mind of the engineers doing the low-level architecture on a daily basis. It
%emerged that low-level architects seldom, if ever, read the documentation
%produced by the high-level architects. Now, this does not mean that the work of
%the high-level architects is without value. However, having the high-level
%architects in a group of their own, separated from the other groups creates
%tensions between the different groups and a power struggle. 
%
%This shows a discrepancy between the planned architecture defined according to a V-Model process, and the architecture that is actually emerging from the system development. From the study it stems out in fact that the team responsible for the architecture, called high-level architecture above, tends to get isolated from the rest of the development organization, with few communications. This creates tensions within the organizations, as well as suboptimal design of the communication matrix and limited usage of the high-level architecture in the development teams. This clearly shows that in order to adapt to the current pace of software development and rapidly growing software systems new ways of working are required, both on technical and on an organizational level.
%
%What emerges here has been observed also in other domains. Specifically, this recalls the tension between waterfall and agile approaches. On one side waterfall approaches consider architecting as a phase of development that is somehow instructing the other phases. On the other side, agile development processes consider that ``The best architectures, requirements, and designs emerge from self-organizing teams.", as stated in the \#11 in the agile manifesto\footnote{Agile Alliance, Manifesto for Agile Software Development, June 2001 \url{http://agilemanifesto.org/}}.
%
%Building on that, Philippe Kruchten discusses the concept of ``agile architecture" that evokes two different interpretations (by quoting the text from the blog)\footnote{\url{http://philippe.kruchten.com/2013/12/11/agile-architecture/}}:
%
%\begin{itemize}
%\item a system or software architecture that is versatile, easy to evolve, to modify, flexible in a way, while still resilient to changes
%\item an agile way to define an architecture, using an iterative lifecycle, allowing the architectural design to tactically evolve gradually, as the problem and the constraints are better understood
%\end{itemize}
%
%These two interpretations are related but different; in fact as discussed by Kruchten, we may have a non-agile development process leading to a flexible and adaptable architecture, and on the other side, an agile process may lead to a rather rigid and inflexible architecture. The best would be an agile process, leading to a flexible architecture. 

%\subsubsection{Identifying the architects and their role}\label{sec:architects}
%
%In the same blog\footnote{\url{http://philippe.kruchten.com/2014/10/08/three-tures-architecture-infrastructure-and-team-structure/}} Kruchten points out also a conjecture that comes out at the XP 2014 workshop: architects typically work on three distinct but interdependent structures, which are:
%
%\begin{itemize}
%\item The architecture of the system under design, development, or refinement;
%\item The structure of the organization, including also partners, subcontractors, and others;
%\item The production infrastructure used to develop and deploy the system.
%\end{itemize}
%
%When these structures are not kept aligned over time, different kinds of ``debts" may show up: {\bf technical debt} when the architecture is lagging, {\bf social debt} when the structure of the organization is missing. 
%
%The alignment between the architecture and the production infrastructure is getting increasing interest with the concept of DevOps~\cite{Bass2015}, which puts the focus on combining the development organization with the operations organization, and on having the tools in place to ensure continuous delivery or deployment, even in the context of very large mission-critical systems, such as, Netflix, Amazon, and Facebook. Our paradigm is focusing on the evolution within the organization in contrast to the toolchain from the development organization to the customers, which is the main focus of DevOps.
%
%
%%\subsubsection{Organizational aspects}\label{sec:communication}
%
%On the organizational side, we found a need to improve the communication between
%different groups, for instance by making teams more cross-functional. 
%%Today
%%there are several levels between architects and implementers and at some of
%%these steps the connection is not very tight. This leads implementers to think
%%that the high-level architects are sitting in their cloud above, without having
%%any connection to the reality. On the other hand the high-level architects feel
%%a frustration because they are not aware of everything that is happening; as a
%%consequence, a big part of their work is to just keep up with what is happening
%%in the construction groups. 
%Espousing the terminology
%in~\cite{IEEESoftwarePatrizio}, system architects should be also
%``Extrovert'' architects (conceptually related to the external focus
%of~\cite{Kruchten2008}), i.e. devoted to communicating the architectural
%decisions and knowledge to the other stakeholders. 

%The different organizations have different competencies, attitude,
%characteristics. %However, new problems emerge. In fact, we can confirm that we
%%found many 
%We idneitifed some known architecture antipatterns, like %. We found 
%the {\em Goldplating}
%antipattern~\cite{Kruchten2008} since system architects seem to be not
%really engaged with developers. They are doing a good technical job, however,
%their output is not really aligned with the needs of the developers and in the
%end they are often ignored. Another antipattern that we found is the {\em Ivory
%tower}~\cite{Kruchten2008}: the %high-level 
%architecture team looks isolated
%sitting on a separate floor from the development groups and do not engage with the
%developers and the other stakeholders on a daily basis. This creates tensions in
%the organization. 
%
%It emerges the need to explore both organizational and technical
%possibilities for tighter cooperation between architecture levels, and to
%measure effects such improvements would have. 
%%On the technical side, one partial solution might be % idea is
%%to define a framework able to automatically generate high-level views from the
%%low-level architecture. The challenge here is to support multiple views, each
%%devoted to showing only what is relevant for respective stakeholders. Moreover,
%%both high-level and low-level architects need ways to perform early validation
%%of their solution and to sketch and try different visions of how the future
%%systems should look like, to understand the effect of design decisions affect
%%the architecture. 
%%However, this solution cannot solve the architecting problem since this solution only focus to create a ``different view" for something that already exist. Other s
%Solutions are needed to support agile architecting~\cite{shahrokni2016organic} %\footnote{\url{http://philippe.kruchten.com/2013/12/11/agile-architecture/}}  
%as well as to enable stakeholders different from the architects, such as developers, to improve the architecture, such as fixing wrong assumptions or making decision deliberately postponed.



\subsection{Towards ``Just-in-Time Architecture"}

The second and third items of the causes of architecture degradation discussed in Section~\ref{sec:gap} 
call for a ``{\em just-in-time architecture}" or agile architecting~\cite{shahrokni2016organic}  as well as to enable stakeholders different from the architects, such as developers, to improve the architecture, e.g. by fixing wrong assumptions or making decision deliberately postponed.

%
%Our analysis till now shows a discrepancy between the planned architecture defined according to a V-Model process,
%and the architecture that is actually emerging from the system development. According to what discussed at the
%above, this is a clear example of architecture degradation that can take the form of architecture
%erosion and/or architecture drift. From the study it stems out in fact that the team responsible for the architecture tends
%to get isolated from the rest of the development organization, with few communications. This creates tensions within
%the organizations, as well as suboptimal design of the communication matrix and limited usage of the high-level
%architecture in the development teams. This clearly shows that in order to adapt to the current pace of software
%development and rapidly growing software systems new ways of working are required, both on technical and on an
%organizational level.
%What emerges here has been observed also in other domains. Specifically, this recalls the tension between waterfall
%and agile approaches. On one side waterfall approaches consider architecting as a phase of development that is
%somehow instructing the other phases. On the other side, agile development processes consider that ``The best architectures, requirements, and designs emerge from self-organizing teams.'', as stated in the \#11 in the agile manifesto\footnote{Agile Alliance, Manifesto for Agile Software Development, June 2001 \url{http://agilemanifesto.org/}}.
One hypothesis made from some practitioners is that when developing large and complex systems ``a clear and well-defined architecture facilitates and enables
agility". This hypothesis implies that some upfront specification is needed when building complex products like cars.
However this hypothesis is not completely true when the product to be realized is not clearly defined and companies
want to go fast to the market (as for example observed by Waterman et al. \cite{WNA2015}. In these situations, modifiability, support for evolution, etc. are not really main aspects
to be considered. The hypothesis seems to be true when the product is well-defined. Another aspect to be considered
is that agile calls for refactoring, however refactoring often is not performed since the priority is given to what should
be realized.
%Further investigation is needed, however the conclusion we can draw at this point is that there is the need of a
%``just-in-time architecture" that enables even stakeholders that are different from the architects, such as developers, to
%improve the architecture, such as fixing wrong assumptions or making decisions deliberately postponed by the
%architects.




\subsection{Towards a Software Ecosystem Perspective}\label{sec:subcontractors}

Another interesting finding is that the architecture is not clearly considering the highly complex supplier-network that characterizes automotive engineering. 

%a high-level architect did not feel that
%there was a difference between using in-house developers and subcontractors.
%This might be due to the fact that the architects are more distant from the
%product. In the case of the people working with working architecture, this is
%not the case as we also found in our work~\cite{burden_comparing_2014}. They
%find it very frustrating to wait for part of the system to be integrated.  
%\patrizio{Eric, please add a description of the findings} Subcontractors and Architecture - Mozhan RE \cite{Soltani2015,Soltani2015a}
%	\begin{itemize}
%	\item How the architecture can better support the work with suppliers? \eric{Sure, but we had an RE Workshop paper, so it is clear that it will be from the requirements viewpoint}
%	\item How much information should be shared with suppliers (transparency)? \eric{I believe this is more like an outlook and needs to be researched in 2.4 and 2-2.4}
%	\end{itemize}
%
%\eric{Putting some content here, but of course it is not really about architecture.}
%Automotive engineering is characterised by a highly complex supplier-network.
%As a first step, we have investigated the impact of this complexity from the perspective of the requirements viewpoint based on a qualitative case study with an AUTOSAR Tier-2 supplier, a Tier-1 supplier and an OEM \cite{Soltani2015,Soltani2015a}\footnote{A video presentation of this work can be found at \href{https://oerich.wordpress.com/2015/08/14/how-does-the-autosar-ecosystem-impact-requirement-engineering/}{https://oerich.wordpress.com/2015/08/14/how-does-the-autosar-ecosystem-impact-requirement-engineering/}}.
%Through seven semi-structured interviews, we found that a clear, 
In a previous study~\cite{Soltani2015,Soltani2015a} 
we found that a holistic strategy for aligning work across the value-chain is currently missing. 
Specifically, mixing commodity and differentiating components lead to sub-optimal situations, resulting in duplicated work (an observation in line with \cite{OB2015}).  
We argue that automotive architecture needs to assume a holistic perspective with respect to the whole value-chain and optimize the architecture for facilitating beneficial subcontractor interaction. %\eric{perhaps we need to cite Helena and Jan here, no?}
\begin{itemize}
\item \textit{Commodity Components} require clearly defined technical and organizational interfaces. 
The goal is to develop them as efficiently as possible, thus reducing coordination overhead. 
Ideally, of-the-shelf commodity components can be integrated with minimal adjustment. 
\item \textit{Differentiating Components} should be developed as independent from the commodity components as possible, probably in-house. 
\item \textit{Innovative Components} naturally require coordination and iterative work between a number of partners. 
To effectively develop innovative behaviour, could communication channels need to be established. 
\end{itemize}

%This calls for a proper management of the automotive ecosystem%~\cite{knauss2014towards}. %
%, which is characterized by relying heavily on complex supplier networks, 
%and strong dependence on hardware and software development~\cite{knauss2014towards}.

In traditional software engineering, a software product is often the result of an effort of a single independent software vendor, investing into creating a monolithic product \cite{jansen2013defining, wnuk2014evaluating}.
Modern software engineering strongly relies on components and infrastructure from third-party vendors or open source suppliers \cite{jansen2013defining, wnuk2014evaluating}.
The emergence of Software Ecosystems (SECOs) is a recent development within the software industry \cite{hanssen2012longitudinal, wnuk2014evaluating}.
It implies a shift from closed-organizations to open structures where external actors become involved in the development to create competitive value \cite{jansen2013defining, hanssen2012longitudinal}.

Based on the ecosystem classification model \cite{jansen2013defining}, we  understand the automotive value chain as an ecosystem of cross-organizational collaborations among automotive suppliers~\cite{knauss2014towards}. 
In this ecosystem, the Original Equipment Manufacturer (OEM) is in the role of the ecosystem coordinator.
The ecosystem is privately owned, and participation to it is based on a list of screened actors. 

The automotive ecosystem is characterized by relying heavily on complex supplier networks, and strong dependence on hardware and software development \cite {knauss2014towards}.
Due to the increasing number of networked components, a level of complexity has been reached which is difficult to handle using traditional development processes \cite{fennel2006achievements}. 
The automotive industry addresses this problem through a paradigm shift from a hardware-, component-driven to a requirement- and function-driven development process, and a stringent standardization of infrastructure elements. %\eric{Patrizio, does this clash with your earlier reasoning? Please feel free to adjust.}
%One central standardization initiative is the AUTomotive Open System ARchitecture (AUTOSAR) \cite{fennel2006achievements}. 

The principal aim of the AUTomotive Open System ARchitecture (AUTOSAR) standard is to master the growing complexity of automotive electronic architectures \cite{furst2009autosar}.
We refer to the \emph{AUTOSAR ecosystem} as a subset of the \emph{automotive ecosystem}, where different actors participate in value creation (i.e. development of software components) by exchanging products and services based on a technical platform defined by the AUTOSAR standard. %\eric{perhaps too much focus on AUTOSAR? Then could delete starting from ``One central standardization initiative...''}

Several challenging areas, including requirements engineering, are reported in the automotive domain \cite{broy2006challenges}.
OEMs and suppliers need to communicate requirements based on the requirements documents, which are imprecise and incomplete nowadays \cite{broy2006challenges}.
In this regard, \cite{fricker2010requirements} reports that the requirements value chain is little understood beyond software projects. 
It is unclear which requirements communication, collaboration, and decision-making principles lead to efficient, value-creating and sustainable alignment of interests between interdependent stakeholders across software projects and products \cite{fricker2010requirements}. 
This however is important, because the way the interests and expectations of stakeholders of SECOs are communicated is critical for whether they are heard, hence whether the stakeholders are successful in influencing future solutions to meet their needs \cite{fricker2009specification}.
We note that in the automotive domain, communication, collaboration, and decision-making are cross-organizational challenges related to the requirements viewpoint and requirements engineering.


%\begin{itemize}
%\item \textit{Commodity Components} require clearly defined technical and organizational interfaces. 
%The goal is to develop them as efficiently as possible, thus reducing coordination overhead. 
%Ideally, of-the-shelf commodity components can be integrated with minimal adjustment. 
%\item \textit{Differentiating Components} should be developed as independent from the commodity components as possible, probably in-house. 
%\item \textit{Innovative Components} naturally require coordination and iterative work between a number of partners. 
%To effectively develop innovative behaviour, could communication channels need to be established. 
%\end{itemize}

% !TEX root = main.tex

\section{Volvo Cars architecture framework}\label{sec:VCGAF}

The starting point for defining an architecture framework is to start from the identification of established stakeholders within the domain of the framework. Stakeholders may be individuals, teams,
organizations or classes (of individuals, teams or organizations), while concerns may be fine-grained or very broad in scope~\cite{Emery-Hilliard:2009}. 

%Table \ref{tab:stakeholders} describes the main actors in the challenging scenarios.
% \todo[inline]{elaborate and describe. Also: Include other stakeholders such as customer, driver, ...}

%A suitable electrical architecture needs to take into account the specific needs of each of these stakeholders. 
%In the next section, we will discuss \emph{challenging scenarios} that explore how these stakeholders will interact with the electrical system, its architecture, and its development.

\begin{table}[htb]
\scriptsize
\caption{Overview of Stakeholders}
\label{tab:stakeholders}
\begin{tabular}{rv{0.14\textwidth}v{0.2\textwidth}v{0.26\textwidth}v{0.2\textwidth}}
\toprule
& \textbf{Stakeholder} &	\textbf{Group} & \textbf{Comment} & \textbf{Synonyms} \tabularnewline
\midrule
$\bullet$& Passengers & end-user	\tabularnewline
$\bullet$& Drivers & end-user		\tabularnewline
$\bullet$& Customers & customer & Purchaser of a car or related service & \tabularnewline
$\bullet$& Product planner & customer & Acquirer of electrical system		\tabularnewline
$\bullet$& Purchaser & customer & Purchasers of electrical system		\tabularnewline
$\bullet$& Line managers & management & Has scheduling responsibility, long term quality responsibility, includes group, department	\tabularnewline
$\bullet$& Project managers & management & Owns budget for development	\tabularnewline
$\bullet$& System architects & developers of electrical system & 	\tabularnewline
$\bullet$& Functional developers & developers of electrical system & Owns functional and non-functional aspects & function owner; function realizer; function developer, function realizer, system developer\tabularnewline
$\bullet$& Component developers & developers of electrical system		\tabularnewline
$\bullet$& SW supplier (internal/external) & developers of electrical system	& Can be internal or external from the perspective of the OEM.	\tabularnewline
$\bullet$& HW supplier (internal/external) & developers of electrical system	& Can be internal or external from the perspective of the OEM.	\tabularnewline
%SW supplier (external & developers of electrical system		\tabularnewline
%HW supplier (external) & developers of electrical system		\tabularnewline
$\bullet$& Testers & developers of electrical system		\tabularnewline
$\bullet$& Attribute Owners & developers of electrical system & Owns non-functional attributes like performance	\tabularnewline
$\bullet$& Tool Engineers & developers of electrical system & Specifically testing tools, including design tools (e.g. for requirements)	\tabularnewline
$\bullet$& Calibrators & developers of electrical system & \tabularnewline
$\bullet$& Diagnostic method engineers & maintainers of electrical system		\tabularnewline
$\bullet$& Workshop Personnel & maintainers of electrical system		\tabularnewline
$\bullet$& Fault Tracing Specialists & maintainers of electrical system		\tabularnewline
$\bullet$& Technical Specialist &  specialists &	Support developers and maintainers on specific topics \tabularnewline
\bottomrule
\end{tabular}
\end{table}

Table~\ref{tab:stakeholders} describes the main stakeholders we have identified; they fall into five major groups: 

\begin{itemize}
\item \emph{End-users} of the electrical system, like drivers and passengers.
\item \emph{Customers} stakeholders, such as paying customers of products and services that depend on the electrical system (i.e. the car) and product planners, who acquire the electrical system as part of the overall product.
\item \emph{Management} with responsibility for scheduling, long term quality, groups, departments, and budget.
\item \emph{Developers of the electrical system}  include engineers throughout the value chain that create the electrical system, its architecture, and the necessary tools as well as that test and integrate the various components. 
\item \emph{Maintainers of the electrical system} who interact with the electrical system throughout its lifetime. 
\end{itemize}


Then the identified stakeholders motivate the set of concerns on which the architecture framework will focus.
This will help the consumer of the architecture framework and of the views and connected modeling tools to understand why
they are modeling and when they are done.
In order to define the stakeholders concerns we identified a set of challenging scenarios through a number of workshops for elicitation and validation. 
Figure~\ref{fig:challenging-scenarios} gives an overview of the scenarios that are strongly connected to the viewpoints we will detail in this paper. These scenarios are described in the following items.
\magnus{Rogardt and I made a draft picture with the use cases from WP3 added.
Color, exact naming, bubble placement etc., TDB.
Do you think this is a workable way forward?}\patrizio{Magnus, might you please change the figure according to the new text I wrote? I don't have the source files.}

\begin{figure}[htb]
\begin{center}
\includegraphics[width=\textwidth]{figures/use_cases}
\caption{Overview of challenging scenarios}
\label{fig:challenging-scenarios}
\end{center}
\end{figure}


\begin{itemize}
\item Scenario 2.1 {\bf decision management } aims at exploring how to make, communicate, and manage decisions. 

\begin{quote}
{User Story:} \emph{``As a member of the development ecosystem I would like to have a clear understanding on how to take decisions and how to communicate them.''}
\end{quote}


Interesting sub-scenarios include decisions about:

\begin{itemize}
\item  {\em Requirements (2.1.1)}: When decisions on requirements are made too early, it will lead to unnecessary changes.

%\textbf{Trigger:} Requirement needs to be refined so that others can continue working.

\begin{quote}
{User Story:} 
\emph{``As a component responsible, I need to write a requirement. 
Currently, I am forced to write the requirement in a certain document or certain structure. 
This determines whether the requirement refers to a hw or sw component, whether it will be implemented in-house or by an external supplier. 
Often, it is too early to make such decisions and changes are necessary later, when more information becomes available. 
I do not feel comfortable to make this decision so early.''}
\end{quote}

\item {\em Architecture (2.1.2)}: Architectural decision making involves making the right decision, communicating it, ensuring that it is followed, and changing it when needed. 

%\textbf{Trigger:}  Architectural decision required (it is beneficial and possible). 

\begin{quote}
{User Story:} 
\emph{``As an architect} (one of \{system architect; functional developer; low level architect\})\emph{, I need to make the right decision at the right moment (i.e. when it is useful to make the decision and when the necessary information is available).  
I need to make this decision on the right level. 
I need to be introvert to make the best possible decision on the available data and extrovert to communicate it.''}
\end{quote}

\item {\em Customer functions (2.1.3)}:  Electrical architecture is guiding realization of customer functions, but it is not obvious how the architecture support customer functions (with respect to tracing and methodology).

%\textbf{Trigger:}  Decision about customer function required. 

\begin{quote}
{User Story:} 
\emph{``As a product manager, I want to be supported by the electrical architecture in making decisions about customer functions. Based on a wishlist from the Market Analysis Department, I engage in a dialog with the departments that design the system. 
For this task, I wish for support from an electrical architecture that not only takes into account non-functional aspects, but also the nature of customer functions (which changes over time).''}
\end{quote}

\item {\em Change management (2.1.4)}: Good flexibility of the architecture allows us to continuously remove assumptions and do changes even late in the process. 
However, in a weak electrical architecture, such changes will impact the stability of the electrical system because of technical dependencies. 

\begin{quote}
{User Story:} 
\emph{``As a product planner I want to have a flexible Change Management process, allowing me to change or add functions late in the process, often with the goal of removing assumptions. 
This includes for example defining and changing the allocation of Functions to ECU late in the process.''}
\end{quote}
\end{itemize}
 
 
\item Scenario 2.2 {\bf define/evolve architecture} explores aspects of long-time evolution of the electrical architecture.
This includes:

\begin{itemize} 
\item {\em The impact of long-term sourcing decision on the logical architecture (2.2.1)}: Long term sourcing contracts allow to optimize production cost, but constraint evolution of the architecture. 
\begin{quote}
{User Story:} 
\emph{``As a Function Developer I want to optimise the evolution of functions without considering sourcing agreements. 
Today I have to discuss with the product planner about the plan for a function two years or more in advance to reflect on existing contracts, e.g. when related nodes are sourced for different time intervals. Problem: Sourcing needs make physical layout dominate logical decisions.''}
\end{quote}

%Architectural frameworks such as AUTOSAR can support functional evolution on component level, but addressing this scenario would also require more  thinking on how to support moving around functions on high level architecture.

\item {\em The danger of architecture and design  evolving in different directions (2.2.2)}: If not actively managed,  architecture and design diverge over time. 
The architecture is then perceived as outdated and not useful, thus it looses its ability to guide design decisions and implementation.

\begin{quote}
{User Story:} 
\emph{``As a system architect or function developer I want a stringent correlation between architecture and design. Otherwise, one or the other is wrong.''}
\end{quote}
\end{itemize}

\item Scenario 2.3 {\bf flexibility of architecture} addresses different scenarios that emphasize flexibility of the electrical architecture, both on a technical and on a process level. On a technical level, more capacity in the ECUs allows to add functionality late. 
On a process level, the available resources need to be managed across several contributing partners.
This includes:

\begin{itemize}
\item {\em application workload management from a process perspective (2.3.1)}: 
\begin{quote}
{User Story:} 
\emph{``As a functional developer, I want to be flexible when it comes to application workload. 
Suppliers should be able to acquire and return resources dynamically throughout the lifecycle.''}
\end{quote}
\item {\em application workload management from a technical perspective (2.3.2)}: higher capacity of ECUs and Buses will facilitate late or even very late updates (i.e. when the vehicle is on the street), because new functionality could use more resources.
This flexibility comes at a cost and it is not trivial to understand where the break-even point is.

\begin{quote}
{User Story:} 
\emph{``As a system architect I want to balance capacity of ECUs and Buses against cost.''}
\end{quote}

\item {\em easy and secure add-ons (2.3.3)}: being able to add new functions in a secure and easy way would allow to detach software development to some extend from the development cycle. 
During the development of a car, an OEM could focus on the critical basic functionality. 
Convenience features as well as more advanced connected features could be added independent from the start of production.
This would allow to develop more continuously and should decrease the peak of trouble reports before critical deadlines observed earlier~\cite{modelsward13}.

\begin{quote}
{User Story:} 
\emph{``As a  functional developer I want to be able to add new functions (very) late. Such add-ons could originate from third parties and should be added even after the vehicle has been put into use.''}
\end{quote}

\item {\em separation of concerns (2.3.4)}: 
\begin{quote}
{User Story:} 
\emph{``As a system architect, I want to balance separation of concerns on two levels: between domains (e.g. safety vs. infotainment) and levels of abstractions (e.g. architecture vision vs architecture implementation).''}
\end{quote}

\item {\em flexible functionality (2.3.5)}:
\begin{quote}
{User Story:} 
\emph{``As a functional developer, I want to be flexible about the functions that are running in the car and allow for very late deployment.''}
\end{quote}
\end{itemize}
 
\item Scenario 2.4 {\bf continuous deployment and transparency} includes:

\begin{itemize}
\item {\em the need for openness and transparent information through out the different value chains in the automotive ecosystem (2.4.1)}: Good transparency in the value chain supports flexibility, since all partners can take appropriate action to reach a (changing) goal quickly. 
The \emph{cone of uncertainty} is a good example for this~\cite{cone-of-uncertainty,cone-of-uncertainty2}: In the beginning of a project, not much data is available and decisions are very uncertain. 
As the project proceeds, assumptions are removed and uncertainty is reduced, but as  a consequence, it becomes harder to make decisions that change a lot. 
Good transparency can help to reduce the uncertainty quicker and can allow to decide fast in order to learn fast.
\begin{quote}
{User Story:} 
\emph{``As any developer of the electrical system (internal or external), I want to have access to relevant decision, to the status of relevant assumptions, and knowledge generated during the development. 
This allows me to participate in the fast learning throughout the value chain and enables me to be effective and flexible in my work.''}
\end{quote}

\item {\em the need for using this information to continuously improve an continuous integration, delivery, and deployment flow (2.4.2)}: 
Cross-organizational continuous integration, delivery, and deployment facilitates fast feedback and rich learning.
This learning should also help to improve the interaction of organizations and the integration flow between them.

Have a culture of continuous improvement between VCC and partners	Apply Continuous Deployment to the CD tool chain

\begin{quote}
{User Story:} 
\emph{``As a Project Manager, I want to have a culture of continuous improvement within the OEM and with partners in the value-chain. As any developer of the electrical system (internal and external) I want to benefit from and take responsibility in improving the continuous delivery flow.''}
\end{quote}

\item {\em the need for establishing short feedback cycles (2.4.3)}: While developing new functionality, basic software, and hardware, one should plan on how to receive feedback, which data to collect, and how to use it in the development.

\begin{quote}
{User Story:} 
\emph{``As any developer of the electrical system (internal or external) I want to have quick feedback on how my contribution will work on the various levels of integration. 
As a Functional Developer, I want to have fast and defined feedback cycles. 
As a tester, I want quick updates on all levels of tests and continuous improvement of functionality.''}
\end{quote}
\end{itemize}



\item Scenario 2.5 {\bf System of Systems} includes:

\begin{itemize}
\item {\em the need of having reliable communication (2.5.1)}: To have the car as a constituent of a System of Systems (SoS), the car needs to be connected to other cars, infrastructure, cloud and any other kind of constituent system of the SoS. Most probably the car will be connected through heterogeneous communication means and with various degree of quality of service.

\begin{quote}
{User Story:} 
\emph{``As an architect I need to engineer the car so that it will support heterogeneous communication means with various quality of service (QoS). The QoS of the different communication means should be clearly identified so that it will be possible to develop end-to-end functionalities that go beyond the boundaries of the car.''}
\end{quote}

\item {\em guaranteerning that communication from and to the car is properly understood (2.5.2)}: It is not enough to have proper communication means; the exchanged messages should be properly understood by all involved parts.

\begin{quote}
{User Story:} 
\emph{``As an architect I need to ensure a high degree of interoperability between the car and other constituent systems of the SoS. End-to-end functionalities that go beyond the boundaries of the car might require to build under the assumption that exchanged messages are properly understood and required actions are taken by the receiving system.''}
\end{quote}

\item {\em Degree of autonomy and readiness to be at the service of the SoS (2.5.3)}: Often SoS scenarios require a degree of autonomy of the vehicle. Moreover, often a constituent system has to temporarily sacrifice its own goal in order to take actions needed to fulfil the goal of the SoS.

\begin{quote}
{User Story:} 
\emph{``As an architect I need to manage the transition towards autonomous behaviours of the car that are needed to achieve the goal of the SoS. The car should be engineered so to, often immediately, switch to another mode and perform the actions required by the SoS. The right tradeoff between independence of the vehicle (from the SoS) and service offered to the SoS needs to be found.''}
\end{quote}

\item {\em Dealing with uncertainty and functional safety (2.5.4)}: Vehicles are starting receiving a huge amount of information from the environment as communication coming from other vehicles, pedestrians, road signals, city in general. This information might be precious for supporting new safety behaviours\footnote{Here there is an example: \url{https://www.media.volvocars.com/global/en-gb/media/pressreleases/159478/volvo-cars-connected-car-program-delivers-pioneering-vision-of-safety-and-convenience}}. However, this information is often unreliable and subject to different dimensions of uncertainty, like presence of other constituent systems, communication means, etc.

\begin{quote}
{User Story:} 
\emph{``As an architect I would like to support innovative and very promising safety scenarios that can involve cyclists, pedestrians, etc. However, this would imply that the way functional safety is ensured should change since the SoS is open (constituent systems might join or leave the SoS at any moment) and we cannot relay 100\% on information sent by uncontrollable and independent constituent systems.''}
\end{quote}


\item {\em Cyber security and privacy (2.5.5)}: Once the car is connected it is exposed to attacks as any other computer or device that is connected to Internet. The effects might be tragic.

\begin{quote}
{User Story:} 
\emph{``As an architect I need to protect the car from attacks to avoid dangerous or catastrophic scenarios and to guarantee the privacy of user data.''}
\end{quote}


\end{itemize}


%
%{\bf System of Systems} includes:
%\patrizio{We need to add scenarios for the System of Systems part; unfortunately this is a bit decoupled in NGEA but we can recover that from WP3}
%\magnus{Placeholder text below, should perhaps be part of the exposition at Fig.~\ref{fig:challenging-scenarios}}
%\rogardt{Yes, it is not too impressive the use cases that we come up with in WP3.x, and the use cases Mangus and I picked are supposed to be the best once, so you can think how bad the other once are. Should we make new use cases ourselves, ignore 3.x use cases?}
%
%%It was a different group of people involved in finding these use cases. %When the other
%%use cases were picked on current challenges together with Volvo, these use %cases occur over several workshops with divers group, . 
%
%We had several workshops to identify different use cases for system of systems for the automotive. The use cases were identified by the use of brain storming, focusing on potential future business opportunity. 
%Since many of these use cases will have similar impact on the 
%architecture, we only picked the three
%\rogardt{We should just remove the two use cases that Patrizio don't find interesting in regards to system of systems}
%most interesting and dimensioning 
%once, base on:
%%From this work we picked the five most interesting and dimensioning Use %Cases based on:
%\begin{itemize}
%    \item Type of actors (More actors require more complex communications)
%    \item Amount of information exchange (The more info to exchange the broader the bandwidth needed)
%    \item Speed in interaction (Quicker response require short latencies)
%    \item Availability/Continuous connection (Use cases sensitive to data loss etc.)
%    \item Safety (if use case is not correctly executed it can cause harm) \rogardt{is this really about system of system concern? Should we remove it?}
%    \item Integrity/Security (Actors authorised, information trustworthy) \rogardt{Again, very important, but is this about System of systems, and in addition Integrity is Security, so it should be maybe just Integrity or Security or nothing at all?}
%    \item Sharing information (People/System must be willing to share information) \rogardt{sound like security}
%    %Found in the excel sheet  \patrizio{Do we have them in some deliverable %that we can refer?}
%\end{itemize}.
%
%
%\begin{itemize}
%\item {\em Remote control of the vehicle (2.5.1)}
%\begin{quote}
%{User Story:} 
%\emph{``The driver does not take back control from the vehicle being in automated driving mode when prompted to. The call center is informed and an operator takes over the vehicle control remotely and guides the vehicle to a safe stop.''}
%\end{quote}
%
%In the future, if we will have self-driving cars, there might be cases where one might want to take over the control of the car, but not necessary the driver as in this use case. For example, the driver might
%have become ill and there are a situation that require human interaction. 
%This require the possibility of interacting in safe way with the vehicle from outside.  \rogardt{When I think about it, is this really a system of system use case, is this not mostly about an interface to the car?}
%%When vehicle become
%%Time-out reached for manual control - message sent to - call center - call %center operator overtakes control.The vehicle is remotely guided by an %operator in a call center.
%
%\rogardt{A more interesting use case (I just made them up now): (1) AD vehicle want to park as close as possible to place where passengers want to be. The vehicle obtain information about the closes parking places, ask for available places and drive to the one that is closes parking place where there are available places. (At least three systems: Vehicles, Parking places, and GPS) (2) There is a road sign to reduce the car speed and give instruction on how to place the vehicle. The vehicle inform the passengers and follow the instructions. The vehicle are told when to take over the control again. (At least two systems: Vehicles and Road Sign (3) The driver have sensors to indicate his status, in case of fatal illness the sensor system signal to the car to come to a safe stop, and send the GPS position to a SOS center (kind of similar to the previous one, the only different is that there are a system to monitor the human:-}
%\rogardt{In the three cases above you have system to systems interaction, is that not the whole point? Should we just make our own use cases, that have an interesting impact on the architecture, I guess it should be: One that have input to the vehicle from another system, one that have output from the vehicle to another system, and a process (input and output)???? }
%\rogardt{Question: from a system of System perspective I guess the communication would be similar, some sort of protocol. I guess the main different is how much can you trust the information, and how much should you relay on it }
%\rogardt{I think we have worked with the wrong questions in SoS, we should have focused on: What impact can another systems have on Vehicles, how can you trust them, what happen when you get disconnected, what part of the architecture can you move outside the vehicles, ... }
%\rogardt{**************************************}
%\rogardt{meet with Mangus this afternoon, maybe we should make
%completely different type of use cases, maybe the use cases we produced in 
%WP3.5 is not the correct one, they should be more like the one we made in WP2.2, influence the architecture, for example:
%Main use case: 1. Communication with other systems, 1.1: Control of car, 1.2 trusted information, 1.3: Share information ...
%Find all the use cases that is necessary for a vehicle to work in a system of system context. For example, in the use case "Share information" one need to model what is public and what is private. For control of car, one need to have interfaces, and one ... ...}
%\rogardt{***************************************}
%
%\item {\em Approaching vehicle warning (2.5.1)}
%\begin{quote}
%{User Story:} 
%\emph{`` The vulnerable road user is warned in case the vehicle threatening to hit him/her. The warning is issued via the phone or other device that the VRU is carrying. ''} \patrizio{is this similar to the ice on the street? \url{https://www.media.volvocars.com/global/en-gb/media/pressreleases/159478/volvo-cars-connected-car-program-delivers-pioneering-vision-of-safety-and-convenience}}
%\rogardt{I think this was people walking or cycling getting warning, we need to check this. I think this could be an interesting case}
%\end{quote}
%
%
%
%\item {\em Crash notification (2.5.3)}
%\begin{quote}
%{User Story:} 
%\emph{``In case of a crash, notifications are automatically sent to preselected contacts. Can contain vehicle info (registration number, VIN, etc.), personal info (name of driver, ID-number, blood group, etc.), vehicle position and photos (inboard and outboard), etc.  ''}\patrizio{not so clear a SoS perspective here}
%\end{quote}
%
%\item {\em Road prescribed settings (2.5.4)}
%\begin{quote}
%{User Story:} 
%\emph{`` Conditions on the AD road are such that the Traffic control Center prescribes or recommends a certain behavior of the vehicles in AD mode. This could be recommended speed, headway or OK/NOK to AD, etc. (due to traffic density, bad weather, emergency vehicles, …) ''} \patrizio{This can be merged with the ``vulnerable road user" user story above (two items above)}
%\end{quote}
%
%\item {\em Pull over request (2.5.5)}
%\begin{quote}
%{User Story:} 
%\emph{``The vehicle being in automated driving mode, is overtaken by a police car and prompted to pull over and stop at the roadside. The vehicle recognizes this and performs the maneuvers necessary.''} \patrizio{not much SoS. This is can be interesting if we want to show security needs: if the police can take control, any hacker will be able to take control ;)}
%\rogardt{That was the point, it had high risk of being misused and one of the reason for picking it}
%\end{quote}
%
%\item {\em Collision avoidance with cyclists (....)}
%\begin{quote}
%{User Story:}
%\emph{``A cyclist wears an helmet that connects to the cloud through its smartphone. If the bicycle is on a collision course with another bicycle or a car, a notification will be sent to both of them. The helmet will warn the cyclist with a notification light and the car will display the warning on a heads up display on the windshield.''}\\\\
%\url{http://analysis.tu-auto.com/auto-mobility/volvo-connecting-cars-cyclists-safer-mobility} and \url{http://www.mirror.co.uk/news/technology-science/technology/volvos-connected-helmet-warns-cyclists-4852610} 
%
%\end{quote}

%\end{itemize}

\end{itemize}





The identified architecture-related concerns determine the choice of viewpoints and view to be included. 
It is important to note that the concepts involved in almost each viewpoint are already handled within the current architecture of Volvo Cars. However, we are investigating the definition of proper viewpoints that will create the architecture framework according to the ISO 42010 international standard~\cite{42010}. %of the
%Viewpoints are the principal content of an architecture framework. In fact the viewpoint defines the conventions and establishes the basis for interpreting views. Moreover, ``{\em each viewpoint establishes the notations, models, techniques and methods to be used in architecture descriptions resulting from applying the framework.}"~\cite{Emery-Hilliard:2009}.
%\patrizio{here we have to explain that there are the viewpoints in the litareture, other important viewpoints for Volvo Cars, but that in this paper, due to space restrictions we only focus on three viewpoints}
Most of the viewpoints summarized in Section~\ref{sec:AAF} are indeed interesting also for Volvo Cars. In addition to these viewpoints we foresee a set of viewpoints that are emerging from the new challenges of the automotive domain, as explained in Section~\ref{sec:lessonsLearned} and illustrated by the scenarios described above. The new viewpoints are listed in the following; then for space constraints in the reminder of the paper we will focus on three of the most challenging ones:
	\begin{itemize}
	\item \emph{Continuous integration and deployment} (detailed in Section~\ref{sec:CID_VP}) - OEMs are increasingly interested to reduce the development time, to increase flexibility, to have early feedback on decisions made, and to add new functionalities incrementally even after production. %However, another trend is the rapid development of more advanced active safety systems that require \del{a} special handling.\eric{does the last point here not relate more to connected cars and safety?}
		\item \emph{Ecosystem and transparency} (detailed in Section~\ref{sec:ET_VP}) - related to the value net viewpoint of AAF~\cite{TUM-I0915,Broy}. %activities of, and the dependencies between the stakeholders of the end-toend value creation process which happens within a specific value net. 
		The ecosystem around the OEM can be seen as a virtual %A value net is defined as a virtual
	organization consisting of the OEM, its suppliers and other partners %and other constituents 
	involved in the process of creating customer value.
	\item \emph{System of systems viewpoint: vehicle point of view} (detailed in Section~\ref{sec:SoSVP}) - Future scenarios of collaborating autonomous vehicles are posing new challenges on the vehicle architecture. This viewpoint aims at understanding how the architecture should change in order to engineer a car that is a constituent of a system of systems.
%	like platooning, will require to extend the vehicle architecture across the classical boundaries of single vehicles and will ask for an open and adaptive architecture able to support runtime assessment of safety. 
	%\item \emph{Security and privacy of connected cars} - Connected cars open new important challenges from the point of view of security and privacy.
	\item \emph{Autonomous cars} - autonomous cars require special architecture solutions, e.g. inspired to autonomous and self-adaptive systems~\cite{Salehie2009}.
	\item \emph{Modes management} - a mode viewpoint is needed to design the different modes of a vehicle as well as the transitions from one mode to another.
	\item Special viewpoints and views might be conceived to enable dissemination and communication of the architecture to developers and other stakeholders.
	\end{itemize}

%In order to define the viewpoints we use a template similar to the one suggested in~\cite{Yania}:
%\begin{itemize}
%\item Definition: Definition of the viewpoint is presented.
%\item Stakeholders: Although the stakeholders are not explicitly identified for the viewpoints
%in the AAF and ADF, we list the stakeholders.
%\item Concerns: Stakeholder concerns are defined.
%\item Views: The views governed by the viewpoints are presented.
%\item Model kinds: The model kinds used in the viewpoint are presented.
%\end{itemize}



The natural consequence of the use of multiple viewpoints and views in architecture
descriptions is the need to express correspondences and consistency rules between those views.
The mechanisms introduced in~\cite{42010} is called model correspondences and it allows the definition of relations between
two or more architecture models. Since architecture views are composed of architecture models~\cite{42010}, model correspondences can be used to
relate views to express consistency, traceability, refinement or other dependencies~\cite{Emery-Hilliard:2009}.
These mechanisms allow an architect to impose constraints between types of models and then demonstrate that them
are satisfied by the architecture. 






%
%
%
%Architecture descriptions take many forms and
%serve many purposes throughout the life cycle of development,
%operation and maintenance activities. The use of {\em multiple views}
%- diverse representations for distinct audiences and uses - has
%been a major tenet of architecture description since the earliest
%work in software architecture. 
%The use of multiple views has become standard practice
%in industry~\cite{4+1,ICSE2010,42010}. A survey recently conducted
%on the industrial needs from architectural languages~\cite{whatindustrywants} (see Section~\ref{sect:survey})
%revealed that 85\% of the 48 interviewed practitioners use
%multiple views when architecting a software system, with
%a total of nine different views and a predominant use of
%structural, behavioral, and physical views reported. 

% !TEX root = main.tex
%%% Version 2.1b %%%

\section{Continuous Integration and Deployment viewpoint}\label{sec:CID_VP}
%%%%%%%%%%
\renewcommand{\Fillin}[1]{{Continuous Integration and Deployment}}
\subsection{\Fillin{Viewpoint Name}}\label{vp:template}
%%%%%%%%%%

%\del{\must{Provide the name for the viewpoint.}}

%\del{If there are any synonyms or other common names by which this viewpoint is
%known or used, record them here.}

The viewpoint ``Continuous Integration and Deployment'' adds a development perspective to the system %\del{electrical}\eric{Should there not be some qualification? Should we perhaps say system architecture?} 
architecture and aims to give an answer to the following questions:
\begin{enumerate}
\item How do continuous integration and deployment practices in automotive engineering impact the system architecture?
\item How do architectural decisions in the system architecture impact continuous integration and deployment practices?
\end{enumerate}
% \eric{I think it is crucial that we align on those questions. First draft above, please comment. I am not sure I hit a good level of ganularity.}

%%%%%%%%%%
\subsection{Overview} 
%%%%%%%%%%

Agile approaches and practices such as continuous integration and deployment promise to help reducing development time, to increase flexibility, and to generally shorten the feedback cycle time, which in turn can lead to a better management of complex system knowledge. 
However, the  complex supplier network (see the Ecosystem and Transparency viewpoint in Section~\ref{sec:ET_VP}),
and typical setup with a large number of ECUs,
pose specific challenges to %\chg{agile development methods, and specifically 
%with respect 
%to continuous integration and deployment of software.}{
these practices. %}.

First,  dependencies between ECUs raise multiple concerns,
regarding organization, versioning and testing:
(i)  organization -
%the question is related to who should be 
identifying the recipient
of a given software change; (ii)
 versioning -
the question is related to the compatibility of the software version of specific ECUs; and
% software
%that are compatible.
(iii)  testing -  %This then carries over to the testing effort,
compatible combinations need to be validated. 
Second, support for continuous deployment has to face safety concerns.
%relates to the connectivity of ECUs. 
%the extent of this, in turn, raises not least security concerns.
%\chg{Should, for instance,}
An example for this is the question on whether it should be possible, at runtime,
to modify or update
the software running on an ECU responsible for a safety critical function.
%\eric{does an ECU responsible actually deliver software? Seems weird, as this seems to imply responsibility for a specifc ECU, while the software should be hw independent. Should it be a ``Software Responsible'' instead?}
%\magnus{it's an ECU -- responsible for a safety critical\ldots not the role of ECU responsible\eric{I see. Next try then...}}
This might be desirable, e.g. when better algorithms for autonomous driving become available.

Dependencies between ECUs are a property of the architecture.
As mentioned, the emergent architecture may differ from the intended architecture,
and continuous integration and deployment of software may entail architectural changes.
This highlights both the need for collaboration % touches on the concern of the need for collaboration
between parts of the organization working on different architectural levels, and the need of a proper support
for agile and flexible architecting, 
%Furthermore, the architectural framework should support agile architecting.
%
%\rogardt{We need to decide quickly if we go for one or two viewpoints. This will have consequences on the Correspondence rules, to permit teams to work as independent as possible quite a bit of the architecture need to be done upfront and dependencies found, however too much upfront work will also hamper agility}
%\chg{Addressing these concerns suggests two architectural views and viewpoints}{ 
which raises two main concerns: %} %\eric{Should we then not split it into two viewpoints? Otherwise needs to be rephrased. I think we should have one viewpoint answering both questions.}
(i) one concerning architecture as an enabler
of continuous integration and deployment,
facilitating variant handling and coordination of updates, and
%
(ii) another concerning continuous integration and deployment
on the architecture level,
facilitating reasoning about modifications to the architecture itself.%\eric{In a way, continuous integration and deployment might also enable certain architecural decisions that otherwise might not be possible...}


%%%%%%%%%%
\subsection{Concerns and stakeholders} 
%%%%%%%%%%
\todo[inline, size=\footnotesize]{
Architects looking for an architecture viewpoint suitable for their
purposes often use the identified concerns and typical stakeholders to
guide them in their search.  Therefore it is important (and required
by the Standard) to document the concerns and stakeholders for which a
viewpoint is intended.
}
%%%%%%%%%%
\subsubsection{Concerns}\label{vp:concerns}
%%%%%%%%%%
In this section we focus on the concerns that are essential to enable continuous integration and deployment when engineering software for cars. 
We express the concerns in the form of questions as suggested by the ISO/IEC/IEEE 42010 standard.
%\eric{Suggesting some edits and a change in order here to improve the flow. Please check if it is still correct!}
%\magnus{Made some minor tweaks}
\begin{itemize}
\item \emph{How can we avoid building the wrong architecture?}
%\chg{It is no use in making a good architecture if it is for the wrong purpose.}
{Even a technically sound architecture is worthless if it does not fit the desired purpose.
Naturally, an electrical architecture needs to exist long before the product under consideration is developed and can reach the market, since it is required to guide organization of development activities. 
Even though a lot of information is needed early in the product lifecycle, a crucial amount of information will only become available during its development and early market phase.}
\item \emph{How can we reduce the number of architectural assumptions?} The more of the architecture 
%\chg{one make}
{is defined} up-front,  the more 
%\ins
{it is based on}
assumptions 
%\chg{are needed to be added}
{instead of facts}. 
%\del{It is hard to guarantee that these assumptions are true, moreover, if there are too many assumption one might loose the overviews as well. }\ins
{This can lead to late changes, duplicate work, or in the worst case to problems that only surface once the product is on the market.}
\item \emph{How can %\chg{we}
{a system} respond quicker to changes in the market?} 
%\ins
{Many of the assumptions will concern the market situation once the product has been released. 
Markets are however prone to constant change and such change might invalidate architectural decisions during the development of software. }
\item \emph{How can we deal with changing interfaces?} {Reacting to change in the market will lead to late changes of the architecture, which in turn will affect continuous integration and deployment: It is not possible to integrate a component into the system after its interface changes, unless the platform and other dependent components have already been adjusted to the new interface.
Deployment after an interface change might therefore involve a large number of updated components, which will not be possible in a continuous fashion. 
In other words, after an interface change it will be hard to guarantee that the main branch of all components is in a deployable state.}
\item \emph{How can we deal with dependencies?}
There are two types of dependencies, \chg{real}{structural} \rogardt{Don't know if real is the right word here}\eric{have not seen it used and cannot google. Reminds me about the discussion on the difference between complicated and complex, where complexity lies in the problem and complicatedness in imperfect implementation. Previously, we have discussed planned vs. unplanned, but that is very different. what could work instead of real could be \ins{domain and problem inherent dependencies} or \ins{structural dependencies}, i.e. those that you cannot remove with good architecture} and accidental dependencies.
An example for \chg{real}{structural} dependencies is the increasing degree to that cars become dependent on different sensors, which themselves are rapidly evolving.
%\del{As cars get more and more dependent on different sensors that also increases the dependencies in the overall system.} 
This type of dependencies needs to be {planned for during defining an architecture in order to allow } handling {them later}.  
In contrast, the risk of accidental dependencies\eric{should we give an example or definition?} should be systematically reduced and avoided.  
\ins{This is important, since} unresolved dependencies \ins{(both structural and accidental)} require additional coordination effort between teams, which effectively slows down or even hinders continuous integration and deployment.
%\chg{team to work in isolation}{separation of work between teams}.} 

%\rogardt{Dependencies seem to be one of the largest show stopper for agile development}
%\magnus{Are dependencies alone a strong enough culprit,
%or is it when combined with slow feedback/low automation that they cause trouble for CI\&D?}
%\eric{Agreed with Rogardt, same feeling as Magnus (but we just don't know yet?). Paragraph needed a bit more work, running it back by you. Is this text correct?}
%\rogardt{A team is working away, then it suddenly depend on another group to do some work, they will go to this group to ask them kindly to complete the job they depend on. However, this group has their own backlog and ask them even more kindly to come back in a few weeks after they have empty their own backlog:-}

\rogardt{Will we leave our issue on continuous deployment to a virtual car in this paper. I think this is quite an interesting issue, but I don't know completely how it fit in. One can view the virtual car as some short of executable architecture.A new type of architecture that permit continuous deployment of part of the architecture, controlled by the high level architecture (that is more agile). Maybe this is too hard to understand}
\eric{Let's keep that one for the next paper, okay? I agree it would fit, but it will be hard and perhaps it is even worth a paper of its own.}
\end{itemize}


%\must{Provide a listing of architecture-relevant concerns to be framed by
%this architecture viewpoint per \std{7a}.}

%Describe each concern.

%Concerns name ``areas of interest'' in a system.

%\note{Following ISO/IEC/IEEE 42010, \textbf{system} is a shorthand for
%  any number of things including man-made systems, software products
%  and services, and software-intensive systems such as ``individual
%  applications, systems in the traditional sense, subsystems, systems
%  of systems, product lines, product families, whole enterprises, and
%  other aggregations of interest''.}

%Concerns may be very general (e.g., \textit{Reliability}) or quite
%specific (\textit{e.g., How does the system handle network latency?}).
  
%Concerns identified in this section are critical information for an
%architect because they help her decide when this viewpoint will be
%useful.

%When used in an architecture description, the viewpoint becomes a
%``contract'' between the architect and stakeholders that these
%concerns will be addressed in the view resulting from this viewpoint.

%It can be helpful to express concerns \emph{in the form of questions}
%that views resulting from that viewpoint will be able to answer. E.g.,
%\begin{itemize}
%\item \textit{How does the system manage faults?}
%\item \textit{What services does the system provide?}
%\end{itemize}

%\note{``In the form of a question'' is inspired by the television quiz
%  show, \textit{Jeopardy!}}
 
%\std{5.3} contains a candidate list of concerns that must be considered
%when producing an architecture description. These can be considered
%here for their relevance to the viewpoint being specified:
%\begin{itemize}
%\item What are the purpose(s) of the system-of-interest?
%\item What is the suitability of the architecture for achieving the
%  system-of-interest's purpose(s)?
%\item How feasible is it to construct and deploy the
%  system-of-interest?
%\item What are the potential risks and impacts of the
%  system-of-interest to its stakeholders throughout its life cycle?
%\item How is the system-of-interest to be maintained and evolved?
%\end{itemize}

%See also: \std{4.2.3}.

%%%%%%%%%%
\subsubsection{Typical stakeholders} 
%%%%%%%%%%

For the continuous integration and deployment viewpoint, we need to consider all stakeholders described in Section~\ref{sec:VCGAF} and summarized in Table~\ref{tab:stakeholders}.
In addition, we need to consider suppliers, as continuous integration and deployment will depend on their deliveries (see also the ecosystem and transparency viewpoint in the next section).

%\todo[inline, size=\footnotesize]{
%\must{Provide a listing of the typical stakeholders of a system who
%  are in the potential audience for views of this kind, per \std{7b}.}
%
%Typical stakeholders would include those likely to read such views
%and/or those who need to use the results of this view for another
%task.
%
%Stakeholders to consider include:
%%\begin{itemize}
%
%~~--~users of a system; 
%
%~~--~operators of a system; 
%
%~~--~acquirers of a system;
%
%~~--~owners of a system; 
%
%~~--~suppliers of a system; 
%
%~~--~developers of a system; 
%
%~~--~builders of a system; 
%
%~~--~maintainers of a system.
%%\end{itemize}
%}

%%%%%%%%%%
%\subsubsection{``Anti-concerns'' \Optional}
%%%%%%%%%%%
%
%\del{It may be helpful to architects and stakeholders to
%document the kinds of issues for which this viewpoint is \emph{not
%  appropriate or not particularly useful}.}
%
%\del{Identifying the ``anti-concerns'' of a given notation or approach may
%be a good antidote for certain overly used models and notations.}
%\eric{I think we can remove this optional section here, no?}
%% \tbd{Examples!}



%%%%%%%%%%
\subsection{Model kinds for Continuous Integration and Deployment Viewpoint}\label{mk:list}
%%%%%%%%%%
\todo[inline, size=\footnotesize]{
\must{Identify each model kind used in the viewpoint per \std{7c}.}

In the Standard, each architecture view consists of multiple
architecture models. Each model is governed by a \textit{model kind}
which establishes the notations, conventions and rules for models of
that type.  See: \std{4.2.5, 5.5 and 5.6}.

Repeat the next section for each model kind listed here the viewpoint
being specified.}

Various notations have been proposed to model continuous integration and deployment pipelines \cite{SB2014}\eric{Could have more citations here based on CID SLR}.
Among those, several could be applicable to this viewpoint.
In addition, one would need to relate those to architectural approaches as well as to the concerns defined above. 
We are currently working on a qualitative assessment method that allows to identify disruptors and bottlenecks for continuous integration and deployment\eric{I am thinking about the KACI model here (not published yet). What else?}.

\rogardt{What about models to capture dependencies \eric{which? UML Package diagrams? Use Case diagrams? Feature diagrams?}}
\rogardt{Will a service oriented architecture improve CI\&D, however we might leave out this discussion \eric{agreed, I am not aware of any evidence on this.}}
% \eric{Several subsections commented out in accordance to Section 8 (I believe).}
%%%%%%%%%%
% \subsection{\Fillin{Model Kind Name}}\label{vp:mk}
%%%%%%%%%%

%\must{Identify the model kind.}


%%%%%%%%%%
%\subsubsection{\Fillin{Model Kind Name} conventions} 
%%%%%%%%%%

%\must{Describe the conventions for models of this kind.}

%Conventions include languages, notations, modeling techniques,
%analytical methods and other operations. These are key modeling
%resources that the model kind makes available to architects and
%determine the vocabularies for constructing models of the kind and
%therefore, how those models are interpreted and used.

%It can be useful to separate these conventions into a \emph{language
%  part}: in terms of a metamodel or specification of notation to be
%used and a \emph{process part}: to describe modeling techniques used
%to create the models and methods which can be used on the models that
%result.  These include operations on models of the model kind.

%The remainder of this section focuses on the language part. The next
%section focuses on the process part.

%The Standard does not prescribe \emph{how} modeling conventions are to
%be documented.  The conventions could be defined:
%\begin{description}
%\item[I)] by reference to an existing notation or language (such as
%  SADT, UML or an architecture description language such as ArchiMate
%  or SysML) or to an existing technique (such as $M/M/4$ queues);
%\item[II)] by presenting a metamodel defining its core constructs;
%\item[III)] via a template for users to fill in;
%\item[IV)] by some combination of these methods or in some other
%  manner.
%\end{description}

%Further guidance on methods I) through III) is provided below.
 
%Sometimes conventions are applicable across more than one model kind
%-- it is not necessary to provide a separate set of conventions, a
%metamodel, notations, or operations for each, when a single
%specification is adequate.


%%%%%%%%%%
%\subsubsection*{I) Model kind languages or notations \Optional}
%%%%%%%%%%

%Identify or define the notation used in models of the kind.

%Identify an existing notation or model language or define one that can
%be used for models of this model kind. Describe its syntax, semantics,
%tool support, as needed.


%%%%%%%%%%
%\subsubsection*{II) Model kind metamodel \Optional} 
%%%%%%%%%%

%A metamodel presents the AD elements that constitute the
%vocabulary of a model kind, and their rules of combination. There are
%different ways of representing metamodels (such as UML class diagrams, OWL,
%eCore). The metamodel should present:
%\begin{description}
%\item[entities] What are the major sorts of conceptual elements that
 % are present in models of this kind?
%\item[attributes] What properties do entities possess in models of
%  this kind?
%\item[relationships] What relations are defined among entities in
%  models of this kind?
%\item[constraints] What constraints are there on entities, attributes
%  and/or relationships and their combinations in models of this kind?
%\end{description}

%\note{Metamodel constraints should not be confused with architecture
%  constraints that apply to the subject being modeled, not the
%  notations used.}

%In the terms of the Standard, entities, attributes, relationships are
%\textit{AD elements} per \std{3.4, 4.2.5 and 5.7}.

%In the \textit{Views-and-Beyond} approach~\cite{DSA:2010}, \eric{reference missing?} each
%viewtype (which is similar to a viewpoint) is specified by a set of
%elements, properties, and relations (which correspond to entities,
%attributes and relationships here, respectively).

%When a viewpoint specifies multiple model kinds it can be useful to
%specify a single viewpoint metamodel unifying the definition of the
%model kinds and the expression of correspondence rules.  When defining
%an architecture framework, it may be helpful to use a single metamodel
%to express multiple, related viewpoints and model kinds.

% \tbd{EXAMPLE -- In \cite{Hilliard:1999} and earlier work, we said that
%   all views are built from primitives called components, connections
%   and constraints which basically gives views a graph structure with
%   components as nodes and two types of edges (connections and
%   constraints). There are two issues with this: (\textit{1})
%   components and \textit{connectors} have taken on a specialized
%   meaning from the work by CMU and others \cite{Shaw-Garlan:1996};
%   (\textit{2}) this ur-ontology may be over-commiting for some views.}


%%%%%%%%%%
%\subsubsection*{III) Model kind templates \Optional}
%%%%%%%%%%

%Provide a template or form specifying the format and/or content of
%models of this model kind.

%% \tbd{EXAMPLE} 


%%%%%%%%%%
%\subsubsection{\Fillin{Model Kind Name} operations \Optional} 
%%%%%%%%%%

%Specify operations defined on models of this kind.

%See~\S\ref{Opns} for further guidance.


%%%%%%%%%%
%\subsubsection{\Fillin{Model Kind Name} correspondence rules}
%%%%%%%%%%

%\must{Document any correspondence rules associated with the model
%  kind.}

%See~\S\ref{CRs} for further guidance.


%%%%%%%%%%
\subsection{Operations on views}\label{Opns}
%%%%%%%%%%

\eric{Cannot see how we could write something completely different from what is in Section 8.5. Made a suggestion for addition there. Perhaps copy that text here and reference it in Section 7.5 and 8.5?}

\ins{Finally, operations should be also provided to evaluate architecture descriptions and decisions. For instance, analysis methods might be defined to evaluate the efficiency of the feedback cycle in continuous integration and deployment as well as its ability to help resolving assumptions.}


\todo[inline, size=\footnotesize]{

Operations define the methods to be applied to views and their models.
Types of operations include:

%\begin{description}

%\item[construction methods] are the means by which views are  constructed under this viewpoint. 
% These operations could be in the  form of process guidance (how to start, what to do next); or work product guidance (templates for views of this type). 
% Construction techniques may also be heuristic: identifying styles, patterns, or other idioms to apply in the synthesis of the view.

% \item[interpretation methods] which guide readers to understanding and interpreting architecture views and their models.

% \item[analysis methods] are used to check, reason about, transform,  predict, and evaluate architectural results from this view, including operations which refer to model correspondence rules.

% \item[implementation methods] are the means by which to design and build systems using this view.

% \end{description}


\textbf{construction methods} are the means by which views are  constructed under this viewpoint. 
 These operations could be in the  form of process guidance (how to start, what to do next); or work product guidance (templates for views of this type). 
 Construction techniques may also be heuristic: identifying styles, patterns, or other idioms to apply in the synthesis of the view.

\textbf{interpretation methods} which guide readers to understanding and interpreting architecture views and their models.

\textbf{analysis methods} are used to check, reason about, transform,  predict, and evaluate architectural results from this view, including operations which refer to model correspondence rules.

\textbf{implementation methods} are the means by which to design and build systems using this view.


Another approach to categorizing operations is from Finkelstein et
al. \cite{Finkelstein+1992}. The \emph{work plan} for a viewpoint
defines 4 kinds of actions (on the view representations):
\textit{assembly actions} which contains the actions available to the
developer to build a specification; \textit{check actions} which
contains the actions available to the developer to check the
consistency of the specification; \textit{viewpoint actions} which
create new viewpoints as development proceeds; \textit{guide actions}
which provide the developer with guidance on what to do and when.
}


%%%%%%%%%%
\subsection{Correspondence rules}\label{CID-CRs}
%%%%%%%%%%
%\magnus{Can we get away without this section? How strictly should we fullfil the template?}

\ins{This viewpoint has correspondences with various other viewpoints, including functional safety, security, privacy, and the new viewpoint Ecosystem and Transparency proposed in this paper. 
The latter needs to be considered when continuous integration and deployment involves contributions from other actors in the ecosystem.}
%\must{Document any correspondence rules defined by this viewpoint or
%  its model kinds.}

%Usually, these rules will be across models or across views since,
%constraints within a model kind will have been specified as part of
%the conventions of that model kind.

%See: \std{4.2.6 and 5.7}

%%\tbd{examples or specs}

%%%%%%%%%%
%\subsection{Examples \Optional} 
%%%%%%%%%%

%Provide helpful examples of use of the viewpoint for the reader
%(architects and other stakeholders).


%%%%%%%%%%
%\subsection{Notes \Optional} 
%%%%%%%%%%

%Provide any additional information that users of the viewpoint may
%need or find helpful.


%%%%%%%%%%
%\subsection{Sources} 
%%%%%%%%%%

%\must{Identify sources for this architecture viewpoint, if any,
%  including author, history, bibliographic references, prior art, per
%  \std{7e}.}


% !TEX root = main.tex
%%% Version 2.1b %%%
\section{Ecosystem and transparency viewpoint}\label{sec:ET_VP}
\renewcommand{\Fillin}[1]{{Ecosystem and Transparency}}
%%%%%%%%%%
\subsection{\Fillin{Viewpoint Name}}\label{vp:eco}
%%%%%%%%%%

\todo[inline,size=\small]{Provide the name for the viewpoint.

If there are any synonyms or other common names by which this viewpoint is
known or used, record them here.}

The viewpoint ``\Fillin{Viewpoint Name}'' focuses on the value-chain of \ugh{logical components of the  architecture}\eric{is this a good way of describing what is delivered by the ecosystem? It should be some abstraction of hw, software, logical components, no?} and aims at answering the following questions:
\begin{enumerate}
\item How can an electrical architecture enable a network of organizations (including OEM, Tier-1 supplier, Tier-2 supplier) to work together to (continuously) provide value?
\item How are architectural decisions affected by the need for transparent and continuous collaboration within the value-chain?
\end{enumerate}
\eric{Really struggling with those questions...}

%%%%%%%%%%
\subsection{Overview} 
%%%%%%%%%%

\todo[inline,size=\small]{Provide an abstract or brief overview of the viewpoint. 

Describe the viewpoint's key features.}

The \Fillin{Viewpoint Name} viewpoint takes into account that architectural assets in automotive engineering are provided based on a complex network of cooperating organizations. 
A logical component in the architecture will be implemented by a variaty of physical components, including hardware (ECUs) and software on different levels of abstraction. 

In a concrete scenario, a logical component could be based on several AUTOSAR ECUs.
The hardware in this scenario would be provided by different Tier-1 suppliers, while the AUTOSAR layer would be provided by one or more certified AUTOSAR suppliers.
These AUTOSAR suppliers could be characterised as Tier-2 suppliers, since they are contracted by the Tier-1 suppliers who are supposed to deliver an ECU with basic software, but different setups are possible. 
On top of these ECUs, application software would be developed either by the OEM or by separate software suppliers in order to provide the services defined by a logical component. 

The example scenario above shows the importance of the \Fillin{Viewpoint Name} viewpoint: if an electrical architecture should take into account strategic business goals such as improved flexibility, reduced time-to-market, or development efficiency, the \Fillin{Viewpoint Name} viewpoint will offer important information that must be considered.
This viewpoint ties into business decisions and often long-running sourcing contracts, but also into new business models with respect to after market software updates.
Thus, we expect that lifecycles of hardware and software components are important architectural aspects as well as managing different development cycles of suppliers and components.


%%%%%%%%%%
\subsection{Concerns and stakeholders} 
%%%%%%%%%%

\todo[inline, size=\small]{Architects looking for an architecture viewpoint suitable for their
purposes often use the identified concerns and typical stakeholders to
guide them in their search.  Therefore it is important (and required
by the Standard) to document the concerns and stakeholders for which a
viewpoint is intended.}

%%%%%%%%%%
\subsubsection{Concerns}\label{vp:concerns}
%%%%%%%%%%
In this section we focus on the concerns that are essential to enable efficient work in the ecosystem of automotive electrical systems. 
We express the concerns in the form of questions as suggested by the ISO/IEC/IEEE 42010 standard.
\eric{Wonder if it is good or bad that all three viewpoints have very similar text here...}

\begin{itemize}
\item \emph{Which types of value-chains are implied by a given electrical architecture and what is their purpose?} A given electrical architecture will define the interplay of different types of logical components with the goal to support user visible features. 
While some of these features will be well-understood and stable, others will be highly innovative and their development will be subject to volatile requirements. 
Value-chains for implementing those logical components will differ based on the type of feature that they support. 

\item \emph{How to map supplier development capabilities to demands created by a specific electrical architecture?} Different capabilities are demanded for development across the value-chain, depending of the type of feature. 
An electrical architecture might be able to minimize or limit interaction on critical innovative features in the ecosystem to only actors that can support a highly iterative development. 

\item \emph{What is the suitability of the architecture for managing different kinds of value-chains?}

\item \emph{How can we establish the required level of transparency in a value-chain?} The more volatile the requirements of a feature are, the more information exchange is needed between ecosystem actors. 
If we were to implement continuous delivery throughout the value-chain, all involved partners should have access to critical information so that they can take responsibility. 
In a less volatile environment the (legal and organizational) effort of creating high transparency will not pay off. 

\item \emph{How can we manage transparency (e.g. of architectural decisions) in the face of changing suppliers?} In order to maintain a powerful automotive engineering ecosystem we need to be able to change actor relationships. This will also affect the level of transparency (including what information to share and at what frequency).
\end{itemize}

\todo[inline, size=\footnotesize]{\must{Provide a listing of architecture-relevant concerns to be framed by
this architecture viewpoint per \std{7a}.}

Describe each concern.

Concerns name ``areas of interest'' in a system.

\note{Following ISO/IEC/IEEE 42010, \textbf{system} is a shorthand for
  any number of things including man-made systems, software products
  and services, and software-intensive systems such as ``individual
  applications, systems in the traditional sense, subsystems, systems
  of systems, product lines, product families, whole enterprises, and
  other aggregations of interest''.}

Concerns may be very general (e.g., \textit{Reliability}) or quite
specific (\textit{e.g., How does the system handle network latency?}).
  
Concerns identified in this section are critical information for an
architect because they help her decide when this viewpoint will be
useful.

When used in an architecture description, the viewpoint becomes a
``contract'' between the architect and stakeholders that these
concerns will be addressed in the view resulting from this viewpoint.

It can be helpful to express concerns \emph{in the form of questions}
that views resulting from that viewpoint will be able to answer. E.g.,

~~--~\textit{How does the system manage faults?}

~~--~\textit{What services does the system provide?}


\note{``In the form of a question'' is inspired by the television quiz
  show, \textit{Jeopardy!}}
 
\std{5.3} contains a candidate list of concerns that must be considered
when producing an architecture description. These can be considered
here for their relevance to the viewpoint being specified:
%\begin{itemize}

~~--~ What are the purpose(s) of the system-of-interest?

~~--~What is the suitability of the architecture for achieving the
  system-of-interest's purpose(s)?

~~--~How feasible is it to construct and deploy the
  system-of-interest?

~~--~What are the potential risks and impacts of the
  system-of-interest to its stakeholders throughout its life cycle?

~~--~How is the system-of-interest to be maintained and evolved?

See also: \std{4.2.3}.}

%%%%%%%%%%
\subsubsection{Typical stakeholders} 
%%%%%%%%%%
For the ecosystem and transparency viewpoint, we need to consider stakeholders within the different ecosystem actors.
With respect to the OEM (the keystone actor in the automotive engineering ecosystem), potentially every stakeholder described in Section~\ref{sec:VCGAF} and summarized in Table~\ref{tab:stakeholders} will be part of the audience of this viewpoint. 

Other actors (such as Tier-1 and Tier-2 suppliers) will have very similar stakeholders that need to be considered. 
Those need to be taken into account, even when constructing this viewpoint strictly from the perspective of the OEM. 
For example for the concerns above, it is crucial to consider both managers and technical experts at suppliers.

\todo[inline, size=\footnotesize]{
\must{Provide a listing of the typical stakeholders of a system who
  are in the potential audience for views of this kind, per \std{7b}.}

Typical stakeholders would include those likely to read such views
and/or those who need to use the results of this view for another
task.

Stakeholders to consider include:
%\begin{itemize}

~~--~users of a system; 

~~--~operators of a system; 

~~--~acquirers of a system;

~~--~owners of a system; 

~~--~suppliers of a system; 

~~--~developers of a system; 

~~--~builders of a system; 

~~--~maintainers of a system.
%\end{itemize}
}
%%%%%%%%%%
%\subsubsection{``Anti-concerns'' \Optional} 
%%%%%%%%%%
%\todo[inline, size=\footnotesize]{
%It may be helpful to architects and stakeholders to
%document the kinds of issues for which this viewpoint is \emph{not
%  appropriate or not particularly useful}.

%Identifying the ``anti-concerns'' of a given notation or approach may
%be a good antidote for certain overly used models and notations.
%}
% \tbd{Examples!}



%%%%%%%%%%
\subsection{Model kinds+}\label{mk:list}
%%%%%%%%%%
\todo[inline, size=\footnotesize]{

\must{Identify each model kind used in the viewpoint per \std{7c}.}

In the Standard, each architecture view consists of multiple
architecture models. Each model is governed by a \textit{model kind}
which establishes the notations, conventions and rules for models of
that type.  See: \std{4.2.5, 5.5 and 5.6}.

Repeat the next section \eric{commented out} for each model kind listed here the viewpoint
being specified.

}
%%%%%%%%%%
%\subsection{\Fillin{Model Kind Name}}\label{vp:mk}
%%%%%%%%%%
%\todo[inline, size=\footnotesize]{

%\must{Identify the model kind.}

%}
%%%%%%%%%%
%\subsubsection{\Fillin{Model Kind Name} conventions} 
%%%%%%%%%%
%\todo[inline, size=\footnotesize]{

%\must{Describe the conventions for models of this kind.}

%Conventions include languages, notations, modeling techniques,
%analytical methods and other operations. These are key modeling
%resources that the model kind makes available to architects and
%determine the vocabularies for constructing models of the kind and
%therefore, how those models are interpreted and used.

%It can be useful to separate these conventions into a \emph{language
%  part}: in terms of a metamodel or specification of notation to be
%used and a \emph{process part}: to describe modeling techniques used
%to create the models and methods which can be used on the models that
%result.  These include operations on models of the model kind.

%The remainder of this section focuses on the language part. The next
%section focuses on the process part.

%The Standard does not prescribe \emph{how} modeling conventions are to
%be documented.  The conventions could be defined:
%\begin{description}

%\textbf{I)} by reference to an existing notation or language (such as
%  SADT, UML or an architecture description language such as ArchiMate
%  or SysML) or to an existing technique (such as $M/M/4$ queues);
  
%\textbf{II)} by presenting a metamodel defining its core constructs;

%\textbf{III)} via a template for users to fill in;

%\textbf{IV)} by some combination of these methods or in some other
 % manner.
%\end{description}

%Further guidance on methods I) through III) is provided below.
 
%Sometimes conventions are applicable across more than one model kind
%-- it is not necessary to provide a separate set of conventions, a
%metamodel, notations, or operations for each, when a single
%specification is adequate.
%}


%%%%%%%%%%
%\subsubsection*{I) Model kind languages or notations \Optional}
%%%%%%%%%%
%\todo[inline, size=\footnotesize]{

%Identify or define the notation used in models of the kind.

%Identify an existing notation or model language or define one that can
%be used for models of this model kind. Describe its syntax, semantics,
%tool support, as needed.
%}
%\eric{seems like we will have to be a bit soft on how to do this. One could use goal models to define win-win situations. One could use process models to describe development cycles. What are good models to describe component lifecycles? Probably a template would be good at this point of time.}

Since the we are proposing a new viewpoint for Ecosystem and Transperency concerns in this paper, there is no established way of modelling it so far. 
Models would need to provide the following information:
\begin{description}
\item[Analysis of Value-Chain:] In order to document the different value-chains, on could either create a map of a (part of an) ecosystem as proposed by Janssen et al. \cite{Jansen2012b} or single out a specific value chain as we for example did in our previous work on the AUTOSAR ecosystem \cite{Soltani2015a} based on Boucharas et al.'s notation \cite{BJB2009}.
\item[Analysis of Actors and Goals:] In addition to the critical value chains, one should maintain information about the actors involved in those chains and their goals.
Yu and Franch have proposed to use goal models for this purpose, which is a promising approach to align goals of different actors based on architectural decisions and to achieve win-win situations. These are important when targeting strong collaborations and partnerships \cite{FSY2015}\eric{might need to find a better reference}.
\item[Reasoning about anonymous actors:] To some extend, the concrete partners and suppliers are not known when creating the electrical architecture. 
In such situations, we propose a persona based approach, i.e. defining archetypical actors with important characteristics as well as their expectation towards the ecosystem \cite{Knauss2014c,Hammouda2015}.
\item[Defining Transparency Requirements:] Hussaini et al. have proposed a framework for defining transparency requirements which could be a good starting point for these aspects \cite{HSP+2016}\eric{I believe they have a full paper at ASE or so, but cannot google}. 
For practical use in this viewpoint, this framework would need to be extended with dynamic aspects, e.g. for withdrawing information when a partnership ends. 
Obviously, this includes also ownership and digital rights management issues, for which we hope to borrow concepts from other works \cite{Muller,Averbakh2014}.
\end{description}
\eric{obviously need to add some details and references here...}

%%%%%%%%%%
%\subsubsection*{II) Model kind metamodel \Optional} 
%%%%%%%%%%
%\todo[inline, size=\footnotesize]{

%A metamodel presents the AD elements that constitute the
%vocabulary of a model kind, and their rules of combination. There are
%different ways of representing metamodels (such as UML class diagrams, OWL,
%eCore). The metamodel should present:
%\begin{description}
%\item[entities] What are the major sorts of conceptual elements that  are present in models of this kind?
% \item[attributes] What properties do entities possess in models of this kind?
% \item[relationships] What relations are defined among entities in models of this kind?
% \item[constraints] What constraints are there on entities, attributes  and/or relationships and their combinations in models of this kind?
%\end{description}

%\textbf{entities} What are the major sorts of conceptual elements that
%  are present in models of this kind?
  
%\textbf{attributes} What properties do entities possess in models of
%  this kind?
  
%\textbf{relationships} What relations are defined among entities in
%  models of this kind?
  
%\textbf{constraints} What constraints are there on entities, attributes
%  and/or relationships and their combinations in models of this kind?


%\note{Metamodel constraints should not be confused with architecture
%  constraints that apply to the subject being modeled, not the
 % notations used.}

%In the terms of the Standard, entities, attributes, relationships are
%\textit{AD elements} per \std{3.4, 4.2.5 and 5.7}.

%In the \textit{Views-and-Beyond} approach~\cite{DSA:2010}, each
%viewtype (which is similar to a viewpoint) is specified by a set of
%elements, properties, and relations (which correspond to entities,
%attributes and relationships here, respectively).

%When a viewpoint specifies multiple model kinds it can be useful to
%specify a single viewpoint metamodel unifying the definition of the
%model kinds and the expression of correspondence rules.  When defining
%an architecture framework, it may be helpful to use a single metamodel
%to express multiple, related viewpoints and model kinds.
%}
% \tbd{EXAMPLE -- In \cite{Hilliard:1999} and earlier work, we said that
%   all views are built from primitives called components, connections
%   and constraints which basically gives views a graph structure with
%   components as nodes and two types of edges (connections and
%   constraints). There are two issues with this: (\textit{1})
%   components and \textit{connectors} have taken on a specialized
%   meaning from the work by CMU and others \cite{Shaw-Garlan:1996};
%   (\textit{2}) this ur-ontology may be over-commiting for some views.}

%\eric{seems like we can anticipate a few entities and their relationshipts that will be important in this viewpoint, but much is left for future work.}

%%%%%%%%%%
%\subsubsection*{III) Model kind templates \Optional}
%%%%%%%%%%
%\todo[inline, size=\footnotesize]{

%Provide a template or form specifying the format and/or content of
%models of this model kind.
%}
%% \tbd{EXAMPLE} 


%%%%%%%%%%
%\subsubsection{\Fillin{Model Kind Name} operations \Optional} 
%%%%%%%%%%
%\todo[inline, size=\footnotesize]{

%Specify operations defined on models of this kind.

%See~\S\ref{Opns} for further guidance.

%}

%
%%%%%%%%%%
%\subsubsection{\Fillin{Model Kind Name} correspondence rules}
%%%%%%%%%%
%\todo[inline, size=\footnotesize]{

%\must{Document any correspondence rules associated with the model
%  kind.}

%See~\S\ref{CRs} for further guidance.
%}

%\eric{no clue what to write here.}

%%%%%%%%%%
\subsection{Operations on views}\label{Opns}
%%%%%%%%%%
\eric{Cannot see how we could write something completely different from what is in Section 8.5. Made a suggestion for addition there. Perhaps copy that text here and reference it in Section 7.5 and 8.5?}


\eric{Suggestions: ``identify win-win situations'', identify ``development lifecycle compatibility'', Define transparency needs and collaboration patterns}

\todo[inline, size=\footnotesize]{

Operations define the methods to be applied to views and their models.
Types of operations include:

%\begin{description}

%\item[construction methods] are the means by which views are  constructed under this viewpoint. 
% These operations could be in the  form of process guidance (how to start, what to do next); or work product guidance (templates for views of this type). 
% Construction techniques may also be heuristic: identifying styles, patterns, or other idioms to apply in the synthesis of the view.

% \item[interpretation methods] which guide readers to understanding and interpreting architecture views and their models.

% \item[analysis methods] are used to check, reason about, transform,  predict, and evaluate architectural results from this view, including operations which refer to model correspondence rules.

% \item[implementation methods] are the means by which to design and build systems using this view.

% \end{description}


\textbf{construction methods} are the means by which views are  constructed under this viewpoint. 
 These operations could be in the  form of process guidance (how to start, what to do next); or work product guidance (templates for views of this type). 
 Construction techniques may also be heuristic: identifying styles, patterns, or other idioms to apply in the synthesis of the view.

\textbf{interpretation methods} which guide readers to understanding and interpreting architecture views and their models.

\textbf{analysis methods} are used to check, reason about, transform,  predict, and evaluate architectural results from this view, including operations which refer to model correspondence rules.

\textbf{implementation methods} are the means by which to design and build systems using this view.


Another approach to categorizing operations is from Finkelstein et
al. \cite{Finkelstein+1992}. The \emph{work plan} for a viewpoint
defines 4 kinds of actions (on the view representations):
\textit{assembly actions} which contains the actions available to the
developer to build a specification; \textit{check actions} which
contains the actions available to the developer to check the
consistency of the specification; \textit{viewpoint actions} which
create new viewpoints as development proceeds; \textit{guide actions}
which provide the developer with guidance on what to do and when.
}


%%%%%%%%%%
\subsection{Correspondence rules}\label{CRs}
%%%%%%%%%%
\todo[inline, size=\footnotesize]{

\must{Document any correspondence rules defined by this viewpoint or
  its model kinds.}

Usually, these rules will be across models or across views since,
constraints within a model kind will have been specified as part of
the conventions of that model kind.

See: \std{4.2.6 and 5.7}
}
\eric{What ever could that be? Obviously, Continuous Integration and Deployment, since if we want to do that, the syncing of development cycles becomes important. Also Systems of Systems. But in what way? Perhaps that is rather something for discussion!}
%%\tbd{examples or specs}



%%%%%%%%%%
%\subsection{Examples \Optional} 
%%%%%%%%%%
%\todo[inline, size=\footnotesize]{

%Provide helpful examples of use of the viewpoint for the reader
%(architects and other stakeholders).
%}
%\eric{Future work}

%%%%%%%%%%
%\subsection{Notes \Optional} 
%%%%%%%%%%
%\todo[inline, size=\footnotesize]{

%Provide any additional information that users of the viewpoint may
%need or find helpful.

%}
%\eric{future work}

%%%%%%%%%%
%\subsection{Sources} 
%%%%%%%%%%
%\todo[inline, size=\footnotesize]{

%\must{Identify sources for this architecture viewpoint, if any,
%  including author, history, bibliographic references, prior art, per
%  \std{7e}.}
%}
%\eric{future work?!?}

%%% Version 2.1b %%%
\section{System of Systems viewpoint: vehicle point of view}
\renewcommand{\Fillin}[1]{{System of Systems: vehicle point of view}}
%%%%%%%%%%
\subsection{\Fillin{Viewpoint Name}}\label{vp:template}
%%%%%%%%%%

\must{Provide the name for the viewpoint.}

If there are any synonyms or other common names by which this viewpoint is
known or used, record them here.


%%%%%%%%%%
\subsection{Overview} 
%%%%%%%%%%
Connected cars will benefit from Intelligent Transport Systems (ITS), Smart Cities and IoT to provide new application scenarios like smart traffic control,  smart platooning coordination, collective collision avoidance, etc. 
Vehicles will combine data collected through its sensors %from the inside vehicle 
with external data coming from the environment, e.g. other vehicles, road, cloud, etc. %In such scenario, different applications will be possible: smart traffic control, better platooning coordination and enhanced safety in general. 
%While solving congestion is undoubtedly beneficial, safety remains the top priority for OEMs.

%In connected vehicles, safety aspects become more complex. 
Connected vehicles will face new challenges and opportunities related to safety issues.
%On the one hand, c
Current %While there are clear 
regulations for safety aspects, like the ISO 26262 standard, do not account for scenarios in which the %provide a clear regulation for  
% regarding isolated car system (seat belts, airbags etc..) like in the safety standard ISO 26262, there is a lack of 
%safety requirements for 
%systems of connected cars. The 
vehicle is part of a more complex system; the challenge is on how to manage new hazards that coming from the environment could jeopardize safety. 
%The core ECUs of the car (in charge of braking, steering, engine control, etc.) are part of a closed system in which all the safety requirements can be certified at design time. Exposing these crucial components to the outside environment would invalidates the assumptions that have been made so far. In fact, manufactures have preferred to limit the interfaces of the communication infrastructure due to the enormous hazard potential. In a connected world,  this is no longer possible. Sharing and controlling sensitive information could be crucial in dangerous situations. 
%On the other hand, c
At the same time, connected cars open new opportunities for safety, called ``connected safety" within Volvo Cars\footnote{\url{https://goo.gl/mIWWS3}}: %({\small \url{https://goo.gl/mIWWS3}}):
e.g., this new technology will allow a connected car to be aware of a slippery road, of cyclists on the road, etc., so to initiate all the actions needed to avoid accidents and collisions. 
%
%after it has been informed by another car and the information has been propagated via the Volvo Cloud. In such scenario, the informed car could automatically slow down reducing the risk for potential accidents. 

These scenarios are posing new requirements to the architecture. %To deal with similar scenarios, changes need to be made in the architecture. 
We foreseen two different viewpoints and views: (i) %one viewpoint and view showing the architecture 
from the point of view of a single connected car, which has to offer the functionalities needed to realize the scenario, and (ii) %another viewpoint and view representing the 
from the point of view of the system of systems  (i.e. cars connected with other entities of their environment), spanning from the agreements between the different systems affected, e.g. different OEMs, cloud providers, road infrastructures, etc., to the definition of the scenario that the system of systems has to realize, like the slippery road mentioned above. 

% agreements between the different systems affected. In order to achieve the proper behavior of such system different stakeholders must be involved. From tier 1 and tier 2 suppliers to the cloud provider, they all have to agree on common interfaces and safety guarantees that must be respected.

%The core ECUs of the car (in charge of braking, steering, engine control, etc.) are part of a closed system in which all the safety requirements can be certified at design time. Exposing these crucial components to the outside environment would invalidates the assumptions that have been made so far. In fact, manufactures have preferred to limit the interfaces of the communication infrastructure due to the enormous hazard potential. In a connected world,  this is no longer possible. Sharing and controlling sensitive information could be crucial in dangerous situations. 

%Until lately, safety has been dedicated in surviving crashes, with the emerging connected vehicle it will be about avoiding them. For example, the new technologies developed by Volvo allow a connected car to be aware of a slippery road after it has been informed by another car and the information has been propagated via the Volvo Cloud. In such scenario, the informed car could automatically slow down reducing the risk for potential accidents.

%To accomplish similar scenarios, radical changes need to be made in the architecture descriptions and agreements between the different systems affected. In order to achieve the proper behavior of such system different stakeholders must be involved. From tier 1 and tier 2 suppliers to the cloud provider, they all have to agree on common interfaces and safety guarantees that must be respected.

%<<<<<<< HEAD
In general, the main issue with safety guarantees in connected cars is that a full analysis of the system is not possible at design time. When moving from a single vehicle to a cooperative system, a new safety analysis is required to handle uncertainties at runtime. Some approaches have been proposed to deal with certification at runtime, e.g.~\cite{runtime1, runtime3}, but a clear framework that can be used to define the connected safety requirements is still missing.
%=======
%The main issue with safety guarantees in connected cars, is that a full analysis of the system is not possible at design time. When moving from a single vehicle to a cooperative system, a new safety analysis is required to handle the uncertainties at runtime. Therefore, in order to benefit from the collaborative system, is necessary to move some of the safety analysis from design time to runtime. Some approaches have been proposed to deals with certification of the safety guarantees at runtime \cite{runtime1}, \cite{runtime2}, \cite{runtime3}, but there is still not a clear framework utilized to define the connected safety requirements.
%>>>>>>> 4ffb434a25eb0042a3685e7d03d264e7c0eb62b1



%%%%%%%%%%
\subsection{Concerns and stakeholders} 
%%%%%%%%%%

Architects looking for an architecture viewpoint suitable for their
purposes often use the identified concerns and typical stakeholders to
guide them in their search.  Therefore it is important (and required
by the Standard) to document the concerns and stakeholders for which a
viewpoint is intended.

%%%%%%%%%%
\subsubsection{Concerns}\label{vp:concerns}
%%%%%%%%%%

\must{Provide a listing of architecture-relevant concerns to be framed by
this architecture viewpoint per \std{7a}.}

Describe each concern.

Concerns name ``areas of interest'' in a system.

\note{Following ISO/IEC/IEEE 42010, \textbf{system} is a shorthand for
  any number of things including man-made systems, software products
  and services, and software-intensive systems such as ``individual
  applications, systems in the traditional sense, subsystems, systems
  of systems, product lines, product families, whole enterprises, and
  other aggregations of interest''.}

Concerns may be very general (e.g., \textit{Reliability}) or quite
specific (\textit{e.g., How does the system handle network latency?}).
  
Concerns identified in this section are critical information for an
architect because they help her decide when this viewpoint will be
useful.

When used in an architecture description, the viewpoint becomes a
``contract'' between the architect and stakeholders that these
concerns will be addressed in the view resulting from this viewpoint.

It can be helpful to express concerns \emph{in the form of questions}
that views resulting from that viewpoint will be able to answer. E.g.,
\begin{itemize}
\item \textit{How does the system manage faults?}
\item \textit{What services does the system provide?}
\end{itemize}

\note{``In the form of a question'' is inspired by the television quiz
  show, \textit{Jeopardy!}}
 
\std{5.3} contains a candidate list of concerns that must be considered
when producing an architecture description. These can be considered
here for their relevance to the viewpoint being specified:
\begin{itemize}
\item What are the purpose(s) of the system-of-interest?
\item What is the suitability of the architecture for achieving the
  system-of-interest's purpose(s)?
\item How feasible is it to construct and deploy the
  system-of-interest?
\item What are the potential risks and impacts of the
  system-of-interest to its stakeholders throughout its life cycle?
\item How is the system-of-interest to be maintained and evolved?
\end{itemize}

See also: \std{4.2.3}.

%%%%%%%%%%
\subsubsection{Typical stakeholders} 
%%%%%%%%%%

\must{Provide a listing of the typical stakeholders of a system who
  are in the potential audience for views of this kind, per \std{7b}.}

Typical stakeholders would include those likely to read such views
and/or those who need to use the results of this view for another
task.

Stakeholders to consider include:
\begin{itemize}
\item users of a system; 
\item operators of a system; 
\item acquirers of a system;
\item owners of a system; 
\item suppliers of a system; 
\item developers of a system; 
\item builders of a system; 
\item maintainers of a system.
\end{itemize}

%%%%%%%%%%
\subsubsection{``Anti-concerns'' \Optional} 
%%%%%%%%%%

It may be helpful to architects and stakeholders to
document the kinds of issues for which this viewpoint is \emph{not
  appropriate or not particularly useful}.

Identifying the ``anti-concerns'' of a given notation or approach may
be a good antidote for certain overly used models and notations.

% \tbd{Examples!}



%%%%%%%%%%
\subsection{Model kinds+}\label{mk:list}
%%%%%%%%%%

\must{Identify each model kind used in the viewpoint per \std{7c}.}

In the Standard, each architecture view consists of multiple
architecture models. Each model is governed by a \textit{model kind}
which establishes the notations, conventions and rules for models of
that type.  See: \std{4.2.5, 5.5 and 5.6}.

Repeat the next section for each model kind listed here the viewpoint
being specified.


%%%%%%%%%%
\subsection{\Fillin{Model Kind Name}}\label{vp:mk}
%%%%%%%%%%

\must{Identify the model kind.}


%%%%%%%%%%
\subsubsection{\Fillin{Model Kind Name} conventions} 
%%%%%%%%%%

\must{Describe the conventions for models of this kind.}

Conventions include languages, notations, modeling techniques,
analytical methods and other operations. These are key modeling
resources that the model kind makes available to architects and
determine the vocabularies for constructing models of the kind and
therefore, how those models are interpreted and used.

It can be useful to separate these conventions into a \emph{language
  part}: in terms of a metamodel or specification of notation to be
used and a \emph{process part}: to describe modeling techniques used
to create the models and methods which can be used on the models that
result.  These include operations on models of the model kind.

The remainder of this section focuses on the language part. The next
section focuses on the process part.

The Standard does not prescribe \emph{how} modeling conventions are to
be documented.  The conventions could be defined:
\begin{description}
\item[I)] by reference to an existing notation or language (such as
  SADT, UML or an architecture description language such as ArchiMate
  or SysML) or to an existing technique (such as $M/M/4$ queues);
\item[II)] by presenting a metamodel defining its core constructs;
\item[III)] via a template for users to fill in;
\item[IV)] by some combination of these methods or in some other
  manner.
\end{description}

Further guidance on methods I) through III) is provided below.
 
Sometimes conventions are applicable across more than one model kind
-- it is not necessary to provide a separate set of conventions, a
metamodel, notations, or operations for each, when a single
specification is adequate.


%%%%%%%%%%
\subsubsection*{I) Model kind languages or notations \Optional}
%%%%%%%%%%

Identify or define the notation used in models of the kind.

Identify an existing notation or model language or define one that can
be used for models of this model kind. Describe its syntax, semantics,
tool support, as needed.


%%%%%%%%%%
\subsubsection*{II) Model kind metamodel \Optional} 
%%%%%%%%%%

A metamodel presents the AD elements that constitute the
vocabulary of a model kind, and their rules of combination. There are
different ways of representing metamodels (such as UML class diagrams, OWL,
eCore). The metamodel should present:
\begin{description}
\item[entities] What are the major sorts of conceptual elements that
  are present in models of this kind?
\item[attributes] What properties do entities possess in models of
  this kind?
\item[relationships] What relations are defined among entities in
  models of this kind?
\item[constraints] What constraints are there on entities, attributes
  and/or relationships and their combinations in models of this kind?
\end{description}

\note{Metamodel constraints should not be confused with architecture
  constraints that apply to the subject being modeled, not the
  notations used.}

In the terms of the Standard, entities, attributes, relationships are
\textit{AD elements} per \std{3.4, 4.2.5 and 5.7}.

In the \textit{Views-and-Beyond} approach~\cite{DSA:2010}, each
viewtype (which is similar to a viewpoint) is specified by a set of
elements, properties, and relations (which correspond to entities,
attributes and relationships here, respectively).

When a viewpoint specifies multiple model kinds it can be useful to
specify a single viewpoint metamodel unifying the definition of the
model kinds and the expression of correspondence rules.  When defining
an architecture framework, it may be helpful to use a single metamodel
to express multiple, related viewpoints and model kinds.

% \tbd{EXAMPLE -- In \cite{Hilliard:1999} and earlier work, we said that
%   all views are built from primitives called components, connections
%   and constraints which basically gives views a graph structure with
%   components as nodes and two types of edges (connections and
%   constraints). There are two issues with this: (\textit{1})
%   components and \textit{connectors} have taken on a specialized
%   meaning from the work by CMU and others \cite{Shaw-Garlan:1996};
%   (\textit{2}) this ur-ontology may be over-commiting for some views.}


%%%%%%%%%%
\subsubsection*{III) Model kind templates \Optional}
%%%%%%%%%%

Provide a template or form specifying the format and/or content of
models of this model kind.

%% \tbd{EXAMPLE} 


%%%%%%%%%%
\subsubsection{\Fillin{Model Kind Name} operations \Optional} 
%%%%%%%%%%

Specify operations defined on models of this kind.

See~\S\ref{Opns} for further guidance.


%%%%%%%%%%
\subsubsection{\Fillin{Model Kind Name} correspondence rules}
%%%%%%%%%%

\must{Document any correspondence rules associated with the model
  kind.}

See~\S\ref{CRs} for further guidance.


%%%%%%%%%%
\subsection{Operations on views}\label{Opns}
%%%%%%%%%%

Operations define the methods to be applied to views and their models.
Types of operations include:

\begin{description}

\item[construction methods] are the means by which views are
  constructed under this viewpoint. These operations could be in the
  form of process guidance (how to start, what to do next); or work
  product guidance (templates for views of this type). Construction
  techniques may also be heuristic: identifying styles, patterns, or
  other idioms to apply in the synthesis of the view.

\item[interpretation methods] which guide readers to understanding
  and interpreting architecture views and their models.

\item[analysis methods] are used to check, reason about, transform,
  predict, and evaluate architectural results from this view,
  including operations which refer to model correspondence rules.

\item[implementation methods] are the means by which to design and
  build systems using this view.

\end{description}

Another approach to categorizing operations is from Finkelstein et
al. \cite{Finkelstein+1992}. The \emph{work plan} for a viewpoint
defines 4 kinds of actions (on the view representations):
\textit{assembly actions} which contains the actions available to the
developer to build a specification; \textit{check actions} which
contains the actions available to the developer to check the
consistency of the specification; \textit{viewpoint actions} which
create new viewpoints as development proceeds; \textit{guide actions}
which provide the developer with guidance on what to do and when.


%%%%%%%%%%
\subsection{Correspondence rules}\label{CRs}
%%%%%%%%%%

\must{Document any correspondence rules defined by this viewpoint or
  its model kinds.}

Usually, these rules will be across models or across views since,
constraints within a model kind will have been specified as part of
the conventions of that model kind.

See: \std{4.2.6 and 5.7}

%%\tbd{examples or specs}

%%%%%%%%%%
\subsection{Examples \Optional} 
%%%%%%%%%%

Provide helpful examples of use of the viewpoint for the reader
(architects and other stakeholders).


%%%%%%%%%%
\subsection{Notes \Optional} 
%%%%%%%%%%

Provide any additional information that users of the viewpoint may
need or find helpful.


%%%%%%%%%%
\subsection{Sources} 
%%%%%%%%%%

\must{Identify sources for this architecture viewpoint, if any,
  including author, history, bibliographic references, prior art, per
  \std{7e}.}



%\section{Continuous integration and deployment viewpoint}\label{sec:CIDviewpoint}
%\patrizio{Magnus, Eric, Rogardt}
%
%\subsection{Overview}
%Agile approaches and practices such as continuous integration and deployment promise to help reducing development time, to increase flexibility, and to generally shorten the feedback cycle time. However, 
%the  complex supplier network,
%and typical setup with a large number of ECUs,
%pose specific challenges to %\chg{agile development methods, and specifically 
%%with respect 
%%to continuous integration and deployment of software.}{
%these practices. %}.
%
%First,  dependencies between ECUs raise multiple concerns,
%regarding organization, versioning and testing:
%(i)  organization -
%%the question is related to who should be 
%identifying the recipient
%of a given software change; (ii)
% versioning -
%the question is related to the compatibility of the software version of specific ECUs; and
%% software
%%that are compatible.
%(iii)  testing -  %This then carries over to the testing effort,
%compatible combinations need to be validated. 
%Second, support for continuous deployment has to face with safety concerns.
%%relates to the connectivity of ECUs. 
%%the extent of this, in turn, raises not least security concerns.
%Should, for instance, the software of an ECU responsible for a safety critical function
%be modifiable at runtime?
%
%Dependencies between ECUs are a property of the architecture.
%As mentioned, the emergent architecture may differ from the intended architecture,
%and continuous integration and deployment of software may entail architectural changes.
%This highlights both the need for collaboration % touches on the concern of the need for collaboration
%between parts of the organization working on different architectural levels, and the need of a proper support
%for agile and flexible architecting.
%%Furthermore, the architectural framework should support agile architecting.
%
%Addressing these concerns suggests
%two architectural views and viewpoints:
%(i) one covering architecture as an enabler
%of continuous integration and deployment,
%facilitating variant handling and coordination of updates, and
%%
%(ii) another considering continuous integration and deployment
%on the architecture level,
%facilitating reasoning about modifications to the architecture itself.
%
%\subsection{Concerns}
%\patrizio{A listing of the architecture-related concerns framed by this viewpoint. This is
%crucial information for the Architect, because it helps her decide whether this
%viewpoint will be useful to apply to a given system of interest, and to
%communicate with its stakeholders.}
%
%\subsection{Anti-concerns}
%\patrizio{Optional. It can be useful to document the kinds of issues a viewpoint is not
%appropriate for. Articulating anti‐concerns may be a good antidote for certain
%overly used notations.}
%
%\subsection{Typical stakeholders}
%\patrizio{Optional. The typical audiences for views prepared using this viewpoint. Who
%are the usual stakeholders for this kind of view?}
%
%\subsection{Model languages}
%\patrizio{For each type of model used, describe the language or modeling techniques to
%be used. Each model language is a key modeling resource that the viewpoint
%makes available. Model languages provide the vocabularies for constructing the
%view. ISO/IEC 42010 does not specify how a modeling language is documented.
%It could be by reference to an existing modeling language (e.g., EAST-ADL or UML)
%or technique (e.g., M/M/4 queues); by providing a metamodel for the language
%to define the language's core constructs; via a template that users fill in; or by
%some combination of these methods.}
%
%\subsection{Model correspondence rules}
%\patrizio{The viewpoint may specify model correspondence rules. Each one may be
%documented here.}
%
%\subsection{Operations on views}
%\patrizio{Operations define the methods which may be applied to views and their
%models. Operations can be divided into categories: Creation methods are the
%means by which views are prepared using the viewpoint. These could be in the
%form of process guidance (how to start, what to do next); or work product
%guidance (templates for views of this type); heuristics, styles, patterns, or other
%idioms. Interpretive methods provide the means by which views are to be
%understood by readers and system stakeholders. Analysis methods are used to
%check, reason about, transform, predict, apply and evaluate architectural
%results from this view. Implementation methods capture how to realize or
%construct systems using information from this view.}
%
%\subsection{Examples}
%\patrizio{Optional. This section provides examples for the reader.}
%
%\subsection{Sources}
%\patrizio{What are the sources for this viewpoint, if any? This may include author,
%history, literature references, prior art, etc.}
%
%\section{System of Systems: vehicle point of view}\label{sec:SoSviewpoint}
%\patrizio{Piergiuseppe, Patrizio.}
%
%\subsection{Overview}
%Connected cars will benefit from Intelligent Transport Systems (ITS), Smart Cities and IoT to provide new application scenarios like smart traffic control,  smart platooning coordination, collective collision avoidance, etc. 
%Vehicles will combine data collected through its sensors %from the inside vehicle 
%with external data coming from the environment, e.g. other vehicles, road, cloud, etc. %In such scenario, different applications will be possible: smart traffic control, better platooning coordination and enhanced safety in general. 
%%While solving congestion is undoubtedly beneficial, safety remains the top priority for OEMs.
%
%%In connected vehicles, safety aspects become more complex. 
%Connected vehicles will face new challenges and opportunities related to safety issues.
%%On the one hand, c
%Current %While there are clear 
%regulations for safety aspects, like the ISO 26262 standard, do not account for scenarios in which the %provide a clear regulation for  
%% regarding isolated car system (seat belts, airbags etc..) like in the safety standard ISO 26262, there is a lack of 
%%safety requirements for 
%%systems of connected cars. The 
%vehicle is part of a more complex system; the challenge is on how to manage new hazards that coming from the environment could jeopardize safety. 
%%The core ECUs of the car (in charge of braking, steering, engine control, etc.) are part of a closed system in which all the safety requirements can be certified at design time. Exposing these crucial components to the outside environment would invalidates the assumptions that have been made so far. In fact, manufactures have preferred to limit the interfaces of the communication infrastructure due to the enormous hazard potential. In a connected world,  this is no longer possible. Sharing and controlling sensitive information could be crucial in dangerous situations. 
%%On the other hand, c
%At the same time, connected cars open new opportunities for safety, called ``connected safety" within Volvo Cars\footnote{\url{https://goo.gl/mIWWS3}}: %({\small \url{https://goo.gl/mIWWS3}}):
%e.g., this new technology will allow a connected car to be aware of a slippery road, of cyclists on the road, etc., so to initiate all the actions needed to avoid accidents and collisions. 
%%
%%after it has been informed by another car and the information has been propagated via the Volvo Cloud. In such scenario, the informed car could automatically slow down reducing the risk for potential accidents. 
%
%These scenarios are posing new requirements to the architecture. %To deal with similar scenarios, changes need to be made in the architecture. 
%We foreseen two different viewpoints and views: (i) %one viewpoint and view showing the architecture 
%from the point of view of a single connected car, which has to offer the functionalities needed to realize the scenario, and (ii) %another viewpoint and view representing the 
%from the point of view of the system of systems  (i.e. cars connected with other entities of their environment), spanning from the agreements between the different systems affected, e.g. different OEMs, cloud providers, road infrastructures, etc., to the definition of the scenario that the system of systems has to realize, like the slippery road mentioned above. 
%
%% agreements between the different systems affected. In order to achieve the proper behavior of such system different stakeholders must be involved. From tier 1 and tier 2 suppliers to the cloud provider, they all have to agree on common interfaces and safety guarantees that must be respected.
%
%%The core ECUs of the car (in charge of braking, steering, engine control, etc.) are part of a closed system in which all the safety requirements can be certified at design time. Exposing these crucial components to the outside environment would invalidates the assumptions that have been made so far. In fact, manufactures have preferred to limit the interfaces of the communication infrastructure due to the enormous hazard potential. In a connected world,  this is no longer possible. Sharing and controlling sensitive information could be crucial in dangerous situations. 
%
%%Until lately, safety has been dedicated in surviving crashes, with the emerging connected vehicle it will be about avoiding them. For example, the new technologies developed by Volvo allow a connected car to be aware of a slippery road after it has been informed by another car and the information has been propagated via the Volvo Cloud. In such scenario, the informed car could automatically slow down reducing the risk for potential accidents.
%
%%To accomplish similar scenarios, radical changes need to be made in the architecture descriptions and agreements between the different systems affected. In order to achieve the proper behavior of such system different stakeholders must be involved. From tier 1 and tier 2 suppliers to the cloud provider, they all have to agree on common interfaces and safety guarantees that must be respected.
%
%%<<<<<<< HEAD
%In general, the main issue with safety guarantees in connected cars is that a full analysis of the system is not possible at design time. When moving from a single vehicle to a cooperative system, a new safety analysis is required to handle uncertainties at runtime. Some approaches have been proposed to deal with certification at runtime, e.g.~\cite{runtime1, runtime3}, but a clear framework that can be used to define the connected safety requirements is still missing.
%%=======
%%The main issue with safety guarantees in connected cars, is that a full analysis of the system is not possible at design time. When moving from a single vehicle to a cooperative system, a new safety analysis is required to handle the uncertainties at runtime. Therefore, in order to benefit from the collaborative system, is necessary to move some of the safety analysis from design time to runtime. Some approaches have been proposed to deals with certification of the safety guarantees at runtime \cite{runtime1}, \cite{runtime2}, \cite{runtime3}, but there is still not a clear framework utilized to define the connected safety requirements.
%%>>>>>>> 4ffb434a25eb0042a3685e7d03d264e7c0eb62b1
%
%\subsection{Concerns}
%\patrizio{A listing of the architecture-related concerns framed by this viewpoint. This is
%crucial information for the Architect, because it helps her decide whether this
%viewpoint will be useful to apply to a given system of interest, and to
%communicate with its stakeholders.}
%
%\subsection{Anti-concerns}
%\patrizio{Optional. It can be useful to document the kinds of issues a viewpoint is not
%appropriate for. Articulating anti‐concerns may be a good antidote for certain
%overly used notations.}
%
%\subsection{Typical stakeholders}
%\patrizio{Optional. The typical audiences for views prepared using this viewpoint. Who
%are the usual stakeholders for this kind of view?}
%
%\subsection{Model languages}
%\patrizio{For each type of model used, describe the language or modeling techniques to
%be used. Each model language is a key modeling resource that the viewpoint
%makes available. Model languages provide the vocabularies for constructing the
%view. ISO/IEC 42010 does not specify how a modeling language is documented.
%It could be by reference to an existing modeling language (e.g., EAST-ADL or UML)
%or technique (e.g., M/M/4 queues); by providing a metamodel for the language
%to define the language's core constructs; via a template that users fill in; or by
%some combination of these methods.}
%
%\subsection{Model correspondence rules}
%\patrizio{The viewpoint may specify model correspondence rules. Each one may be
%documented here.}
%
%\subsection{Operations on views}
%\patrizio{Operations define the methods which may be applied to views and their
%models. Operations can be divided into categories: Creation methods are the
%means by which views are prepared using the viewpoint. These could be in the
%form of process guidance (how to start, what to do next); or work product
%guidance (templates for views of this type); heuristics, styles, patterns, or other
%idioms. Interpretive methods provide the means by which views are to be
%understood by readers and system stakeholders. Analysis methods are used to
%check, reason about, transform, predict, apply and evaluate architectural
%results from this view. Implementation methods capture how to realize or
%construct systems using information from this view.}
%
%\subsection{Examples}
%\patrizio{Optional. This section provides examples for the reader.}
%
%\subsection{Sources}
%\patrizio{What are the sources for this viewpoint, if any? This may include author,
%history, literature references, prior art, etc.}

%\section{Ecosystem and transparency viewpoint}
%\patrizio{Magnus, Eric, Rogardt}
%
%\subsection{Overview}
%
%\subsection{Concerns}
%\patrizio{A listing of the architecture-related concerns framed by this viewpoint. This is
%crucial information for the Architect, because it helps her decide whether this
%viewpoint will be useful to apply to a given system of interest, and to
%communicate with its stakeholders.}
%
%\subsection{Anti-concerns}
%\patrizio{Optional. It can be useful to document the kinds of issues a viewpoint is not
%appropriate for. Articulating anti‐concerns may be a good antidote for certain
%overly used notations.}
%
%\subsection{Typical stakeholders}
%\patrizio{Optional. The typical audiences for views prepared using this viewpoint. Who
%are the usual stakeholders for this kind of view?}
%
%\subsection{Model languages}
%\patrizio{For each type of model used, describe the language or modeling techniques to
%be used. Each model language is a key modeling resource that the viewpoint
%makes available. Model languages provide the vocabularies for constructing the
%view. ISO/IEC 42010 does not specify how a modeling language is documented.
%It could be by reference to an existing modeling language (e.g., EAST-ADL or UML)
%or technique (e.g., M/M/4 queues); by providing a metamodel for the language
%to define the language's core constructs; via a template that users fill in; or by
%some combination of these methods.}
%
%\subsection{Model correspondence rules}
%\patrizio{The viewpoint may specify model correspondence rules. Each one may be
%documented here.}
%
%\subsection{Operations on views}
%\patrizio{Operations define the methods which may be applied to views and their
%models. Operations can be divided into categories: Creation methods are the
%means by which views are prepared using the viewpoint. These could be in the
%form of process guidance (how to start, what to do next); or work product
%guidance (templates for views of this type); heuristics, styles, patterns, or other
%idioms. Interpretive methods provide the means by which views are to be
%understood by readers and system stakeholders. Analysis methods are used to
%check, reason about, transform, predict, apply and evaluate architectural
%results from this view. Implementation methods capture how to realize or
%construct systems using information from this view.}
%
%\subsection{Examples}
%\patrizio{Optional. This section provides examples for the reader.}
%
%\subsection{Sources}
%\patrizio{What are the sources for this viewpoint, if any? This may include author,
%history, literature references, prior art, etc.}

% !TEX root = main.tex
\section{Discussion and Concluding remarks}\label{sec:conclusion}

In this paper we describe the current evaluation of Volvo Cars towards the definition of an architecture framework.
In this stage we are identifying potential viewpoints of the vehicles of the near future based on our shared experience in the automotive domain as well as a series of workshops for identification and validation for important concepts with a wide range of domain experts. 
Specifically, we present challenging scenarios about future demands related to architecture and proposed three new viewpoints (Continuous Integration and Deployment, Ecosystem and Transparency, System of System).
For each of these viewpoints, we started to collect actual needs and consumers of their respective views. 
Based on these, future work needs to identify the most suitable modeling languages to be used to describe the architecture. 
For these modeling languages, it is important to find the right tradeoff between a language's
(i) ability to support the level of formality and precision required by disciplined development processes, or
(ii) simplicity and it being intuitive enough to communicate the right message to stakeholders and to promote collaboration~\cite{whatindustrywants,IEEESoftwarePatrizio}.

For what concerns the realization of the architecture framework we will investigate MEGAF~\cite{MEGAF2010,MEGAF2012}, 
%
%MEGAF has been conceived with the aim of overcoming limitations of existing architecture frameworks. Most of the architecture framework in use are
%closed (i.e. they cannot be easily extended or adapted to new needs),
%%and the construction of existing ones requires a complete rework since reuse is not supported at all.
%and tool support for architecture frameworks exhibits this limitation
%as well: automated support, where it exists, follows the predefined
%viewpoints and models; support for developing extensions of
%architecture frameworks in terms of new viewpoints is non-existent.
%
%MEGAF is 
which is an infrastructure that enables software architects to realize their
own architecture frameworks specialized for a particular
application domain or community of stakeholders and compliant to the ISO/IEC/IEEE 42010 standard~\cite{42010}. 
%It offers
%an effective way of managing, storing, retrieving, and combining
%existing viewpoints, by properly selecting and reusing
%models previously defined and resident in MEGAF. 
%Architecture
%frameworks that can be realized by using MEGAF
%conform to ISO/IEC/IEEE 42010~\cite{42010}. 
%Once an architecture
%framework has been defined within MEGAF, it can be used
%to create architecture descriptions of systems-of-interest.
%%Architecture descriptions are easily made to adhere to the
%%architecture framework used to realize them. For instance,
%%the realized architecture models adhere to the conventions of
%%their model kinds as defined by the architecture framework
%%and its viewpoints.
%Moreover, the MEGAF infrastructure enables the architect
%to express and enforce relations both between various
%elements within an architecture description and across architecture
%descriptions. %Further details about MEGAF\footnote{\url{http://megaf.di.univaq.it/}} might be found at~\cite{MEGAF2010,MEGAF2012}. 

\section*{Acknowledgement}

We would like to thank Rich Hilliard for his precious comments and support. We would like to thanks also the consortium of the NGEA project for the valuable input that allowed the definition of the architecture framework.


%% References
%%
%% Following citation commands can be used in the body text:
%% Usage of \cite is as follows:
%%   \cite{key}         ==>>  [#]
%%   \cite[chap. 2]{key} ==>> [#, chap. 2]
%%

%% References with bibTeX database:

\bibliographystyle{elsarticle-num}
% \bibliographystyle{elsarticle-harv}
% \bibliographystyle{elsarticle-num-names}
% \bibliographystyle{model1a-num-names}
% \bibliographystyle{model1b-num-names}
% \bibliographystyle{model1c-num-names}
% \bibliographystyle{model1-num-names}
% \bibliographystyle{model2-names}
% \bibliographystyle{model3a-num-names}
% \bibliographystyle{model3-num-names}
% \bibliographystyle{model4-names}
% \bibliographystyle{model5-names}
% \bibliographystyle{model6-num-names}

\bibliography{biblio}


\end{document}

%%
%% End of file `elsarticle-template-num.tex'.
